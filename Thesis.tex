%% ----------------------------------------------------------------
%% Thesis.tex -- MAIN FILE (the one that you compile with LaTeX)
%% ---------------------------------------------------------------- 

% Set up the document
\documentclass[a4paper, 11pt, oneside]{Thesis}  % Use the "Thesis" style, based on the ECS Thesis style by Steve Gunn
\graphicspath{Figures/}  % Location of the graphics files (set up for graphics to be in PDF format)
\usepackage[T1]{fontenc}
\usepackage{helvet}
\renewcommand{\familydefault}{\sfdefault}
% Include any extra LaTeX packages required
\usepackage[square, numbers, comma, sort]{natbib}  % Use the "Natbib" style for the references in the Bibliography
\usepackage{verbatim}  % Needed for the "comment" environment to make LaTeX comments
\usepackage{vector}  % Allows "\bvec{}" and "\buvec{}" for "blackboard" style bold vectors in maths
\hypersetup{urlcolor=black, colorlinks=false}  
\usepackage{caption} 
\usepackage{subcaption} % This relies on commenting out line 111 in Thesis.cls to remove "subfigure"
\usepackage{datetime}
\usepackage{lscape}
\usepackage{array} % for the \newcolumntype facility used for our custom column type
\usepackage{multirow} % to allow spanning rows and columns in tables
\usepackage[a-1b]{pdfx} % For PDF/A generation
\usepackage{hyperref} % For PDF/A generation


% Suppress warnings about Underfull or Overfull boxes
\hbadness=99999

% This allows code formatting blocks through the 'listings' package
\usepackage{listings}
\lstset{frame=tb,
  language=Java,
  aboveskip=6mm,
  belowskip=6mm,
  frame=none,
  showstringspaces=false,
  columns=flexible,
  basicstyle={\small\ttfamily},
  numbers=none,
  % numberstyle=\tiny\color{gray},
  % keywordstyle=\color{blue},
  % commentstyle=\color{dkgreen},
  % stringstyle=\color{mauve},
  breaklines=true,
  breakatwhitespace=true,
  tabsize=3
}

\newcolumntype{P}[1]{>{\centering\arraybackslash\hspace{0pt}}p{#1}}

%% ----------------------------------------------------------------
\begin{document}
\frontmatter      % Begin Roman style (i, ii, iii, iv...) page numbering

% Set up the Title Page
\title  {Addressing Energy Efficiency in System Design: A Journey from Architecture to Operation}
\authors  {\texorpdfstring
            {\href{u1157744@uel.ac.uk}{Eoin Woods}}
            {Eoin Woods}
            }
\addresses  {\groupname\\\deptname\\\univname}  % Do not change this here, instead these must be set in the "Thesis.cls" file, please look through it instead
\date       {\today~(\currenttime)}
\subject    {}
\keywords   {}

\maketitle
%% ----------------------------------------------------------------

\pagestyle{plain}  % Set the page headers to the "fancy" style

% The Abstract Page
\addtotoc{Abstract}  % Add the "Abstract" page entry to the Contents
\abstract{
\setstretch{1.1}  % Set to small inter-line spacing to get abstract on one page

Digital-transformation initiatives have led to major efficiencies and cost savings, but at the cost of consuming nearly 10 percent of the world's electricity.  Researchers have been studying IT energy consumption to increase datacentre, network, and hardware efficiency and as a result, datacentre energy efficiency has improved considerably. Individual hardware devices have experienced a similar trend and computations per joule of energy have doubled every 1.5 years over the past two decades. 

However a neglected aspect of energy efficiency is the energy consumption of the software applications that provide the internet-connected services underpinning digital transformation.  So far we have largely failed to provide software architects with knowledge, guidance and tools to allow them to understand the energy properties of their systems.

The research reported in this thesis starts to address this situation by developing practical knowledge, techniques and tools to allow software architects to play their part in controlling the energy consumption of our modern digital world.

The work starts with a comprehensive survey of architectural description languages, a technology that could provide the basis of tools for addressing the energy consumption of complex systems.  During a major project to create a large industrial architectural description, existing research ADLs did not prove to be successful and instead a lightweight custom notation was created for the project.

Our attention then turns to helping architects to prioritise energy efficiency and we perform a study to investigate how experienced architects focus their attention for maximum effectiveness.  This leads to the development of a model to guide architecture practitioners which we validate and refine through a large survey of practicing software architects.

We then examined the energy-related guidance available to architects and found little generally applicable information in the literature.  This led us to identify three energy-related architectural principles from analysis of a successful industrial case study.

Finally, we turned to the practical problem of assisting architects to understand the runtime energy properties of their systems and designed a novel approach to estimate the energy consumption of workload scenarios via application execution tracing and a cost-based energy model.  We created a proof of concept implementation of the approach and validated its consistency and correctness through practical testing.

The result of this work is a collection of contributions to the fields of software architecture and energy efficiency that can help architecture practitioners to treat energy efficiency as a first class architectural concern in their work.

\clearpage  % Abstract ended, start a new page
%% ----------------------------------------------------------------

% Declaration Page required for the Thesis, your institution may give you a different text to place here
%\Declaration{
%
%\addtocontents{toc}{\vspace{1em}}  % Add a gap in the Contents, for aesthetics
%
%I, AUTHOR NAME, declare that this thesis titled, `THESIS TITLE' and the work presented in it are my own. I confirm that:
%
%\begin{itemize} 
%\item[\tiny{$\blacksquare$}] This work was done wholly or mainly while in candidature for a research degree at this University.
% 
%\item[\tiny{$\blacksquare$}] Where any part of this thesis has previously been submitted for a degree or any other qualification at this University or any other institution, this has been clearly stated.
% 
%\item[\tiny{$\blacksquare$}] Where I have consulted the published work of others, this is always clearly attributed.
% 
%\item[\tiny{$\blacksquare$}] Where I have quoted from the work of others, the source is always given. With the exception of such quotations, this thesis is entirely my own work.
% 
%\item[\tiny{$\blacksquare$}] I have acknowledged all main sources of help.
% 
%\item[\tiny{$\blacksquare$}] Where the thesis is based on work done by myself jointly with others, I have made clear exactly what was done by others and what I have contributed myself.
%\\
%\end{itemize}
% 
% 
%Signed:\\
%\rule[1em]{25em}{0.5pt}  % This prints a line for the signature
% 
%Date:\\
%\rule[1em]{25em}{0.5pt}  % This prints a line to write the date
%}
%\clearpage  % Declaration ended, now start a new page

%% ----------------------------------------------------------------
\lhead{\emph{Contents}}  % Set the left side page header to "Contents"
\tableofcontents  % Write out the Table of Contents

%% ----------------------------------------------------------------
\lhead{\emph{List of Figures}}  % Set the left side page header to "List if Figures"
\listoffigures  % Write out the List of Figures

%% ----------------------------------------------------------------
\lhead{\emph{List of Tables}}  % Set the left side page header to "List of Tables"
\listoftables  % Write out the List of Tables

%% ----------------------------------------------------------------
% The Acknowledgements page, for thanking everyone
\acknowledgements{
%\addtocontents{toc}{\vspace{1em}}  % Add a gap in the Contents, for aesthetics

I need to thank a number of people here \ldots

}
\clearpage  % End of the Acknowledgements

%% ----------------------------------------------------------------
% End of the pre-able, contents and lists of things
% Begin the Dedication page
\pagestyle{empty}  % Page style needs to be empty for this page
\dedicatory{For/Dedicated to/To my\ldots}

\addtocontents{toc}{\vspace{2em}}  % Add a gap in the Contents, for aesthetics


%% ----------------------------------------------------------------
\setstretch{1.5}  % Return the line spacing back to 1.5
\mainmatter	  % Begin normal, numeric (1,2,3...) page numbering
\pagestyle{plain}  % Return the page headers back to the "fancy" style

% Include the chapters of the thesis, as separate files
% Just uncomment the lines as you write the chapters

\chapter{Introduction}
\label{chapter:introduction}

\section{Context and Motivation}

The IT industry and the penetration of IT services into the life of most people has entered a new era. The number of devices connected to the internet has been growing steadily.  Cisco estimated that by 2008 the number of internet-connected devices had exceeded the number of people in the world and that by 2011 the internet usage of 20 typical households was generating more internet traffic than the world's entire  internet use in 2008 \cite{evans2008-iotinfo}. Partly due to this network capacity, we are witnessing a parallel growth in data, driven by more affordable storage systems and  new applications of the internet including mobile applications, IoT, social media, and smart cities.  This combination of data and connectivity has resulted in so-called "digital transformation" in many industries, leading to a huge ecosystem of software applications, from business analytics of customer behaviour to mobile apps that can allow a farmer to monitor the climatic and ground conditions in their fields in real time through local sensor systems. These new applications rely on being internet-connected, which results in constantly increasing demand for private and "cloud" data centre capacity.  This is why large Tech companies and major enterprises are continuously expanding their computing capabilities. 

Currently, data centres consume a substantial amount of energy and are thought to produce more greenhouse gas emissions than the entire aviation sector. In 2013, data centres in the U.S. alone consumed an estimated 91 billion kWh of electricity, and this is expected to rise to 140 billion kWh recent survey of data centre managers showed that energy efficiency is seen as their single biggest challenge \cite{cainc2016-ausdcenergy}. This is particularly important for colocation and managed service data centre providers who are based within or close to large cities where grid power availability is limited.

So, can the data centre community evolve to cope with this ever-increasing demand and support the world's relentless digital transformation? We cannot be sure, but it is clear that the challenges that it poses are widely understood.  A large number of industrial, governmental and academic research programmes \cite{loken2010-scinet, bahsoon2010-cloudpower, greengrid2011-dcefficiency, dc4cities2014_dcmetrics,ida2015-gdcip} have been investigating and continue to explore topics that can help, from new cooling technologies to more energy-efficient servers and building designs in the physical domain, from runtime workload consolidation to energy consumption monitoring in the software domain and even economic models for balancing system quality properties and power consumption from cross-disciplinary research. However, the software architecture community has been slower to recognise their potential contribution and to mobilise to meet this challenge. Addressing energy efficiency at the architecture level is still far from being mainstream. Can we continue designing systems without any consideration of their energy and power efficiency and let others worry about running them in an energy-efficient way? Should energy efficiency be a bolt-on system property or a quality attribute that is addressed at design time?

We believe that software architects may not be prioritising energy efficiency for a number of reasons.

Firstly, we currently have very little understanding of the impact of design decisions on energy efficiency or an understanding of how it affects other system qualities such as user experience, reliability, and performance.  Without this knowledge, it is difficult to perform trade-off analysis to understand the benefit or cost of improving energy efficiency. Minor changes to the system design could yield substantial benefits, such as avoiding unnecessary component redundancy or eliminating low-priority housekeeping tasks that prevent equipment from entering lower-power states. However, a lack of relevant design tools and frameworks mean that it is still difficult to understand the energy characteristics of software at an architectural level, let alone understand or test the energy implications of design decisions. 

Second, in order to achieve the next order of magnitude in energy efficiency, we need to think outside traditional design boundaries. This will require people from different specialisations and departments to prioritise energy efficiency as a design goal and to work together. This can often be difficult to achieve, given current organisational governance structures, with different teams sometimes having competing objectives, not to mention human dynamics and political barriers.

Finally, with the exception of mobile applications, where battery life is a visible reminder of the need for energy efficiency, energy rarely features as a high priority requirement or concern for system acquirers or end-users. On one hand, there is the problem of split incentives where operators of systems (e.g. administrators or data centre managers) do not pay the energy bill (which usually comes from the facilities budget). This means that they would see very little return from any savings made from energy efficiency. On the other hand, while energy costs can be anywhere from 25\% to 60\% of total data centre operating cost \cite{techuk2013-dcpower}, this is often a relatively small percentage of an organisation's overall spending. So, when cost savings are required, it may be easier to achieve them by reducing cost in other areas.

We believe that this situation can be addressed by creating tools and guidance which is aimed specifically at software architects and system stakeholders, to allow them to understand the energy efficiency implications of architectural design decisions.  The work reported in this thesis is a first step on the road to provide this.

\section{Objectives and Research Questions}

The objective of this research is to improve the assistance available to software architects to help them understand the impact of their work on the energy efficiency of the systems that they design.  This process started in the area of architectural description languages (ADLs) because the unambiguous description of the architecture that they provide offered the potential to provide the architect with visibility of the energy implications of their design decisions.

To reach our objective, this work aims to answer four specific research questions, namely:
\nopagebreak
\begin{description}
\item [RQ1] What architecture description languages exist and can they be used to reason about the energy properties of a system?
\item [RQ2] How can architects prioritise energy efficiency as an architectural concern?
\item [RQ3] What design guidelines can we provide to assist architects to improve the energy efficiency of their systems?
\item [RQ4] How can we make architects aware of the runtime energy characteristics of their systems?
\end{description}

Answering these questions has involved the investigation of the use of ADLs for large-scale architectural description, the identification of practical advice for how architects can focus attention on critical topics (such as energy efficiency), the identification of design principles to guide energy-efficient architectures, and the creation of a practical approach for estimating the energy usage of request processing in distributed applications.

\section{Research Methodology}

The research reported in this thesis comprises four areas of investigation, aligned with the four research questions defined above.  Each area of investigation has utilised different combinations of research techniques, which are outlined below.  Fuller discussion of the research methdology for each piece of work is provided in the relevant chapters of the thesis.

\subsection{Architectural Description Languages Investigation}

The ADL investigation work began with a systematic literature review \cite{kitchenham2007-slr}, which was presented in \sref{sec:adl-lit-review}, with the goal of systematically identifying, analysing and understanding the work that had been done to date in creating and applying architectural description languages.  

The second stage of this research was to undertake a practical exercise to apply an ADL from the research results to the description of a large industrial system to understand how practical and effective this would be.

This research enabled us to answer research question RQ1.

\subsection{Study of Prioritising Architectural Effort}

Our study of architectural effort prioritisation began with a narrative literature review to survey the state of research in this area \cite{baumeister1997-narrativereviews}, which is described in \sref{section:litreview-prioritisation}.  The literature review did not identify a satisfactory approach to the problem of prioritising architectural effort, so we undertook the process of creating one.  This a was a four-stage process involving surveys, model building and validation.  Full details are provided in \cref{chapter:prioritisation}, but in summary, the four stages of research were:
\nopagebreak
\begin{description}
	\item [Stage 1] gathering primary data using semi-structured interviews with practitioners, using a written introduction to the question we wanted to answer and then some specific questions to illustrate our area of interest. 
	\item [Stage 2] analysis of the primary data and creation of a preliminary model through a simple application of Grounded Theory \cite{charmaz2006-groundedtheory}.
	\item [Stage 3] validation of the preliminary model via a structured online questionnaire \cite{gillham2000-questionnaire}, completed by practitioners in relevant architecture roles (primarily software, solution and enterprise architects).
	\item [Stage 4] analysis of the validation data and refinement of the preliminary model into a final, validated model.
\end{description}

To ensure practitioner involvement in the research, we used the LinkedIn and Twitter social media networks to publicise and engage with architecture practitioners and to report preliminary results.

The result of this research was a validated model to guide the prioritisation of architectural effort, which allowed us to answer research question RQ2.

\subsection{Development of Energy Efficiency Design Principles}

This aspect of the research was undertaken to answer research question RQ3 by attempting to identify energy efficiency related design guidance for practicing software architects.

The work began with a narrative literature review to survey the state of research in this area, described in section \sref{section:litreview-energyguidance}, which identified some useful ideas in the research literature, but relatively little design guidance relevant to the software architecture practitioner.

To identify some initial design guidance we decided to study an industrial situation that had been successful in reducing energy consumption and found a case study from a major internet firm that had managed to reduce energy consumption considerably \cite{ebay2013-digitalefficiency}.  We analysed this scenario and synthesised and captured key principles from it.

This enabled us to answer research question RQ3.

\subsection{Design and Implementation of Application Energy Monitoring}

This section of the research aimed to design and create a proof-of-concept implementation of a technical solution to the problem of providing architects with visibility of the energy consumption of their systems.  This was an Improvement Problem to Develop a Useful Artefact in TAR terminology, which involved problem investigation, the design of the solution (a piece of software) and the validation of it using realistic testing.  We excluded the further TAR steps of \emph{real-world implementation} and \emph{validation} \cite{wieringa2012-tar} from the scope of the work reported in this thesis due to the scale of the work required.

The work began with a narrative literature review to survey the state of research in this area, reported in \sref{sec:litreviewenergy}.  This exercise discovered that some application energy measurement systems have been designed and reported in the literature, but most are not available for general use.  The other problem with the measurement systems found was that they do not allow the architect to understand the energy consumption of application execution scenarios, just of application components over a period of time.  From the architect's perspective, this limits the value of the information the tools produce and means that the tools are difficult to use in production environments with a synthetic workload.

In response to these limitations, an approach to capturing representative energy usage for application execution scenarios was designed in a largely technology independent manner, as reported in \cref{chapter:monitoring}.  Our solution to the problem approaches it in a slightly different way to the existing systems and allocates estimated server energy usage to the applications running on that server, rather than trying to calculate an absolute energy consumption for each.  This sidesteps a number of problems with the existing approaches, as we will illustrate later.  We also performed the measurements and calculation from the perspective of the architect, using execution scenarios (tracing requests through the distributed application components), rather than measuring the energy usage of operating system processes. This design was then implemented using a specific set of technologies, including Linux, Docker and Java, as described in \cref{chapter:implementation}.  Once implemented, a set of practical tests was designed and executed to validate the implementation's internal consistency and external correctness with respect to its runtime environment, as described in \cref{chapter:validation}.

This enabled us to answer research question RQ4.

\section{Contribution}

The research described in this thesis has contributed elements to software architecture research and energy efficiency research and in particular, has contributed to bringing the two closer together.
\pagebreak
The specific research contributions in this thesis are as follows: \nopagebreak
\begin{enumerate}
	\item \textbf{A comprehensive systematic survey of architecture description languages}.  This literature review surveyed all ADLs created from 1991 to 2015, reviewing 135 potential languages, with 51 of them being confirmed as architectural description languages, according to our criteria.  A detailed characteristation of the languages and the field were produced from this analysis.
	\item \textbf{A case study of practical ADL use at scale in industry}.  Having surveyed the ADLs, we considered how to apply them to a real problem for a real stakeholder for a large industrial system.  This experience led to the academic ADLs being abandoned and a simple, lightweight, specialised notation to describe the architectural style used in the system being used instead.  A set of lessons learned and constructive suggestions for future ADL research were derived from the experience.
	\item \textbf{A model for architectural effort prioritisation}.  We interviewed expert practitioners and created a model for effort prioritisation based on common approaches that they (unknowingly) shared.  This model was then validated and refined using the results of a survey of 84 software architecture practitioners from across the world.
	\item \textbf{Principles for energy-efficient architectural design}.  Having found no energy specific architecture principles and relatively few generally applicable energy efficiency tactics in the literature, we identified a small set of principles from a successful industrial case study that reduced energy consumption for application services.
	\item \textbf{A system for allocating energy to application scenarios}.  Our literature review revealed that a number of application energy estimation systems had been proposed and prototyped.  However, all of these systems measure the energy consumption of operating system processes, which makes the information of limited immediate value to the application architect.  Our contribution has been to design and create a proof-of-concept implementation of a system which estimates the energy consumption of execution scenarios through the application (using application tracing), so providing the architect with significantly more insight into the effect of their architectural decisions. As far as we know, this scenario based energy usage calculation has not been attempted before in prior research.
\end{enumerate}

Much of the work reported here has been previously published in conference proceedings and journals. The publications arising from this work are listed below: \nopagebreak
\begin{description}
	\item Woods, Eoin, and Bashroush, Rabih. "Using an Architecture Description Language to Model a Large-Scale Information System - An Industrial Experience Report." In \emph{Software Architecture (WICSA) and European Conference on Software Architecture (ECSA), 2012 Joint Working IEEE/IFIP Conference on}. IEEE, 2012.
	\item Woods, Eoin, and Bashroush, Rabih. "Modelling Large-Scale Information Systems Using ADLs - An Industrial Experience Report." \emph{Journal of Systems and Software} 99 (2015).
	\item Bashroush, Rabih, Woods, Eoin, and Noureddine, Adel. "Data Center Energy Demand: What Got Us Here Won't Get Us There." \emph{IEEE Software} 33, no. 2 (2016).
	\item Bashroush, Rabih, and Woods, Eoin. "Architectural Principles for Energy-Aware Internet-Scale Applications." \emph{IEEE Software} 34, no. 3 (2017).
	\item Woods, Eoin, and Bashroush, Rabih. "A Model for Prioritization of Software Architecture Effort." \emph{European Conference on Software Architecture}. Springer, Cham, 2017.
	\item Woods, Eoin, and Bashroush, Rabih. "How Software Architects Focus Their Attention." \emph{Journal of Systems and Software}.  Submitted.
\end{description}


\section{Structure of Thesis}

This thesis is structured into 9 chapters, each presenting a specific aspect of the research work.  The structure of the thesis is illustrated by the diagram in \fref{figure:intro-chapters}.

\begin{figure}[h]
\centering
\includegraphics[width=1.0\textwidth]{Figures/intro-chapters}
\caption{Structure of the Thesis}
\label{figure:intro-chapters}
\end{figure}

\emph{Chapter 1} is this introductory chapter, setting the motivation and context for the work, defining the research questions, explaining the research approach and explaining the structure of the thesis.

\emph{Chapter 2} contains a literature review, structured into four parts, exploring the research literature in the areas of architectural description languages (ADLs), how architects should prioritise their focus for maximum effectiveness, design guidelines for energy efficiency and runtime energy consumption for IT systems.

\emph{Chapter 3} explores research question RQ1 and discusses how ADLs can be used to describe large-scale software systems and presents a significant industrial case study that explored how effective this was in practice.

\emph{Chapter 4} investigates research question RQ2 and explores the area of prioritisation of work from an architect's perspective, asking how architects prioritise their time for maximum effectiveness and presenting the results of an industrial survey into the approaches used by experienced practitioners and a validated model to guide architects, based on the insights from the survey.

\emph{Chapter 5} explores energy related design principles to answer research question RQ3 and presents a small set of heuristic design principles for designing energy-efficient software applications, derived from the experience of an industrial case study in reducing energy consumption through architectural change.

\emph{Chapter 6} addresses the fundamentals of research question RQ4 by asking how application energy usage can be monitored and estimated during the operation of a system and presents a theoretical model for solving this problem.

\emph {Chapter 7} addresses the practical aspects of research question RQ4 and presents a proof-of-concept implementation of the energy estimation model, specialised for estimating energy usage of a group of microservices processing incoming requests.

\emph{Chapter 8} validates the energy estimation approach as part of answering research question RQ4 and presents the work performed to validate the energy estimation technology. This involved running a number of carefully controlled test scenarios and using the tool to estimate their energy usage, while also deriving the same estimation through a separate, independent technique, using these secondary estimations to validate the outputs of the model and the tool.

\emph {Chapter 9} summarises the research work, draws conclusions from it to answer the research questions and discusses possible future research work in each area.



 %Ch1

\chapter{Literature Review} \label{chapter:lit-review}

\section{Architectural Description Languages} \label{sec:adl-lit-review}

Software architecture has been an active research field since the mid 1990s and one of its recurring research topics has been how to create, communicate and maintain effective architectural descriptions.  A range of techniques have been proposed over the years, but a recurrent theme is the idea of a specialised architectural description language (or "ADL").

The first ADLs appeared in the early 1990s and 10 significant languages from the first 10 years of research were the subject of a seminal literature review by Medvidovic and Taylor in January 2000 \cite{medvidovic2000-adlcomparison}.  Perhaps inspired by this work, there has been an explosion in the number of ADLs created since that time, but based on our industrial experience and reading of the research literature, there has been little indication of a corresponding increase in their use in industry.  

We are interested in how to assist architects to consider the energy properties of their systems as a first class architectural concern and this led us to ask whether we could use an ADL as the basis of any solution that we designed.   This led to our first research question namely, \emph{RQ1 - What ADLs exist and can they be used to reason about the energy properties of a system?}.  Our goal was to understand the possible applicability of existing architectural description languages to our problem and assess the degree to which the languages that have been created would be useful in industrial practice.

As part of answering this question, we undertook to identify and review the relevant research literature that has been created over twenty-five years of research in the area.  Our aim was to characterise the ADLs that have been developed and consider their possible applicability to addressing the energy properties of industrial software applications.

\subsection{Supplemental Research Questions}

As soon as we started to perform initial investigation into architectural description languages, we realised that it has become a complex and multi-faceted field.  Hence, to approach the review in a structured way, we posed a number of research questions specific to the survey, in order to understand the characterisics of the ADLs that have been designed, and their possible applicability to our problem:

\begin{description}
\item[ADL.RQ1] \emph{Which architectural viewpoints does each ADL support?}  It has been long understood that an architecture contains many structures, not just one.  This challenge is addressed by structuring an architectural description into views defined by viewpoints \cite{iso-42010}. Surveying the set of viewpoints supported by an ADL allows us to understand which architectural structures it can represent.

\item[ADL.RQ2] \emph{Does the ADL provide structuring mechanisms for large architectural descriptions?}  Many academic tools and methods are only tested using small examples whereas industrial systems are often orders of magnitude larger.  Our focus on the industrial application of ADLs meant that we wanted to understand which ADLs included features for structuring large architectural descriptions.

\item[ADL.RQ3] \emph{Does the ADL support the analysis of an architecture?}  Another possible motivation for using an ADL is the ability to perform automated analysis of a machine-readable architectural description, and this could allow the ADL to provide the basis for automated energy estimation and analysis. Hence we were interested to understand which ADLs allow this and what sort of analysis could be performed.

\item[ADL.RQ4] \emph{Can system qualities or quality requirements be captured in the ADL?}  A critical aspect of industrial software architecture work is ensuring that systems exhibit their key quality properties, so we wanted to establish what support each ADL provided to support this process.

\item[ADL.RQ5] \emph{Were prototype or production quality tools developed with the ADL?}  It is unlikely that an ADL will be seriously applied in industry unless it has robust and user-friendly tools available to support it, so we wanted to verify the level of tool support provided with each ADL.

\item[ADL.RQ6] \emph{Has the ADL been applied to non-trivial problems outside the group of people who created it? (e.g. significant research projects from outside the originating group, industrial case studies or industry standards.)}  A software architecture practitioner is likely to want some evidence of the effectiveness of an ADL before adopting it on a significant project.  Therefore, we wanted to know whether researchers had acknowledged this barrier to adoption and had addressed it through realistic case studies or use on real projects outside the originating research group.

\end{description}

It is worth noting that we do not ask if the language supports first class components because this is a prerequisite to the language being included in the study.  (Our view is that languages that do not support first class components are not architectural description languages.) 

\subsection{Research Methodology}

We identified the research literature to include in the study using an electronic literature search, augmented by manual scanning of reference lists in the papers found and our own background knowledge of the field, that led us to identify additional relevant candidate literature (that for example may not have been tagged with the keywords we expected).

We began by searching a range of electronic sources for papers that included the keywords "ADL" or "architecture description language" in their title or keywords.  The ten sources we used were the ACM Digital Library (advanced search), Google Scholar, IEEEXplore and Microsoft Academic Search.

Predictably these queries returned many references, however it was clear from our existing knowledge of the field that these keyword-based searches were not returning all ADL related literature. 

To find further relevant literature we then performed an exhaustive search of Google Scholar, using the relevant keywords, which returned over 10,000 references which were manually scanned for relevant primary studies that we might have missed.  This list contained many false positives, but these were discarded via manual inspection. 

Having searched traditional literature review sources, we also performed manual searches of specific publication venues where ADL researchers were known to publish their findings, specifically the specialist conferences WICSA, QoSA, ECSA and ICSE.

Finally, we performed forward and backward reference checking on the primary studies that we had found. Search engines were used to find citations of the primary studies identified that could be of relevance to the review (forward reference checking). The reference lists of the primary studies were then checked for any potential relevant studies missed (backward reference checking). At this point we were left with 135 potential primary studies for the survey.

Throughout these search activities, we limited the dates of the studies that we included, to limiting our scope to literature published between January 1991 and May 2016. The start date was selected to be early enough to include all those ADLs in the original work \cite{medvidovic2000-adlcomparison} that inspired us to undertake this later comprehensive survey and as noted in \cite{malavolta2013-industryadlneeds} the concept of an ADL was not well defined before this point.  Our literature search was concluded in May 2016, which is the reason for the end date (although in fact we did not discover any additional relevant literature published between January and May 2016).

To focus our efforts on the most relevant ADLs, our initial set of primary studies was filtered further to a more manageable set using the following exclusion criteria:

\begin{description}
\item[EC1] The ADL is a minor enhancement or minor extension to an existing ADL, or the ADL is a different version of an included ADL.
\item[EC2] The ADL focuses on a single area of architectural analysis (e.g. Concurrency) rather than being a general-purpose description language.
\item[EC3] There is not enough detail in the references discovered to address the study research questions.
\item[EC4] The ADL not suitable for modelling a software intensive system at an architectural level of concern (for example a hardware design language or source code module description language).
\item[EC5] The primary study is not available in English or is a short paper (less than 3000 words), abstract, keynote, opinion, tutorial summary, panel discussion, technical report, presentation slides, compilation of work or a book chapter. Book chapters were only included if they were conference or workshop proceedings (e.g., as part of the LNCS or LNBIP series) and are available through the data sources included in our review. 
\end{description}

The result of this further selection exercise was a list of 51 ADLs to include in the survey and 84 ADLs that did not meet our inclusion criteria.  A full list of the ADLs that met our inclusion criteria are characterised in the tables in \aref{appendix:adl-list}.

\subsection{Analysis of the Results}

The first aspect of the ADLs we were interested in was the \emph{basic information} about each and specifically institution(s) who developed them, the dates when the language was first published, the application domain that they address and the breadth of the application that they have been applied to.

When considering the breadth of application of the languages, we identified five possible degrees of application of an ADL that were of interest to us, namely:
\begin{itemize}
	\item "Examples", meaning that the language has only been used to create characteristic examples of its use;
	\item "Experiments", where it has been used to model realistic problems, but only for the purpose of investigating the language;
	\item "Case Studies", meaning that it has been applied to realistic problems from outside the originating research group but by the language creators;
	\item "Research Projects", where the language has been used on other research projects by researchers other than its creators; and
	\item "Industrial Projects", meaning that the language has been used by industrial software engineering teams on real projects (rather than industrial researchers, who would be classified as research project use).
\end{itemize}

We were obviously particularly interested in how many ADLs had been applied beyond its creating research group on other research projects, or ideally on industrial projects.

The complete data set for the ADL's basic characteristics extracted from the literature can be found in \tref{table:adl-basics}.

Two characteristics of the ADLs we wanted to understand were the application domains that they targeted and the degree to which they had been applied.

\begin{figure}
\centering
\includegraphics[width=0.6\textwidth]{Figures/litreview-adl-domains}
\caption{ADL Target Application Domain}
\label{figure:litreview-adl-domains}
\end{figure}

The analysis of the intended application domain of the languages can be found in \fref{figure:litreview-adl-domains}.  Interestingly very few of the languages were created for a specific business domain (e.g. financial analysis or industrial control systems) as over half the ADLs in the study do not explicitly target any business or technical application domain but are for general use.  There are a smaller number of ADLs specialised for embedded systems, distributed systems, highly concurrent systems and SOA, along with a number of individual ADLs for niche domains such as cyber-physical systems.

\begin{figure}
\centering
\includegraphics[width=0.6\textwidth]{Figures/litreview-adl-validation}
\caption{ADL Breadth of Application}
\label{figure:litreview-adl-validation}
\end{figure}

The analysis of the breadth of application of the languages can be found in \fref{figure:litreview-adl-validation}.  Unfortunately, as can be seen, less than 20\% of ADLs have been used beyond the case study level to perform significant research or industrial projects, suggesting a low degree of validation and practical experience with most of the languages.

The second area of interest to us were the \emph{architectural concepts} available in the different languages, to see if common industrial architectural concepts where in the languages or would need to be added.

The characteristics we analysed the ADLs for were the viewpoints that the ADL directly supports, the architectural concepts that they provide, whether they provide the ability to define behavioural semantics, whether they provide first class connectors and whether they provide first class architectural configuration constructs.  We chose to focus on these architectural concepts because of their wide use in the existing research literature and their general familiarity as concepts in industrial practice.

We were particularly interested in which viewpoints each ADL could support, as industrial architectural description nearly always needs a number of views to describe it, and the views supported provide a good insight into what the language can be used for.

None of the ADLs discuss a specific set of viewpoints that they define, so we analysed whether they provided effective support for the 6 viewpoints from \cite{rozanski2011-ssa2e} (which are Functional, Concurrency, Information, Development, Deployment and Operational).  We class a language has having first class connectors if the connector is defined separately to components and so is potentially reusable.  Similarly we consider architectural configuration to be a first class concept if it is described separately to the architectural elements and defines how they are combined, rather than being defined implicitly as part of the definition of the elements.

The complete data set for the architectural concepts available in each of the ADLs can be found in \tref{table:adl-concepts}.

The analysis of which viewpoints are supported by the different ADLs is presented in \fref{figure:litreview-adl-viewpoints}.

\begin{figure}
\centering
\includegraphics[width=0.75\textwidth]{Figures/litreview-adl-viewpoints}
\caption{Viewpoints Supported by ADLs}
\label{figure:litreview-adl-viewpoints}
\end{figure}

What is immediately evident from this analysis is that most ADLs only focus on the functional view of a system (i.e. its functional components and connectors and their organisation).  While clearly a key part of a system's architecture, most architects actually spend a lot of their effort working on other parts of an architecture (such as the deployment of the system).  So most of these ADLs are at best a partial solution to the problem of industrial architectural description.

\begin{figure}
\centering
\includegraphics[width=0.75\textwidth]{Figures/litreview-adl-candc}
\caption{Connector and Configuration Support}
\label{figure:litreview-adl-candc}
\end{figure}

The analysis of the number of ADLs that provide support for first class connectors and architectural configuration as a first class concept is shown in \fref{figure:litreview-adl-candc}.

This analysis reveals a very positive result, as the clear majority of ADLs in the study provide some form of first class configuration, while less, but still nearly two thirds, provide support for first class connectors, which are both possible motivating factors for architects to use ADLs as existing informal and semi-formal notations tend not to support these concepts directly.

The third area of interest to us were the \emph{language mechanisms} available in the different languages, to assess the languages to see whether they could address common challenges (such as structuring and evolution) for large industrial architectural descriptions.

The attributes of the language that we analysed the literature for were as follows:
\begin{description} 
	\item[Structuring] - what mechanisms are available for structuring a large architectural description?
	\item[Evolution] - what mechanisms are provided to allow an architect to evolve an architectural description?  (Such as the ability to describe architectural variations, the ability to version all or parts of the description or support for dynamic architectures).
	\item[Qualities] - how provided or required architectural properties can be captured in the architectural description (e.g. properties, attributes, related models etc.).
	\item[Syntax] - what concrete syntaxes are available to capture architectural descriptions in the language?
	\item[Analysis] - how analysis of an architectural description could be supported using the language and any supporting technologies associated with it.
	\item[Tools] - what maturity are the tools that have been created to support the language? This can be "none", "prototype" (meaning an initial tool implementation applied to small problems), "research" (meaning a fully implemented tool applied to realistic problems by researchers), and "commercial" meaning that one or more tools have been implemented and used in an industrial context by people other than the tool's creators.
\end{description}

\begin{figure}
\centering
\includegraphics[width=0.6\textwidth]{Figures/litreview-adl-analysis}
\caption{ADL Support for Architectural Analysis}
\label{figure:litreview-adl-analysis}
\end{figure}

A common justification for using ADLs is the ability to perform automated analysis on the architectural description once it is represented using an ADL.  Therefore, we were interested to understand how many ADLs provided some sort of direct support for analysis of architectural descriptions.  

When we performed this analysis, we found that it was quite difficult because the analysis capabilities depend on support tools as much as the language and different ADLs provide quite different types of analysis capabilities.  To allow us to answer the question, we have defined four types of analysis capability:
\begin{description}
	\item [Provided] - where the ADL has specific support in the language to capture the data necessary to allow an automated tool to use it for analysis and explicit consideration has been given to making this possible.
	\item[Via Extension] - where the ADL has been designed such that its extension mechanisms could be used directly to support automated analysis via a tool.
	\item[Via 3rd Party Tool] - which means that the ADL provides some generic facilities that could allow a 3rd party tool to perform automated analysis, but where no explicit support for it is provided.
	\item[None] - where the language does not appear to be amenable to automated analysis.
\end{description}

This analysis is presented in \fref{figure:litreview-adl-analysis}.  As can be seen, less than half of the languages appear to provide realistic possibilities for automated analysis (and of course of those that do, many do not have working tools available for them).  We conclude therefore that automated analysis is only of interest in a subset of research groups working on ADLs.  A concern that we have with this situation is that an important motivator for adopting ADLs does not appear to be addressed in many of the ADLs that have been created.

\begin{figure}
\centering
\includegraphics[width=0.6\textwidth]{Figures/litreview-adl-qualities}
\caption{ADL Support for Capturing System Qualities}
\label{figure:litreview-adl-qualities}
\end{figure}

A key goal of software architecture is to ensure that a system achieves the set of quality properties required for it to be successful.  This lead us to expect that ADLs would provide strong support for quality properties and we were interested in the types of mechanism used to represent them.  Having read the literature, we discovered that there were three broad levels of support for capturing quality properties in an architectural description:
\begin{description}
	\item[Properties] - where a generalised mechanism of (possibly typed) name/value pairs was available in the language and could be used to capture non-functional requirements and qualities but is not specifically provided for that purpose.
	\item[Attributes] - where specific pre-defined attributes relating to specific qualities (such as "transactions per second" for performance or "max connections" for scalability) can be captured within the language framework.
	\item[Language Support] - the case where languages provide a specific mechanism within the language for capturing quality requirements and capabilities (such as capturing security mechanisms and goals as first class language elements or providing a general purpose QoS or quality requirements sublanguage).
\end{description}

\fref{figure:litreview-adl-qualities} presents our analysis of this aspect of the capability of the ADLs.  It shows clearly that half of the languages provide no support for capturing system qualities, but that about a third (35\%) do have a generic properties mechanism which could be used to capture quality related information.  A much smaller number provide the ability to capture specific attributes (7\%) or have quality property features in their language (8\%).

\begin{figure}
\centering
\includegraphics[width=0.75\textwidth]{Figures/litreview-adl-structuring}
\caption{Support for Structuring Architectural Descriptions}
\label{figure:litreview-adl-structuring}
\end{figure}

Many industrial systems are large, much larger than any case study or prototype experiment in the research domain.  A typical industrial system today can contain 500,000 to 1mm lines of code and dozens to hundreds of architectural elements.  Such systems cannot be described using languages that do not have effective structuring mechanisms to allow a system description to be broken down into smaller discrete parts.  This led us to investigate the mechanisms that each of the ADLs in the study provided for structuring the architectural description.  This analysis is shown in \fref{figure:litreview-adl-structuring}.

While a few of the ADLs (about 12\%) don't provide a structuring mechanism, most do, with nearly all of them offering \emph{composition} and a few offering \emph{packages} or \emph{subsystems} (in most cases in addition to composition - hence the total of values in the chart is larger than the number of ADLs in the study).
This is an interesting result, suggesting that most ADLs can be structured for large architectural descriptions, but that most of them utilise composition as the mechanism to achieve this, rather than providing a separate structuring mechanism like packages or subsystems.  This may imply restrictions in the flexibility of the structuring facilities available in those languages where only composition is available.

\begin{figure}
\centering
\includegraphics[width=0.6\textwidth]{Figures/litreview-adl-toolsupport}
\caption{Tool Support Available for ADLs}
\label{figure:litreview-adl-toolsupport}
\end{figure}

ADLs are often developed in conjunction with supporting tools to help architects to use them, which makes them more attractive for use on significant projects. \fref{figure:litreview-adl-toolsupport} presents our analysis of this feature of the ADLs in the study.

As can be seen, a very small percentage of the ADLs have commercially proven tools available to them, while about a third of them have tooling being used on research projects.  Nearly two thirds of the ADLs in our study provided no effective tool support.

\subsection{Conclusions}

\begin{description}
\item[ADL.RQ1] \emph{Which architectural viewpoints does each ADL support?}
All the ADLs in the study can represent functional views of the system and most of them only provide support for this view, but a small number of them allow deployment, concurrency or development views to be created too. Hence we conclude that the focus of most ADL research groups is how to represent the system's functional structure. This isn't surprising given how central a functional view is for most systems, but given the general acknowledgement of the importance of other viewpoints \cite{bachmann2011-documenting, brown2018-sad, kruchten1995-4plus1, rozanski2011-ssa2e} it does suggest that most of the existing ADLs will not be a complete solution to the problem of representing a software architecture.

\item[ADL.RQ2] \emph{Does the ADL provide structuring mechanisms for large ADs?}
Most ADLs in this study (45) provide the ability to structure a large architectural description by allowing composition of architectural elements (of course composition can also be used for other purposes, such as information hiding).   A smaller number provide specific mechanisms for structuring such as packages (11) and subsystems (1).  A few ADLs, surprisingly, do not appear to provide a structuring mechanism (6).  Some of the languages provide more than one mechanism that can be used to structure an AD (e.g. packages and composition) and this is why the numbers above sum to more than the number of ADLs in the study.\\
It is encouraging that most of the ADLs we surveyed provide at least basic facilities for structuring a large architectural description. This suggests that serious consideration has been given to the use of the ADL for realistic problems.  Those languages that don't allow structuring are presumably in an early stage of development or are not intended for industrial use.

\item[ADL.RQ3] \emph{Does the ADL support the analysis of an architecture?}
We found that about half of the ADLs (24 or 45\%) do not appear to allow a realistic option for automated analysis of architectural descriptions, which was something of a surprise to us.  Some of the languages do provide this though, with about 32\% of them providing direct support in the language, while 15\% allow this by providing mechanisms for 3rd party tools to embed information in the architectural description and 8\% allow for analysis by providing an extension mechanism that could allow analysis information to be added to an architectural description. \\
A clear motivation for capturing and maintaining an architectural description is the ability to gain useful and reliable automated analysis that can provide insight into the design that is otherwise difficult to obtain.  The fact that many ADLs being developed do not appear to provide analysis capabilities suggests that the problem of how to motivate others to use the language is often not part of the research process.

\item[ADL.RQ4] \emph{Can system qualities or quality requirements be captured in the ADL?}
Quality properties are central to the role and activities of the software architect and so we hoped for strong support for capturing qualities and quality requirements in the ADLs.  In fact, we found that more than half of the ADLs studied (28) do not appear to offer a facility to capture qualities and of those that do, most of them just provide a generic "properties" mechanism which can be used for a range of purposes including capturing qualities.  Only about 7\% of the languages provide quality property specific attributes in the language or include the ability to describe qualities as first class elements of the language.  This is a surprising situation, if the ADLs are expected to be used in an industrial setting.  Years of practical experience have taught us that achieving quality properties is a key objective of a software architect \cite{brown2018-sad, rozanski2011-ssa2e}, so we would have expected that supporting quality properties would have been an important requirement for an ADL.

\item[ADL.RQ5] \emph{Were prototype or production quality tools developed with the ADL?}
Given the importance of tool support in achieving adoption of new software technologies, we were surprised to find that 45\% of the ADLs in this survey do not appear to offer tool support that is ready for widespread transfer to industry and use.  29\% of the ADLs provide a tool that has been used for a research problem, and only 8\% of the ADLs have an associated tool that has been tested in an industrial context.\\
An important factor in applying ADLs on industrial projects is good tool support, preferably through extending tools that industry uses already.  In fact, we would go so far as to suggest that industrial adoption of any ADL without practical tool support is unlikely.

\item[ADL.RQ6] \emph{Has the ADL been applied to non-trivial problems outside the group of people who created it?}
Given the effort required to develop ADLs, we assume that most of them are intended for eventual technology transfer to industry.  Assuming so, the current degree of transfer out of the research groups is disappointing.  We found that 58\% of the ADLs have only been used by their creators, to create simple examples or experiments.  Another 26\% of the languages have only been used for case studies, again by their creators.  12\% appear to have been used for research projects, outside the creating group, while a mere 6\% of the languages have been applied in an industrial context. \\
It is our opinion that a technology can only be considered have had an impact when it is used by people other than its creators, and when considering the products of research groups, this means significant usage outside the originating research group and ideally in an industrial context.  Nearly all the ADLs we have surveyed fail this test, with less than 20\% of them having been used outside their originating group (based on the publications we could find).  It is possible that some of these ADLs have been used industrially but the case studies not published, however we feel that this is unlikely given the positive impact that publishing such case studies would have.  We believe that this finding in itself is cause for reflection within the ADL research community (as was the previous similar finding from a workshop some years ago \cite{woodshilliard2005-adlsinpractice}).
\end{description}



\section{Prioritisation of Architectural Effort}

Introduction and list research question

\subsection{Research Methodology}

What did I do?

\subsection{Analysis of the Results}

Data analysis or list of paper summaries and common aspects

\subsection{Conclusions}

Answer the research question


\section{Architectural Guidance for Energy Efficiency}

Introduction and list research question

\subsection{Research Methodology}

What did I do?

\subsection{Analysis of the Results}

Data analysis or list of paper summaries and common aspects

\subsection{Conclusions}

Answer the research question


\section{Application Energy Consumption Analysis} \label{sec:litreviewenergy}

Introduction and list research question

\subsection{Research Methodology}

What did I do?

\subsection{Analysis of the Results}

Data analysis or list of paper summaries and common aspects

\subsection{Conclusions}

Answer the research question
 %Ch2

\chapter{Modelling Large Scale Information Systems using ADLs} \label{chapter:usingadls}

\section{Introduction and Goal}

  As we reported in \sref{sec:adl-lit-review}, there has been a great deal of academic and some industrial research into the definition of Architecture Description Languages (ADLs) to assist with the difficult task of clearly defining the architecture of software-intensive systems and there is still a significant amount of such research underway today \cite{diruscio2010-byadl, cuenot2010-east}.  However, there is limited evidence of significant industrial use of the ADLs that have been produced, which we believe is for a number of reasons \cite{bashroush2006-flexibleadls, woodshilliard2005-adlsinpractice} including the narrow focus of most ADLs and the mismatch between their strengths and the needs of practitioners.  This is particularly marked in the information systems domain, where it is difficult to find any large-scale use of ADLs, whereas there has been some documented use of ADLs in embedded and real-time systems \cite{oquendo2004-piadl, vanommering2000-koala, allen2002-rtsystems}.

  In order to investigate the second part of research question RQ1 ("What ADLs exist and can they be used to reason about the energy properties of a system?") we wanted to apply one or more of the ADLs from the research literature to the description of a significant system.  In this chapter, we describe the project we undertook to create a large industrial architectural description, which led to the conclusion that the existing ADLs would not be effective.  This led us to define a simpler, more specific notation which was successfully used to describe the system.

  This chapter is structured into an explanation of the context of the work, an introduction to the project that we undertook as the case study, an explanation of how we decided on an architectural description language to use, a description of the architectural description language we created, the case study of the application of the language, the experience we gained, the lessons we learned and the validation of the work and its use to answer the research question.

  \section{Context of the Work}

  The case study was undertaken in a financial services firm that has developed a large custom information system to run its business.  The software has been developed over a period of about 15 years and has grown from quite modest beginnings to the large system it is today, comprising millions of lines of code, storing several terabytes of information.  The system includes software modules that have been developed from scratch within the organization along with modules that have been acquired as a result of organizational acquisitions and that have been modified to integrate with the rest of the system.

  Today, the system comprises about 20 major subsystems and over 10 million lines of Java, C++, C\# and Perl, sharing a large multi-terabyte relational database.  Although some members of staff who worked on the system in its early days are still with the firm (and actively involved with the system) it has grown to a size that means no individual understands it all, even at a reasonably high level of abstraction.

  At the start of the project, there was no overall unified system description, although some teams responsible for subsystems did have their own documentation. This meant that the operation and interconnectedness of the system were often difficult to judge and this was starting to hinder change and evolution.

  The organization wanted to perform some wide-ranging evolution and modernization of the system's implementation and realized that a useful first step, to enable better intellectual control over the system, would be to capture a unified description of the system's architecture.  This led to the project described in this paper being undertaken.

\section{Overview of the Project}
\label{sec:overview}

  The lack of a unified system description and the need to modernise and restructure parts of the system led to a desire to create some descriptive documentation for the system.  At the outset, it was not entirely clear what sort of documentation was needed but discussion and exploration led to the conclusion that a current state architecture description was required.  The discussions led us to conclude that the documentation needed to provide a description of the system's architecturally significant elements, responsibilities and interactions, rather than more detailed documentation of the design of individual modules.

  Having gained a remit to proceed, we defined an approach and then worked with the software development teams to create the architecture description.

  In order to have some clear goals and overcome some ambiguity in the goals of the work, some assumptions had to be made and these were:

 \begin{enumerate}

\item The goal of the work was to create a comprehensive description of the architecture of the system as it exists to:
\begin{enumerate}
\item allow the architecture to be understood and analysed to allow estimation of key qualities such as its resilience, modifiability or energy properties;
\item allow impact analysis to allow architectural change to be planned; and 
\item provide a reference to communicate the architecture of the system.
\end{enumerate}

\item The audience for the completed documentation was architects, designers and development teams, so precision and completeness were important attributes.

\end{enumerate}

Another decision which had to be made was whether to try to provide the option of automated processing of the architecture description.  This would allow automated checking and analysis for applications such as power usage estimation or consistency checking.  To achieve this, the architectural description would need to be captured in a parsable form with well-defined semantics.  However, this requirement needed to be balanced against the resources needed to complete the work.  It was decided to capture the information in a form that would be amenable to parsing later but not to slow down the project by imposing an onerous syntax for the information.

When the software development teams were approached to discuss their involvement with the project, it quickly became clear that while there was general enthusiasm for the idea, there was very little appetite for actually performing the work required.  Therefore it was obvious that tolerance for learning new concepts or reworking outputs would be quite low.  Hence, it was going to be necessary to identify a simple, low-ceremony approach that was highly prescriptive in order to minimize the possibility of teams producing inconsistent artefacts that would need to be reworked.

This initial interaction with the development teams, along with our assumptions about the goals of the project and the audience for the artefacts (see Section 4), meant that there were a number of implicit emergent requirements and constraints that we needed to take into account.  These were as follows:

\begin{itemize}

  \item Simplicity - the approach needed to be simple to understand and apply, first because senior managers needed to understand it quickly to agree to its use; and second, because the software development teams who needed to produce the design documents were not prepared to expend a lot of effort on learning a new language.

  \item Low Adoption Effort - given the low tolerance for significant adoption effort, people needed to be able to pick up the basics very quickly and incrementally learn what they needed.  This extended to tooling where there was no enthusiasm for implementing, supporting or learning specialised modelling tools for this project.

  \item Conceptual Familiarity - the requirement for low adoption effort also meant that the notation and approach needed to support existing concepts that people were already familiar with (so the notation needed to contain the type of architectural elements found in the system, rather than generic elements that needed to be specialised or interpreted).

  \item Use Existing Tools - as mentioned above, requiring a new modelling tool to be installed and used for this effort would have caused the project to fail, so we had to use the tools already available in the organisation (which meant general drawing tools and wikis, although some licenses for a tailorable UML tool were available if needed).

\end{itemize}

Having understood and defined the goals of the work, and understood the priorities and constraints of the organisation that was going to perform a lot of the work, we started to consider our choice of language to use for the architectural description.

\section{Selecting an Architectural Description Language}

As became clear during the ADLs literature review work, presented in section \ref{sec:adl-lit-review}, a large number of ADLs have been developed in an academic context, which we considered for use in this work.

We started our consideration of ADLs by looking for a case study that had attempted to apply ADLs in a large industrial context, but we could not find any published case studies that report on an architectural description language being used to describe a large information system. There have been a number of published reports of ADLs being used to describe embedded or real-time systems (such as \cite{feiler2000-realtime, lonn2004-east, cuenot2010-east, vanommering2000-koala, sae2009-aadl}) but these systems differ significantly from a large information system, with different concerns and requirements of an ADL.

As we saw in \sref{sec:adl-lit-review} Many ADLs have been proposed by researchers, including xADL \cite{khare2001-xadl}, ADLARS \cite{bashroush2005-adlars}, ALI \cite{bashroush2008-ali}, ArchiMate \cite{lankhorst2009-archimate} and ByADL \cite{diruscio2010-byadl}, to just a few of the more recent ones. Most of these languages exhibit novel approaches to architecture description, from support for interchange and interoperability to advanced architectural analysis capabilities.  However, we found all of them lacking when we experimented with them to evaluate them for our project.

In general, academic ADLs focus on analytical evaluation and rigour but in this project, in common with many other industrial situations, the focus had to be on accessibility, practicality, and the ability to obtain a reasonably complete view of the structure and behaviour of the system with a modest amount of effort. In most situations, and certainly in this project, it is difficult to persuade practitioners to use an unfamiliar formal notation for architectural description and we were sure that if we did not focus on these pragmatic factors relating to the use of the ADL and the immediate usefulness of the result, we would have failed to get the cooperation of the teams and so the exercise would not have been successful.

A number of the research ADLs (such as ACME, xADL and ByADL) do, in principle, support the kind of description we wanted to create but when we experimented with them, we found that their very generic, general-purpose nature meant that they would have needed a lot of investment in tailoring and extension to be effective in this situation, given the need for conceptual familiarity.  We would also have incurred significant investment to create tutorial materials and to evaluate, integrate or build tool support (such as providing drawing support in standard tools like Visio rather than academic prototypes). This meant that the benefits we would have gained from using these languages were not large enough to justify the adoption overhead and risk to the project.

The third observation that we made was that the majority of ADL applications reported in the literature as experience reports are confined to laboratory-based case studies rather than exploring a practical application beyond an unrealistically small example.  This further reduced our confidence that any of these languages were appropriate for use in this project, where we needed to show success quickly, and build a description of a 10 million line of code system with 20 subsystems, that would provide real benefit to the organisation.

It can be argued that many of these ADLs could be used in an industrial context but simply have not been applied to significant industrial projects to date.  This is true, it is possible that they could be applied successfully but as explained above, we identified a number of serious concerns about their use for a significant industrial project.  When we explored the languages, we found that we could not identify compelling features that would bring enough benefit to justify the risk of what was likely to be a difficult adoption process.

While not strictly an architectural description language, we also considered the use of ArchiMate \cite{lankhorst2009-archimate} given the fairly wide spectrum of features that it provides for enterprise architectural description. However, upon closer investigation, we found that the primitives in the ArchiMate language were not a particularly good fit given our need to describe system (i.e. software) architecture rather than enterprise architecture.

It is also important to acknowledge that outside the area of information systems, there have been a number of industrial applications of ADLs for embedded and real-time systems, from consumer electronics (e.g. Koala \cite{vanommering2000-koala}, $\pi$-ADL \cite{oquendo2004-piadl}) to aeronautics and automotive systems (e.g. AADL \cite{sae2009-aadl} and EAST-ADL \cite{cuenot2010-east}). We investigated these situations through the published research literature and noted that the use of ADLs in these application domains has enabled automated system analysis, and automated code generation (e.g. MetaEdit+ \cite{smolander1991-metaedit}). This could well be one of the reasons that these applications of ADL technology were successful.  However, given that analysis and code generation were not primary goals of this project and that our priorities were straightforward system description with easy and low-cost adoption we did not feel that we could reproduce the success of these case studies given the published ADLs we had available to us.

Considering the combination of these factors, our conclusion was that adopting one of the existing research ADLs was unlikely to be successful and could well endanger the success of the work.  Therefore we reluctantly judged that it was going to be simpler and safer to develop our own special-purpose notation and this was much more likely to result in a successful and useful architectural description.  
  
\section{The Approach}
\label{sec:approach}

  Having discounted the idea of using a formal ADL, we seriously considered using a tailored version of UML, with a suitable UML profile.  The architects leading the effort already knew UML well, had used this approach before, and knew that it would have provided a basis on which to build our own specific notation.  However, even a tailored version of UML needs some background knowledge of the underlying language in order to use it effectively; this was lacking in nearly all of the software development teams.  The use of generic UML without a profile wasn't seriously considered because we knew it would meet with a lot of resistance and we would end up with significant divergence in the models that the teams would create.

  We also considered just letting teams use their own informal notations.  In principle, this would have removed one of the major points of resistance to the project and would have saved the effort of developing a notation.  However, this had already been attempted in the organisation and the results were so varied that the exercise did not yield a useful system-wide description, so we also discounted this option.

  Eventually, given all of the factors involved in this project, we reluctantly concluded that the project was most likely to be successful if we developed a simple, well-defined, very specific, notation that just contained the element types that would be found in this particular system and then provided the teams with support for it in desktop drawing tools and a wiki.

  The initial discussions with the development teams revealed a varied understanding of modelling and abstraction, which led to a further realisation that the approach used was going to have to be comprehensible to modelling novices within minutes, rather than needing much effort to learn.  We concluded that in order to avoid confusion, the models were going to have to capture specific component and connector types that described the physical structure of the software (e.g. runtime processes and inter-process communication channels) rather than more abstract and generalised concepts such as software components and responsibilities.  If the teams had been asked to describe their software in terms of more abstract concepts, we believe that the project would have collapsed under the weight of debatable, unverifiable abstractions and it would not have been possible to validate the models against the implementation.

  Given the resources available, it was decided that using a wiki was going to be the most effective way to capture the data underpinning a graphical representation (the system element descriptions, connection definitions, inter-element dependencies and so on).  A wiki allowed this information to be captured in an accessible way, without special tools but allowed very restricted formats to be prescribed that standardised its presentation and would be amenable to basic machine parsing later if needed.

  The wiki approach of creating simple hyperlinked pages also allowed the architecture description to be decomposed into a set of manageable pieces, each with clear ownership but allowed these different pieces to be linked together to provide cross-referencing and navigation through the documentation.  Hyperlinking also provides a simple sort of type checking in the documentation, as names can be linked to their definitions elsewhere in the wiki and if the name is wrong, a broken link results, which is immediately obvious.

  We found that a wiki provides a lot of the flexibility of a word processor but can also provide basic mechanisms to allow structuring, templating and cross-referencing via simple conventions and most software developers find them very easy to use.

  What a wiki does not usually provide is any support for graphical notations but the diagrams are the part of the architecture description that people spend the most time creating and reading, so they are important to get right.  As explained already, having considered the options available, it was decided to create a new highly constrained graphical notation that would encourage the creation of graphical models at the right level of abstraction.  In order to create a consistent notation that was easy to use, the guidance in \cite{moody2009-notations} was followed in order to design the notation systematically.

  The whole project, and in particular the definition of the graphical notation, was helped by the fact that while the system had grown rather organically, it had evolved according to a specific set of architectural constraints that could loosely be identified as an architectural style.  This had limited the degree of implementation diversity and so reduced the number of concepts that it was necessary to represent in the description language.

  Within the system, nearly all subsystems were comprised of the following types of elements:

  \begin{itemize}

\item Message-driven servers that performed functional processing in response to events or requests arriving from a system-wide message bus;

\item "Thick" clients that provided user interfaces and business logic (and typically communicated with the message-driven servers via the system message bus);

\item Web interface servers that provided web user interfaces (typically written as Java servlets or Perl modules);

\item Batch programs that performed some sort of periodic processing (such as end-of-day reporting); and 

\item Data loaders, which were a particular sort of batch program, which imported data into the system or moved data between subsystems.

\end{itemize}

  The servers, batch programs and data loaders (and occasionally clients) would in turn normally have dependencies on a fairly large number of database objects (that is tables, views and stored procedures).

  This very specific set of architectural element types was used throughout the implementation of the system, which meant that a simple ADL could be defined in terms of those specific element types.

  A corresponding set of wiki page templates was created to support the capture of the supporting textual description for the graphical models in order to make the format required for the descriptions clear. This also made the management of the process easier as there were relatively few concepts that needed to be explained and it made progress easy to track in terms of completed wiki pages and sections.

\section{The Style and Its Architectural Description Language}

\subsection{The Architectural Style}

  An analysis of the system's implementation revealed that it generally followed a set of discernable patterns created from a small number of types of architectural elements, which could loosely be described as an architectural style (taking the definition of architectural style from Shaw and Garlan \cite{shaw1996-softwarearch} to be "a vocabulary of components and connector types, and a set of constraints on how they can be combined").  

  To allow the element types of the system to be described, a few basic concepts were used to set the context and help people to understand the key abstractions:

\begin{itemize}
\item System - the entire information system that is being described, which is a conceptual structure, composed of a number of interconnected subsystems that collectively provide its behaviour and qualities.

\item Subsystem - a subset of the system that has a well-defined, cohesive, set of responsibilities, and in most cases a well-defined boundary and set of interfaces to its services.

\item Component - a tangible software artefact which is delivered to the production environment and which is "executed" in some way at runtime (whether directly or by being called). Nearly all components are binary releasable elements, tracked in the change management system. (Elsewhere in this paper we refer to "components" as "elements" in line with much of the software architecture literature)

\item Connector - the mechanism by which two or more components collaborate (usually by passing data between them).  Examples are a messaging, a file system file, a database table, or a web service endpoint and invocation.

\end{itemize}

  It is worth noting that even though our definitions of concepts like "component" and "connector" were quite specific, most people didn't really understand what we meant until we made the concepts very concrete with the specific types of component and connector that they were familiar with.

  As mentioned above, the basic types of system element used within the system were user interface programs, servers, data stores, external entities and a fairly specific set of connector types were used to link them.  While these generic types of element sound fairly standard, what was interesting was the limited number of variations of them that were used in most of the system.  These element types are summarised in Table \ref{table:archelemtypes}.

  
\begin{table}
\caption{Types of Architectural Elements}
\label{table:archelemtypes}
\footnotesize

\begin{tabular}{| l p{10cm} |}
\hline
User Interfaces \\
\hline
& \\
GUI   & A traditional GUI client written in Java Swing, C\# WebForms or C++ Motif. \\
WebUI & A user interface implemented as a set of web pages (typically as a set of CGI scripts or a Java webapp) \\
Command Line & A user interface implemented as a command line program, such as a script or a Unix command line utility \\
& \\
\hline
Servers & \\
\hline
& \\
Message-Driven Server & A server whose operation is driven by the receipt of messages from the system message bus \\
Server                &  A server whose operation is driven by a mechanism other than messages (such as RPCs, database polling or temporal schedules) \\
Batch Program         & A program that is run from a scheduler and performs its operation in a single execution, without waiting for other system elements to perform any operations or for human intervention. \\
Data Loader           & A program whose primary purpose is to extract data from a source and move it to a destination, typically transforming it in some way during the transmission. \\
& \\
\hline
Data Stores  \\
\hline
& \\
System database   &  The shared system database or a set of tables from it \\
File              & A file on the file system \\
& \\
\hline
External Entities  \\
\hline
& \\
Subsystem            & Another subsystem that communicates with this one in some way \\
External System      & An information system outside our system that a subsystem communicates with in some way \\
External Data Source & A Data Source outside our system that a subsystem receives data from (such as a source of security prices) \\
\hline
\end{tabular}
\end{table}

The fairly restricted set of inter-element connectors in use throughout the system is described in Table \ref{table:archconntypes}.


\begin{table}
\caption{Types of Architectural Connectors}
\label{table:archconntypes}
\footnotesize
\begin{tabular}{| l p{10cm} |}
\hline
RPC                & A synchronous inter-process procedure call (usually XML over HTTP) \\
Direct Invocation  & An in-process direct procedure invocation (calling a library) \\
Database Data Flow & Writing data to a database table or tables to allow it to be used by another element \\
File Data Flow     & Writing data to a filesystem file to allow it to be used by another element \\
System Messaging   &  Dispatch and receipt of messages over the system message bus via a named messaging destination \\
\hline
\end{tabular}
\end{table}

  In order to allow for the inevitable special cases that are found in a system of this scale, an "other" type was also allowed for both components and connectors, which could be annotated using a UML style stereotype to make its type clear.

  Most architectural styles limit the element and connector configurations that they allow.  In this style, there weren't really any such constraints defined formally, although there were combinations that were encouraged and discouraged (e.g. UI Clients should connect to Message-Driven Servers but not access the database).  However, most configurations of element and connector types could be found somewhere in the system. A number of the common patterns were captured as examples in the notation documentation.

  A couple of examples of the patterns identified are shown in Figure \ref{figure:adlnotation1}. 

\begin{figure}[h]
\centering
\includegraphics[width=10cm]{Figures/adls-figure1}
\caption{Examples of the ADL Notation Illustrating Preferred Configurations}
\label{figure:adlnotation1}
\end{figure}  

  The notation used to express the examples is explained more fully in the next section but briefly triangular shapes represent user interfaces, rectangles represent server resident elements (servers, batch programs), files and databases are represented by the fairly conventional "record stack" and "drum" shapes, while connectors are represented by arrows using a variety of line types (the line type in example (a) being messaging, the line type in example (b) being stored data access).

\subsection{The Architecture Description Language}

  Once the universe of required element and connector types was understood, we needed a notation that would allow instances of the style (i.e. the subsystems) to be clearly represented.  As explained earlier, we decided to define a custom notation because the initial discussions with the teams had made it clear that getting people to use a specific tool or invest much effort in learning the notation was going to be very difficult. This was a key reason for creating a very simple notation and "just drawing pictures" rather than trying to apply a general-purpose notation or create machine-readable models.

  Given people's general enthusiasm for diagrams over text, we chose to create a graphical notation rather than a more formal textual one. We could have created an equivalent textual notation to provide an alternative concrete syntax but we didn't need one for this project and as we were not trying to create a reusable ADL we had no reason (or the time) to create alternative notations.

  When defining the graphical detail of the notation, the advice in \cite{moody2009-notations} was particularly useful, in particular the exhortation to avoid construct overload, deficit, redundancy or excess, the suggestion to systematically consider the visual variables of each shape (shape, size, colour, orientation, brightness and texture) and the need for deliberate selection of shapes so that their appearance suggested their meaning, to help achieve semantic transparency.

  We created the graphical notation by selecting a base shape for each major type of element (server, user interface, data store, external entity) and designing a variation of the shape for each subtype of the element.  The diagrams were likely to be printed in black and white, so brightness and colour were used in a very limited way (just being used as an informal diagrammatic annotation, rather than having a predefined meaning).  Each element had to have a name, shown on its symbol and optionally a stereotype (discussed below).  Examples of the notation for some of the more important element types are shown in Figure \ref{figure:adlelementtypes}. 
  
\begin{figure}
\centering
\includegraphics[width=10cm]{Figures/adls-figure2}
\caption{ADL Element Types}
\label{figure:adlelementtypes}
\end{figure}  


  A triangle was used as the base shape for user interfaces and a rectangle for server resident components.  The triangle was chosen as it hinted at the head and shoulders shape of a user and the triangles were then modified slightly for each type of user interface (the thick client having sharp corners, the web user interface having rounded corners as it blurs the distinction between "client" and "server" and the command line utility having a graphical representation of a command line interface added to it).  Similarly, a rectangle is the base shape for server elements (based on long-accepted conventions) with a stereotype being used to indicate the type of server and a "lozenge" variant being used to indicate a data loader (hinting at pieces of data being transmitted through it).

  An arrow of some form was used to represent all of the connector types, with the arrowhead usually indicating the direction of data flow.  All connectors were defined to be one-way connections, with the exception of data access connectors, which could indicate read and write activity with arrowheads at both ends of the connector if appropriate.  The convention for RPC connectors was defined to be a one-way arrow from the caller to the target.  No attempt was made to represent the various complicated possibilities of dependency and initiation of interaction using the connector symbols.  Each connector had to indicate what was carried over the connection, with message flows being annotated with a message data type, file and database connectors being annotated with table or record names, and RPC and direct invocation connectors being annotated with the name of the service or procedure they were calling.  Examples of the notation for the main connector types are shown in Figure \ref{figure:adlconnectortypes}. 

\begin{figure}
\centering
\includegraphics[width=10cm]{Figures/adls-figure3}
\caption{ADL Connector Types}
\label{figure:adlconnectortypes}
\end{figure}  


  The RPC or direct procedure call is shown using a solid arrow, messaging is shown using a line with embedded dots, suggesting messages flowing over it, while data access is shown using a regular chain line, suggesting records being read or written over the connector.

  A general mechanism used on elements and connectors was the stereotype, adopted from UML, where the type of an architectural element is made clear by annotating it with a type name using the 
convention "{\guillemotleft}type{\guillemotright}" on the symbol concerned.  This allowed the casual reader to understand the types of element on the diagram without having to understand the notation and allowed new element types to be easily introduced.

  The semantics of the elements and connectors were generally based on the semantics of the corresponding element and connector implementations in the system: broadcast messaging in the system worked in a particular way, a relational database has well-understood behaviour, a web service call is widely understood and a message-driven server was a concept that most people understood with little further explanation.  Undoubtedly there were cases where elements on diagrams had surprising behaviour because they did not behave entirely as expected given their type but on the whole, the resulting documents were good enough to form a useful architecture description.

  In order to ensure that the process produced more than just pictures, we defined a set of required attributes for each type of element and connector.  Part of this task was defining enumerations of expected standard values for many of the attributes, again to standardise and simplify the process of recording the information (such as standard lists of data domains ["trading", "counterparties", "securities", ...], lists of programming languages in use [C++, Java, C\#, Perl] and so on).

  In order to simplify and standardise the subsystem descriptions, a set of wiki page templates and a comprehensive Microsoft Visio stencil were created, along with clear instructions, quick reference material and - most crucially - a fully worked example of the documentation for one subsystem.  This allowed a number of conventions, such as hyperlinking element names to allow navigation through the documents, to be illustrated and encouraged by example.  A hierarchy of empty wiki pages for the required subsystem descriptions was also created so that authors knew where to put their documents and so they could be unambiguously referenced.

  The result of this process was a relatively informal definition of a simple ADL with a graphical notation and set of well-defined conventions for storing the supporting text needed to explain and fully define the subsystem descriptions.  The ADL is tied very strongly to the particular architectural style of this system (its element and connector types) and we deliberately did not attempt to generalise the language, as this very tight link to the system to be described was one of its major strengths for our situation.  In this way, our ADL is rather like the ADLs defined to support specific implementation frameworks like DAOP-ADL \cite{pinto2003-daopadl} which was developed to describe DAOP applications \cite{pinto2001-daop} and CBabel \cite{rademaker2005-cbabel} which was developed to allow the definition of CR-RIO applications \cite{loques2004-crrio}.

\section{A Case Study of the Approach in Use}

  The system described in the case study is the Asset Management System (AMS) a financial asset management system used by a fund manager to support making and executing investment decisions for a large-scale investment portfolio.  The example is based on a real subsystem from the case study, modified slightly in order to retain anonymity.

  The primary aim of the system is to allow a fund management team to manage a portfolio of holdings in financial instruments (primarily equities in this case).  The system must allow them to view the content of their portfolios and to use analytical tools and market data (such as prices, volatilities, projected interest and foreign exchange rates and projected bond yields) to make investment decisions.  The system provides the ability for suggested changes to portfolios to be automatically calculated on demand or from a temporal schedule and also allows direct entry of orders to buy or sell securities to allow for investment strategies that are outside the scope of the system.  Once lists of orders to buy or sell securities are generated, the system allows them to be dispatched to another system for execution and it receives the results of the execution of those orders in return, to allow the current holdings to be updated.

\subsection{Architectural Description}

  The functional structure of the AMS is described using our system-specific ADL 
notation in Figure \ref{figure:amsdiagram}.  The elements of this architectural structure are described in Table \ref{table:amselements}. 

\begin{figure}
\centering
\includegraphics[width=10cm]{Figures/adls-figure4}
\caption{The Asset Management Systems}
\label{figure:amsdiagram}
\end{figure}  
  
\begin{table}
\caption{Elements of the Asset Management System}
\label{table:amselements}
\footnotesize
\begin{tabular}{| l l p{7cm} |}
\hline
Element Name & Type & Description \\
\hline

Portfolio GUI      & GUI              & The responsibilities of the Graphical User Interface (GUI) are to provide the asset managers using the system with the ability to view and analyse their portfolios, to request (and monitor progress of) long running system operations (such as order generation) and to check, enter, dispatch and monitor orders that go for execution to trading systems.  The GUI provides a human interface and requires an RPC interface to the UI Server to provide it with services and data. \\

UI Server          & Messaging Server & The responsibility of the UI Server is to provide the data access facilities that the UI requires (accessing data from the AMSdb internal database) and to dispatch requests for orders or for long running work (such as analysis processing) to be carried out by other parts of the system.  The UI Server provides an RPC interface to expose its provided services to the GUI and requires an SQL query interface to the system database and a messaging interface to allow it to request and monitor order dispatch and long running work. \\

AMSdb              & Database         & The system database's responsibility is to store the portfolio, analytical, market and (system) operational data that the system requires to operate.  It provides an SQL based DML interface to allow data to be inserted, manipulated or retrieved. \\

Job Processor      & Messaging Server & The responsibilities of the Job Processor are to execute long running processing items ("jobs") such as investment analytics and automated order list generation.  The processor can be configured to run particular jobs on temporal schedules and can also be requested to execute particular jobs on demand.  The processor provides a message based job control and status request interface and requires an SQL query based interface to the database. \\

Market Data Loader & Loader           & The responsibility of the Market Data Loader (MDL) is to retrieve various forms of market data from an internal Market Data Source system and load the data into the database, handling versioning and business date identification as part of the loading process.  The datasets required include securities prices, bond yields, interest rates, FX rates, volatilities, correlations and so on.  The loader requires a data retrieval interface to the MDL system, allowing data sets to be retrieved on demand. \\

Order Gateway      & Messaging Server & The responsibility of the Order Gateway is to accept incoming orders to buy and sell securities (including order parameters such as execution strategies and price limits), to forward these requests to a trading system for execution and then receive the execution reports ("fills") indicating order execution and broadcast these to other interested parts of the system.  The gateway provides a message based order request interface and a broadcast status interface and it requires a message based interface to allow order submission to a trading system.\\
\hline
\end{tabular}
\end{table}

\subsection{Example Scenario - Generate Order List}

  The key functional scenario for this system is to allow a fund manager to generate an order list to "rebalance" a fund based on an analysis that identifies the theoretically optimal holdings for the portfolio and execute that set of buy and sell orders, reflecting the results in the portfolio.  The interactions required to implement this scenario are illustrated in Figure \ref{figure:rebalanceinteractions}. 

\begin{figure}
\centering
\includegraphics[width=10cm]{Figures/adls-figure5}
\caption{Portfolio Rebalance Scenario Interactions}
\label{figure:rebalanceinteractions}
\end{figure}

The interactions between system elements necessary to implement this scenario are described in Table \ref{table:amsinteractions}. 
  
\begin{table}
\caption{Interactions for the Portfolio Rebalance Scenario}
\label{table:amsinteractions}
\footnotesize
\begin{tabular}{| p{0.5cm} p{2cm} p{2cm} l p{2cm} p{4.9cm} |}
\hline
Step & From & To & Type & Connector & Description \\
\hline

1 & GUI & UI Server & RPC & portfolio service & Fund manager selects a portfolio and instructs the system to create an order list for it.  The GUI invokes an RPC indicating that the indicated portfolio should be rebalanced. \\

2 & UI Server & Job Processor & Msg & Rebalance Request & The UI Server sends a request message to indicate that the portfolio should be "rebalanced".  This is routed to the Job Processor. \\

3 & Job Processor & AMSdb & DB & pmgmt and ordermgmt schemas & The Job Processor receives the message and in response initiates a portfolio analysis job to identify the theoretical optimal holdings in the portfolio and generate buy and sell orders to move the portfolio to that state.  Portfolio state read from "pmgmt" and order lists written to "ordermgmt" \\

4 & Job Processor & UI Server & Msg & OrderList Update & The Job Processor sends a status message indicating that new order lists exist, which is routed to the UI Server \\

5 & UI Server & AMSdb & DB & ordermgmt and pmgmt schemas & The UI Server accesses the database to get the new portfolio state and associated order list state \\

6 & GUI & UI Server & RPC & portfolio service & The GUI calls the UI Server for a status update and gets details of the new order list in return \\

7 & GUI & UI Server & RPC & portfolio service & The GUI makes an RPC call to the UI Server to indicate that the order list should be traded \\

8 & UI Server & Order Gateway & Msg & Execution Request & The UI Server creates a message to request the order list to be traded (including the list of orders) which is routed to the Order Gateway \\

9 & Order Gateway & Trading System & - & - & The Order Gateway sends the orders to an external trading system and receives status updates in return as the orders are executed \\

10 & Order Gateway & UI Server & Msg & Execution Report & As the Order Gateway gets execution updates, it creates execution report messages which are routed to the UI Server \\

11 & UI Server & AMSdb & DB & pmgmt and ordermgmt schemas & The UI Server updates the database with the status of the orders and the effect on the portfolio \\

12 & GUI & UI Server & RPC & portfolio service & The GUI makes RPC calls to the UI Server and gets the updated status of the orders and the changes to the portfolio in its response \\
\hline
\end{tabular}
\end{table}

  A full architectural description for a subsystem would also include a lot of operational and implementation oriented information such as links to operational instructions, links to source code control systems and automated build systems and links to test specifications and results.  We do not attempt to reproduce any of that here as the majority of such information was in the form of links to other internal systems and nearly all of the information is context-dependent and so not particularly meaningful outside the organisation operating the system.

\section{Experience Gained}

\subsection{Creating the Architecture Description}

  As mentioned earlier, two experienced architects led the project to create the architecture description, which included identifying the underlying architectural style, defining a clear approach, defining the ADL and leading the work to capture the architectural descriptions.  There were approximately 20 development teams who owned significant subsystems that needed to be included in the scope of the project.

  In order to organise the work, the development teams were ranked in order of the criticality of their subsystems in terms of how central they were to key organisational workflows and this acted as an ordered backlog of work for the architects.

  The general approach taken to the task was simple and involved approaching each team and asking for a single person to be nominated as the owner of their documentation.  A conference call was then held with this person and the group manager to explain the project and the approach.  The team was asked to commit time and effort to complete their documents and to commit to a timeline for completing the agreed deliverables (a team often had a number of subsystems that needed to be documented and for planning purposes, the creation of a subsystem description was decomposed into some standard subtasks).  In return, the architects leading the effort offered training, practical assistance (such as drawing diagrams) and to review the descriptions produced.

  The interactions with different teams varied greatly, with some teams producing their documentation largely unaided, needing only some review and minor correction, while others were simply incapable or unwilling to produce what was needed and the architects ended up writing most of the documentation for these teams. 

  The reasons for the problems encountered with development teams varied.  In some cases, it was simply a lack of interest, often from the development manager who perhaps didn't see the value of the deliverables.  In other cases, there seemed to be a genuine difficulty in understanding how to represent their subsystem.  In general, this seemed to stem from an inability to abstract away from the implementation, resulting in a confusing mix of concrete and totally abstract concepts, which they then struggled to relate to each other.  None of the subsystems was very difficult to represent, and in order to make progress, the architects often stepped in and simply created the models.

  Another interesting problem was tooling.  Everyone in the organisation had access to the wiki and knew how to use it, so document authors could fill in the tables and text without any difficulty.  However, not everyone had access to Microsoft Visio and even of those that did, some obviously didn't know how to use it.  Again, the solution to this was simply for the architects overseeing the process to create diagrams for some subsystems.  This was a useful lesson and provided further evidence that avoiding UML and more specialised modelling tools had been a good decision.  In this organisation, requiring the use of UML and modelling tools would have been a significant barrier to getting architectural descriptions created.

  Over time, a significant and useful body of subsystem descriptions emerged and this allowed the architects to create a summary level architecture description that showed how the subsystems related to each other.  Some use of scripting to process the wiki subsystem descriptions and drawing tool macros to generate parts of the summary level diagrams allowed some degree of automation, although it was still a fairly manual process.

   The process of capturing the architecture description took about six months, with the architects working on it approximately 60\% of their time and the development teams working on it as their project schedules allowed.

\subsection{The Results of the Project}

  The outputs of the project were as follows.

\begin{itemize}
\item A fairly consistent architecture description for most of the system that provided an accurate and largely complete view of its subsystems, their components and their dependencies.  Each subsystem was described using a standardised approach, which captured the same information for each one and presented it in a consistent manner through the use of the templates provided.  This made the information provided easy to navigate and check for completeness.

\item An informal definition of the architectural style used across most of the system and the typical patterns used when implementing it.

\item A degree of visibility and understanding of the structure, scale and interconnectedness of the system which hadn't been achieved before.  The consistent presentation of system design information in a single location allowed the overall system structure to be more easily understood compared to the previous inconsistent descriptions on scattered wikis and websites.  This appeared to allow a number of senior technical managers to achieve new insights into the system.

\item An insight into the degree of implementation uniformity between the different subsystems of the application.  While many subsystems were implemented in a very similar way, like any large system (particularly one which has had other applications integrated into it), parts of this application were implemented in ways that didn't follow the normal set of conventions.  While there was already a general awareness that these less standard subsystems existed, the models made it easier for senior technical staff to gain visibility of this and decide whether they wished to direct any changes to the application as a result.

\end{itemize}

  As mentioned earlier, the project did not have particularly clear goals for the architecture description once developed.  A number of people did find it insightful and there seemed to be a general consensus that it was a useful description to have.  However, organisational changes then meant that the architects involved moved on to other work, so the project effectively came to an end.  Since then another group within the firm has adopted the architectural description and continued its use and maintenance (primarily to support production operation of the system, a use which was not foreseen at the outset of the project).

\subsection{Evaluating the Usefulness of the ADL}

 Our Early practical experience led to some rapid refinement of the notation to remove ambiguities that had not been apparent to its creators and to introduce some missing concepts.  However, after three or four teams had used the approach over a period of about 6 weeks, the ADL itself remained stable for the rest of the project.

  As the project neared completion we started to validate what was being produced by talking to some of the important stakeholders, particularly the senior technical managers in the organisation.  To do this we met with them and demonstrated what was being produced and what the completed architectural description would contain, discussing possible uses of it (such as impact analysis, pre-implementation reviews, incident post-mortems and regulatory enquiries).  We were pleased to find that this stakeholder group reacted positively to what they were shown, with responses ranging from fairly neutral (where the possible usefulness was acknowledged but no specific use of it particularly interested them) to very positive (where they wanted to start using it immediately).  Given this informal but consistently positive sentiment, we felt that our notation and approach had been validated (an outcome which was anything but certain at the start of the project, when the use of a specific notation and a highly prescriptive form for the documentation had been viewed as very risky). 

  A factor that was constant throughout the project was that teams who had the ability to identify clear abstractions for their subsystems also appeared to find the ADL helpful and straightforward to use, as the ADL gave them a clearly defined way to represent their models and they didn't have any difficulty in representing their models using it.  These teams tended to create their models with little or no assistance once they'd asked a few clarifying questions about the purpose of the models and the semantics of the notation.

  In contrast, teams who struggled to identify good abstractions never really grasped how to use the ADL and needed constant assistance, to the point of needing to have parts of their architectural descriptions were completely rewritten for them.   What was interesting about this stark contrast in modelling ability was that we could find no obvious factor to explain it in terms of educational background, age, team size, technology preferences, type of subsystem, geographical location or any other relevant factor.  We speculate that it could be related to a person's thinking style (for example whether they tend towards abstract or concrete thinking \cite{ylvisaker2006-concreteabstract}) but we did not investigate this during the study.  We did observe that even in teams that produced good models, the ability and enthusiasm to do this varied and even for large subsystems we found that it tended to be one or two people in a team who did all of the modelling on behalf of the rest of the team.  We don't know whether there were many other people in those teams who would have done an equally good job but based on hallway conversations, we suspect not.  Our conclusion was that relatively few people in the general population of software engineers we worked with find modelling straightforward but we were not sure why this was the case.

  We viewed this experience as validation of the approach that had been used.  People who could create models and knew what they wanted to represent were able to use the ADL effectively with minimal training, so it was obviously usable by mainstream practitioners.  On the other hand, the approach did not help those people who found it difficult to create a model.  It had been hoped that the straightforward and prescriptive nature of the approach would guide people to create useful models, even if they did not find modelling easy, and it was a disappointment that the approach failed to achieve this.

  \section{Lessons Learned From The Project}

  At the start of the project, no one involved in it had much experience in using ADLs in a large-scale industrial context.  Our experience was limited to the use of ADLs in an academic context and some significant experience of using UML for architectural modelling in large industrial projects.  Therefore, we had relatively few preconceptions as to how successful the project would be and on the whole, we judged it to be a success.

\begin{minipage}{\textwidth}
  The main lessons that were learned during the course of the project were: \nolinebreak
\begin{itemize}
\item A specialised ADL can have benefits over a general modelling language like UML and even a simple ADL can be used to create useful results.

\item The more specialised an ADL is, and so the closer it matches the implementation style of the system being modelled, the easier people seem to find it to use.  While at first glance this sounds like an obvious point, it is contrary to the conventional industrial approach of using a general modelling language like UML or SysML and also contrasts with the domain-independent nature of most academically developed ADLs.
\item Carefully designing the detail of the graphical notation pays off.  Using shapes that hint at their meaning and using a range of graphical dimensions to differentiate shapes helps people to remember them, even if they don't guess the link between the shape and the concept themselves.  Again, this is not reflected in mainstream notations like UML or most existing ADLs, where little effort is made to identify meaningful symbols for concepts.

\item Consistency in the notation is very important and having a base shape for a general concept with refinements to it for different sub-concepts appears to help people when interpreting the diagrams.

\item Providing high-quality support materials including an example-based description of the approach and notation, a number of realistic completed examples and a set of templates for new documents is very important.  We found repeatedly that people are much better at "filling in the gaps" rather than following a set of instructions and creating something from scratch.

\item Utilising familiar tools helps with the acceptance of the approach.  In this particular organisation, there were no complaints or difficulties with the use of a wiki for the text and tables information, whereas a very widely used commercial drawing tool (Visio) caused problems, even with a carefully tailored template, because it was not widely used in the organisation already.

\end{itemize}
\end{minipage}

  These lessons aren't all that surprising but the importance of what seemed to be quite minor things (such as worked examples and quick reference cards) is important and is useful to bear in mind for the future.  The importance of matching the ADL to the specific domain being modelled is also a lesson that is not reflected in most modelling languages today, which tend towards the general rather than the specific.  

  \section{Validation} 
  \label{section:adlvalidation} 

  The goal of this work was to establish whether an ADL could be used to capture the architectural structure of a complex system so that the architectural description could be used to provide insight into the architectural properties of the system such as performance, resilience, modifiability or energy efficiency.

  Looking back to the specific goals we set at the start of the work, we considered whether the architectural description we had created was a useful catalogue of the current state, could allow the architecture to be understood to allow estimation of qualities such as its resilience, modifiability or energy properties, to provide a tool for impact analysis and to provide a reference to communicate the architecture of the system (see Section \ref{sec:overview}).
  
\begin{itemize}

\item Create a Catalogue of the Current State - the project created the first comprehensive description of the system and so provided a very useful descriptive catalogue of the current state of the architecture.  The weakness of the architectural description as a catalogue was that it was only as comprehensive as the authors of each piece decided to make it.  However, it was possible to cross-check it against a number of systems that were known to contain complete lists of the elements in the production system (as they were used for automated tasks relating to deployment).  Sampling about 30 per cent of the architectural description and cross-checking this against the lists of deployment elements revealed a high degree of completeness, so confidence in its use as a catalogue was high.

\item Allow Architectural Characteristics to be Estimated - as described earlier, the architectural description comprised a diagram and a set of tables and text for each major component within the system.  While this was an effective representation for human use and was useful for impact analysis (see below) it was not all that effective for any sort of architectural quality assessment, such as energy properties of a component.  This was because of the fundamental tradeoff between the flexibility, accessibility and usability of the notation and its formality.  While we were successful at enforcing conventions for the graphical models and for the representation of descriptive text and tables, it would not have been possible to persuade the development teams to use a formal,  checked notation.  Therefore we concluded that while the architectural description could have many uses, it would not be realistic to use it for any sort of automated architectural analysis.

\item Allow Impact Analysis - the architectural description quickly proved its worth for impact analysis and helped considerably with the process of understanding the impact of proposed changes.  This was primarily due to the fact that it allowed the interconnectedness of system elements to be quickly assessed, information that hadn't been easy to find before.  An example of this was a small project to migrate the interface to an important internal service from a legacy RPC technology to the strategic message-based interface.  The service interfaces were designed so that they could be used in parallel and the plan was to offer both and then slowly migrate users of the service to the new version.  The problem with this was the time it was going to take to find all of the users of the service and so the length of time that the parallel interfaces would be needed.  The model was in a late stage of development when this project started to think about migration and they were able to use it to discover nearly all of the clients of their service.  So rather than relying on a service provider keeping track of the users of the service, the model provided a structure to allow the users of the service to declare their interest in the services they used, which was a much more effective approach.

\item Communicate - the architectural description was quickly recognised to be a comprehensive knowledge base of the system's design information and so helped inter-team communication (when people in one team could use it to understand another team's subsystem).  An example of the model being used for this sort of collaboration was when a new application, which had been acquired as part of the acquisition of another firm, was being integrated into the existing application as a new subsystem.  The existing models helped the new team see how existing subsystems were integrated with each other and the model that the new team created of their subsystem helped the existing teams to understand what was being added to the system and how it might be used.  The architectural description also acted as a single place where further information could be gathered.  As mentioned earlier, the architects involved in creating the architectural description moved onto other work soon after its initial creation, but it does appear to have continued to be used, to grow and to evolve, suggesting that it did fulfil this role.  Eventually, it was adopted by the Production Services team in the firm, due to the value that they got from having up to date descriptions of the structure and dependencies of each application, for support tasks.

\end{itemize}

  We judged the project to have met the most of the goals we set for ourselves and in particular we met the goal to create a useful shared architectural description and this was confirmed as it became a useful resource within the organisation, within a couple of months, acting as a centralised and standardised source of design information for the system.

  Given the relative success of this project, it is natural to ask how generally applicable its results are and how repeatable it is likely to be.  Given what we learned during the project, particularly the fact that the specialised nature of the notation was a key factor in its success, we feel that these lessons may well have general applicability but only in the broad sense.  People like to be guided, they like familiar tools and techniques and they are unwilling to learn and use formal languages for design and architectural description.  However, the specific tools or techniques that work will be specific to each environment and people in different environments will have different levels of enthusiasm for learning new approaches.  However, when trying to get a significant amount of work done by people who are agnostic to the approach, familiarity and accessibility appear to help greatly with acceptance.

  Based on our experience, the specific suggestions that we would make for future modelling languages are as follows:

\begin{itemize}
\item Create a language that is specific to a domain (e.g. real-time control systems or enterprise information systems) and ensure that it contains the type of modelling elements needed in that domain.  Modelling languages also need to be easily extensible by their users, rather than modelling language experts, to allow missing element types to be added.  Of course specialising a language limits its possible user community but conversely, that user community is more likely to find a language that matches their problems useful and so is more likely to use it.

\item Spend time creating a rich visual notation that communicates as much as possible using the shape, line, fill and other visual aspects of the notation.  This makes diagrams much easier for people to understand.

\item Keep modelling languages as simple as possible so that people can start using them quickly without a great deal of training.  We have observed that modelling language constructs with complex or obscure semantics are rarely used correctly if they are used at all.

\item Consider how people will use the language and what they will need in terms of tools and facilities for structuring and managing large models.  Again simple tools (and ideally, extensions to tools that people are already likely to be familiar with) are much more likely to be successful than tools that require a lot of training and experience to use.

\item As well as the language and tools, develop the materials that people will need in order to successfully adopt the language for practical use.  This includes task-oriented training material, quick reference guides and plenty of samples which show the value of the language in use and provide people will examples of how to use it well (which they will almost certainly copy).
\end{itemize}

  It is worth noting that our experiences from this work and our resulting suggestions are similar to the conclusions of a major academic survey of practitioner requirements for 
  ADLs \cite{malavolta2013-industryadlneeds}, which suggests that these lessons and requirements reflect the needs of a significant number of industrial software architects.

  Beyond the experience we have gained in applying architectural description techniques to a large-scale problem, the particular notation and approach used in this paper may be of use to others but as explained earlier in the paper, this wasn't a goal of the project. While some of the aspects of the notation invented will be generally familiar (e.g. servers that are driven by messaging) the overall set of element types is specific to one environment and may well not be directly useful elsewhere.  We did not set out to contribute yet another general purpose ADL to the world, so reuse of the notation was not considered during its development.  We report this project in order to describe a successful application of the concepts of architectural description notations, to record the factors that we believe made the project successful and to capture the lessons learned and conclusions drawn from the experience.

\section{Conclusions}

  We wanted to investigate the practicality of using an ADL to describe a complex architecture to better understand it, and potentially allow its key quality attributes, such as performance and energy efficiency, to be analysed and understood.  To this end, we worked with an organisation in the financial services industry and created an architecture description for a large existing enterprise system.  In order to achieve this within acceptable cultural and time constraints, existing ADLs from the research domain proved to be unsuitable and so a simple, very specific architecture description language was defined in order to make the process of capturing the architecture description as simple and prescriptive as possible.

  While it was not clear at the outset whether this approach would be successful, our minimal ADL proved to be a helpful and effective tool for capturing this specific architecture description in an entirely industrial context.  The result was that a large, unified, architecture description was created, something that the organisation had not achieved before, and this allowed new understanding of the system to be gained.

  The factor that appeared to make the approach generally successful was the focus on describing the specific structures in the system of interest, rather than trying to create a general-purpose approach, which would be effective for other uses too.  Other factors which contributed to the success of the approach were its simplicity (which traded sophistication for accessibility), a carefully designed, consistent graphical notation, the availability of a large amount of tutorial and reference material to guide document authors, and the use of very familiar tools, which users of the notation were already familiar with.

  What the case study did not achieve was an architectural description that captured the system in a sufficiently precise manner to allow its quality attributes, such as energy efficiency, to be analysed and estimated.  This was the result of a fundamental trade-off that we discovered, between creating an ADL simple and specific enough for mainstream practitioners to use, versus using an ADL from the research domain that, while rigorous, would have been too general purpose, abstract and complex for mainstream practitioner use.

  We therefore reluctantly concluded that we were very unlikely to be successful in helping practitioners to understand the energy properties of their system by using an ADL as a fundamental part of the approach.  If we were to use an ADL from the research domain (such as ACME \cite{garlan1997-acme} or xADL \cite{khare2001-xadl}) it is possible that we could have successfully extended it to allow energy properties to be captured and analysed.  However our experience, as reported in this chapter, made it clear that this would not be used by practitioners, so defeating the object of the exercise.  On the other hand, if we used a very simple and tailored ADL, such as the one that was successfully used in this case study, then the resulting architectural description would not be sufficiently precise and standardised to allow the analysis of architectural quality attributes such as energy efficiency.

  In summary, while we achieved a great deal with our pragmatic, simple and context-specific ADL, we concluded that neither using an existing research ADL nor an extension to our lightweight ADL would be a practical approach for helping architecture practitioners understand and optimise the energy characteristics of their systems.


 % Ch3 - ADLs & Case Study

\chapter{Prioritising Architectural Effort}
\label{chapter:prioritisation}

\section{Introduction}

In our practice in the field of software architecture, we have noticed and experienced how complex it is for software architects to prioritise their work.  The software architect's responsibilities are broad and, in principle, they can be involved in almost any technical aspect of a project from requirements to operational concerns.  In practice, this makes it difficult for an architecture practitioner to prioritise their effort to achieve quality properties like energy consumption, which acquiring stakeholders and end-users rarely prioritise explicitly due to their lack of immediate visibility.  Other important quality properties that suffer from similar prioritisation problems include security \cite{cisco2016-uksecprioritisation} and performance and availability \cite{ozkaya2008-qualityproperties}.

In the case of energy efficiency in particular, a previous survey of practitioners \cite{bashroush2016-datacentreenergy} revealed that the vast majority of them (83\%) admit that they do not prioritise energy efficiency as a key quality attribute of their systems but that a clear majority of them (67\%) thought that energy would be a key architectural concern in coming years.  There are likely to be a number of reasons for the current situation, including lack of stakeholder focus, the perception of limited tools and techniques for application energy usage, and the relative complexity of addressing application energy consumption, as it requires a multi-disciplinary approach to be effective.

However, our experience with other quality properties, particularly scalability and security, suggests that even when relevant tools exist and there is general acceptance that the qualities are important, architects (and development teams more generally) often find it difficult to prioritise these concerns against short term priorities such as feature completion, fixing defects and performance fixes that address usability concerns.  While it is important to address these immediate concerns, they are short-term solutions, which do not improve the fundamental quality of the system that determines its long term viability. 

While addressing fundamental, rather than short-term, concerns is challenging, we observe that successful experienced software architects appear to be able to do this.  These experienced architects are good at focusing their effort for maximum long-term effect, and manage to create performant, secure, highly available systems, and we hope in the future, energy efficient systems too. However not all architects have this skill and anecdotally we observe that inexperienced architects often find prioritising effort very difficult.  This situation led us to wonder how the experienced architects achieve their balance between immediate and long-term concerns.  As discussed in \sref{section:litreview-prioritisation}, there is little to guide them in the existing research literature, although they may use simple techniques like ad-hoc prioritisation or numeric grouping.  They may also use generic time management techniques (like \cite{allen2015-gettingthingsdone}) but we were interested in whether there are common role-specific heuristics which could be taught to new architects.

We decided to investigate this via a questionnaire-based study of a group of experienced architects.  We discovered that there are common heuristics which experienced architects use to prioritise their work and we created a model to capture them.  We then validated the model via an online questionnaire with a much wider group of practitioners and refined the model based on their input.

In this chapter, we explain the approach we took and present both the initial model that we created from the results of the interview process and the final, refined, model that we created after the validation process.  The contribution of the work is not specifically the heuristics in the model, indeed most of them are quite familiar to experienced practitioners, but rather the organisation of the heuristics and the validation that they are used by experienced practitioners to guide their work.  We believe that this makes the model a useful reminder for experienced practitioners and an effective teaching aid for new architects who are learning how to perform the role.

\section{Research Method}

When planning this research, we selected a qualitative research approach because we needed to explore the "lived-experiences" of expert practitioners by asking them questions to encourage reflection and insight \cite{lapan2012-qualitativeresearch} rather than assessing performance or alignment with specific practices via quantitative means.

The process was organised into four distinct stages.

\begin{description}
	\item [Stage 1] gathering primary data using semi-structured interviews with practitioners.
	\item [Stage 2] analysis of the primary data and creation of a preliminary model.
	\item [Stage 3] validation of the preliminary model via a structured online questionnaire, completed by practitioners in relevant architecture roles (primarily software, solution and enterprise architects).
	\item [Stage 4] analysis of the validation data and refinement of the preliminary model into a final, validated model.
\end{description}

We chose to gather our primary data using semi-structured interviews, where we provided the interviewees with a written introduction to the question we wanted to answer and then some specific questions to start their thought processes. 

The analysis of the primary data was performed using a simple application of Grounded Theory as it is a suitable method for theory building, to understand the relationships between abstract concepts \cite{charmaz2006-groundedtheory}, which described our situation and needs very closely.  We performed initial coding on the primary data and then refined this with a more focused coding exercise.  As suggested in \cite{lapan2012-qualitativeresearch}, the process of collection and analysis was a parallel, iterative process, rather than a linear one with fixed phases.  

This exercise produced a set of themes that classify the heuristics that the architects use, as well as the heuristics themselves.  A heuristic had to be mentioned by at least three of the participants (which represented a third of them) for us to consider it significant enough to be included in the model.  We combined the themes and heuristics to form a simple model (the "preliminary model") of how experienced architects go about prioritising their effort. 

Once we had the preliminary model available, we published it at a research conference \cite{woods2017-archpriorisation} and via a LinkedIn post (https://www.linkedin.com/pulse/focusing-software-architects-attention-eoin-woods) and created an online questionnaire \cite{gillham2000-questionnaire} aimed at architecture practitioners to allow them to evaluate and comment on the usefulness of the model.  We publicised the survey via LinkedIn, Twitter and via direct email to our network of architecture practitioners.

We received 84 responses to the survey, containing answers to our closed-ended questions to evaluate the usefulness of the model and also received answers to open-ended questions in 50 of the responses.  We used the closed-ended questions to evaluate the usefulness of the model and analysed the open-ended responses to identify themes which needed to be addressed by the model.

The model was validated strongly across respondents from different locations, with different amounts of experience and from different architectural specialisations. Additionally, a small number of themes emerged from the answers to the open-ended questions.  These themes for improving the model were used to revise and extend it slightly, creating an improved final version that reflected the input from the respondents.

A description of the four stages of the research method is presented below, along with the final version of the architectural effort prioritisation model.

\section{Stage 1 - The Initial Study}
\label{section:initialstudy}

Our primary data gathering was performed using a semi-structured, face-to-face survey of 8 experienced software architecture practitioners working across 4 countries.

We found the participants by approaching suitable individuals from our professional networks.  We were looking for practitioners who had a minimum of 10 years' professional experience and who worked as architects in the information systems domain (rather than architects from \textendash for example \textendash embedded systems).  

We focused on the information systems domain because we know from experience that working practices differ between professional domains like information systems and embedded systems.  Hence, we thought it was more likely that we could create a useful model if we limited ourselves to one broad domain, at least initially. 

We deliberately selected candidates that we knew differed from each other in organisation, specialisation and geography to get a reasonably diverse population and avoid obvious sample bias (we discuss the threat of sample bias further in Section \ref{sec:threats}).

Some characteristics of the participants in the study are summarised in the graphs in Figure \ref{figure:participants}.  As can be seen, they represent a range of experience, role type and country.

%%
%% See this page for the magic:
%% https://tex.stackexchange.com/questions/64858/how-to-create-subfloat-figures-two-in-first-row-and-one-below
%%
%%       \includegraphics[scale=0.9, trim=5 5 5 5,clip]{Figures/prioritisation-roles}

\begin{figure}
   \begin{subfigure}{.5\linewidth}
      \centering
      \includegraphics[scale=1.0]{Figures/prioritisation-roles}
   \end{subfigure}
   \begin{subfigure}{.5\linewidth}
      \centering
      \includegraphics[scale=1.0]{Figures/prioritisation-yearsexp}
   \end{subfigure}
   \begin{subfigure}{\linewidth}
      \centering
      \includegraphics[scale=1.0]{Figures/prioritisation-countries}
   \end{subfigure}

   \caption{Study Participants (8 in total)}
   \label{figure:participants}
\end{figure}  

All of the members of our initial practitioner set had over 10 years of post-graduate experience and some had over 30 years of experience, so ensuring that they all had a significant amount of professional practice upon which to base their answers.

We deliberately selected software and enterprise architects because this is whom the model was primarily aimed at serving.

We approached individuals in a number of countries to try to minimise the risk that we would reflect practice only in one country, although we did not manage to gain representation from beyond North America and Europe.

We used a semi-structured interview format with a written introduction to the question which each interviewee read before being asked a standard set of open-ended questions which explored how they went about prioritisation of architecture work and any specific factors that they used to guide them.  

The question we asked was "how can the architect concentrate their attention so that they are most effective?" The more specific questions used to stimulate the thought process were: 

\begin{itemize}
	\item How do you go about this in your work? 
	\item What factors do you consider when prioritising your attention? 
	\item Do you consider what to focus on?   Or what not to focus on? 
	\item For example, how do you prioritise architectural governance compared to other aspects of the project?
\end{itemize}

The interviewer asked additional questions to understand the answers fully or to encourage the interviewee to add more detail or fill in ambiguous aspects of the answer.

The process of initial coding of the primary data resulted in 25 items, which could be associated with at least one of the interviews.  A further focused coding process revealed that there were 9 underlying heuristics which appeared to be significant to the participants in the study. Then, a further analysis iteration lead to the identification of three categories of prioritisation heuristic which we use to structure our model.

\section{Stage 2 - Preliminary Model for Prioritising Architectural Effort}
\label{section:prelim-model}

\subsection{The Preliminary Model}

Our preliminary heuristic model for focusing architectural effort is shown in Figure \ref{figure:prelmodel}.
 
The three categories of heuristic that the study revealed were: first, the need to focus on stakeholder needs; second, the importance of considering risks when deciding on where to focus effort; and finally the importance of spending time to achieve effective delegation of responsibilities.  These categories form the structure of our model and remind the architect of the general ways in which they should prioritise their efforts. The categories and heuristics are explained in more detail in section \ref{sec:prelim-model-content}.

\begin{figure}
\centering
\includegraphics[width=\textwidth]{Figures/prioritisation-prelim-model}
\caption{Preliminary Model for Focusing Architectural Attention}
\label{figure:prelmodel}
\end{figure}  

It is important to understand the nature of this model and how it should be used.  It is not a prescriptive process for architects to follow or a process for developing an architecture.  This model is an aide memoir to organise and present a set of heuristics that experienced architecture practitioners appear to find useful when prioritising their work.  While we believe this to be a useful model to teach trainee architects and a useful reminder for experienced architects, it is necessary to apply the model in a context-sensitive manner, within whatever method that the architect is using to develop software architectures.

\subsection{Content of the Preliminary Model}
\label{sec:prelim-model-content}

\subsubsection{Understand the Stakeholder Needs and Priorities}

The first theme which emerged strongly in our study was focusing on the needs and priorities of the stakeholders involved in the situation.  The principle that architecture work involves working closely with stakeholders is widely agreed upon \cite{rozanski2011-ssa2e, bass2012-sainp} and this theme reinforces that. Architects need to focus significant effort to make sure that stakeholder needs and priorities are understood to maximise focus on the critical success factors for a project and maximise the chances of its timely completion.  Based on the study, three specific heuristics to achieve this were identified:

\begin{itemize}
	\item \emph{Consider the whole stakeholder community}. Spend time understanding the different groups in the stakeholder community and avoid the mistake of just considering obvious stakeholder groups like end-users, acquirers and the development team.  As the architecture methods referenced above note, ignoring important stakeholders (like operational staff or auditors) can prevent the project from meeting its goals and cause significant problems on the path to production operation.
	\item \emph{Ensure that the needs of the delivery team are understood and met}.  Spend sufficient time to ensure that the delivery team can be effective.  What is the team good at?  What does it know?  What does it not know?  What skill and knowledge gaps does it have?  These areas need attention early in the project so that architecture work avoids risks caused by the capabilities of the team and that time is taken to support and develop the team to address significant weaknesses.
	\item \emph{Understand the perspective and perceptions of the acquirers of the system}.  Acquirers are a key stakeholder group who judge its success and usually have strategic and budgetary control, so can halt the project before delivery if they are unhappy.  Specifically addressing this group's needs, perceptions and concerns emerged as an important factor for some of the experienced architects in our study.  Acquirers are often distant from the day-to-day reality of a project and need clear communication to understand their concerns and ensure that they have a realistic view of the project.
\end{itemize}

\subsubsection{Prioritise Effort According to Risks}

During a project, an effective approach to prioritising architectural attention is to use a risk-driven approach to identify the most important tasks.  If the significant risks are understood and mitigated, then enough architecture work has probably been completed.  If significant risks are open, then more architecture work is needed. The specific heuristics to consider for risk assessment are:

\begin{itemize}
	\item \emph{Consider external dependencies}.  Understand your external dependencies because you have little control over them and they need architectural attention early in the project and whenever things change.
	\item \emph{Look for novel aspects of domain, problem and solution}.  Another useful heuristic, from the experience of our study participants, is to focus on novelty in your project.  What is unfamiliar?  What problems have you not solved before?  Which technology is unproven?  The answers to these questions highlight risks and the participants in our study used them to direct their effort to the most important risks to address.
	\item \emph{Identify the high impact decisions}.  Prioritise architecture work that will help to mitigate risks where many people would be affected by a problem (e.g. problems with the development environment or problems that will prevent effective operation) or where the risk could endanger the programme (e.g. missing regulatory constraints).
	\item \emph{Analyse your local situation for risks}.  Consider the local factors unique to your situation, which you will be aware of due to the knowledge you have of the domain, problem and solution.  It is impossible to give more specific guidance on this heuristic as every situation is different, but the participants in our study noted the importance of "situational awareness" \cite{wikipedia-sitawareness} that allows the architect to find and address the risks specific to the local environment (perhaps due to organisational factors, specific technical challenges, domain complexities or business constraints).
\end{itemize}

\subsubsection{Delegate as Much as Possible}

Delegation was an unexpected theme that emerged from our study. The architects who mentioned this theme viewed themselves as a potential bottleneck in a project and delegation and empowerment of others was a way to minimise this.  Delegation was also seen as a way of freeing the architect to focus on the aspects of the project that they had to focus on rather than all the other aspects that they could possibly get involved in.

The general message of this theme is to delegate as much architecture work as possible to the person or group best suited to perform it, to prevent individuals becoming project bottlenecks, allow architects to spend more time on risk identification and mitigation, and to spread architectural knowledge through the organisation.  The heuristics that were identified to help achieve this are:

\begin{itemize}
	\item \emph{Empower the development teams}. To allow delegation and work sharing, architects need to empower (and trust) the teams they work with.  This allows governance to become a shared responsibility and architecture to be viewed as an activity rather than something that is only performed by one person or a small group.  This causes architectural knowledge, effort and accountability to be spread across the organisation, creates shared ownership, reduces the load on any one individual and prevents reliance on a single individual from delaying progress.
	\item \emph{Create groups to take architectural responsibilities}.  A related heuristic is to formalise delegation somewhat and create groups of people to be accountable for specific aspects of architectural work.  For example, in a large development programme, an architecture review board can be created to review and approve significant architectural decisions.  Such a group can involve a wide range of expertise from across the programme and beyond, so freeing a lead architect from much of the effort involved in gathering and understanding the details of key decisions, while maintaining effective oversight to allow risks to be controlled and technical coherence maintained.  Similarly, a specific group of individuals could be responsible for resilience and disaster recovery for a large programme, allowing them to specialise and focus on this complex area, and allowing a lead architect to confidently delegate to them, knowing that they will have the focus and expertise to address this aspect of the architecture.
\end{itemize}

\section{Stage 3 - Validating the Preliminary Model}
\label{sec:validating-prelim-model}
\subsection{The Questionnaire}

Once we had a preliminary model, we wanted to validate its usefulness with a much larger group of experienced practitioners.  This was conducted using a structured online questionnaire.  The questionnaire asked the respondents to read the model and then comment on its credibility and usefulness.  We asked both closed questions, that asked respondents to rate the model on 5 point scales, and open-ended questions that allowed the respondents to consider whether there were aspects of focusing attention that we had missed and also to collect general comments on the model.  Finally, we asked some closed classification questions to allow us to understand who had completed the survey while preserving their anonymity if desired.

We asked three closed-ended questions to find out whether the respondent thought that the model was credible and useful.  These questions and the possible responses were:

\begin{description}
	\item [Q1] "Is this model similar to how you focus architectural attention in your work already?" (Not at all similar / Not Very Similar / Somewhat Similar / Quite Similar / Very Similar)
	\item [Q2] "Would you find this model helpful in guiding architectural attention for maximum benefit?" (Definitely Not / Probably Not / Possibly / Probably Yes / Definitely Yes)
	\item [Q3] "Are the areas of risk mentioned in the "Prioritise time according to risks" activity valuable?" (Definitely Not / Probably Not / Somewhat / Probably Yes / Definitely Yes)
\end{description}

The open-ended questions that we asked were:

\begin{description}
	\item [Q4] "Are there other general areas of risk that should be added to 'Prioritise time according to risks' that would be applicable to most (information) systems and environments? If so please list and briefly explain them."
	\item [Q5] "Are there any significant factors missing from the model which you use to focus your architectural work?"
	\item [Q6] "Do you have any other comments on the  model or the survey"
\end{description}

The closed-ended questions we asked to allow us to classify the respondents and their possible answers were:

\begin{description}
	\item [Q7] "What environment do you work in?"
	\begin{itemize}
		\item Industry (developing systems, consultancy and related work)
		\item Industrial Research (working in a research environment for an industrial employer) 
		\item Academic (teaching or researching in university or similar environments)
		\item Other (please specify)
	\end{itemize}
	\begin{minipage}{\textwidth} %nolinebreak ignored so force KWN via minipage
	\item [Q8] How many years of post-graduation experience do you have? \nolinebreak
	\begin{itemize}
		\item 1-5 years
		\item 5-10 years
		\item 10-15 years
		\item 15-20 years
		\item More than 20 years
    \end{itemize}
    \end{minipage}
    \vspace{1pt} % needed to make spacing look better after minipage
	\item [Q9] What is your main job role? \nolinebreak
	\begin{itemize}
		\item Software Architect
		\item Enterprise Architect
		\item Software Designer
		\item Researcher
		\item Other (please specify)
    \end{itemize}
	\item [Q10] Where in the world are you based?
	\begin{itemize}
		\item North America
		\item South America 
		\item Europe (including the UK) 
		\item Middle-East and Africa 
		\item Asia-Pacific
		\item Other (please specify)
	\end{itemize}
\end{description}

We also invited them to leave an email address if they wanted to be informed of the outcome of the study and provided a free text box for any final comments or questions.  We did not view the email address and the final text box as survey data for the purposes of analysis.

Having trialled the questionnaire ourselves and with two other individuals, we expected most respondents to take 10 - 15 minutes to complete it.

\subsection{The Respondents}

To use the questionnaire to validate the model, we needed to find a suitable set of architects who could read it and complete the survey for us.  We found our respondents via two main activities.  
Firstly, we created a LinkedIn post \footnote{https://www.linkedin.com/pulse/focusing-software-architects-attention-eoin-woods} which provided an outline of the model, explained that we wanted to validate it and whom we wanted to participate, and provided a link to the survey.  This was posted from my account and so tended to appear in the LinkedIn news feed of practitioners, as my LinkedIn network contains many more practitioners than researchers.  The post was further publicised via Twitter and Facebook to reach a larger audience.  This step was moderately successful, with the post being viewed about 700 times and about 25 people completing the survey successfully.

To gain more responses to the survey, the second activity was a targeted email sent to individuals that we knew personally and whom we knew were practising software related architects (including software, solution and enterprise architects).  This second activity resulted in about 60 more responses to the survey.

In total, we received 84 completed surveys.

The job roles of the respondents are summarized in Figure \ref{figure:resproles}.  About a third of the respondents identified themselves as software architects, about a quarter as enterprise architects, about 12\% as software designers, 10\% as solution architects, and 5\% as technical architects.  Four respondents did not complete this answer and four had other job titles (a risk assessor, a technical manager and systems engineer, a project manager and a strategy consultant).

\begin{figure}
\centering
\includegraphics[width=8cm,trim={2 2 2 2},clip]{Figures/prioritisation-detailed-roles}
\caption{Respondent Roles}
\label{figure:resproles}
\end{figure}


The work environments of the respondents were overwhelmingly industrial (we viewed those building systems in the public sector as part of "industry"), with only one respondent having a purely academic work environment.  Five respondents did not provide an answer to this question.  Of these five, two were from German IP addresses, one Swiss and two UK addresses.  Four were from private ISP connections and one was from a UK public sector body, which, with another respondent who identified themselves as working in the public sector, suggested there were at least two public sector respondents.
 
\begin{figure}
\centering
\includegraphics[width=6cm,trim={2 2 2 2},clip]{Figures/prioritisation-workenv}
\caption{Respondent Work Environments}
\label{figure:workenvs}
\end{figure}

We had some geographical distribution of respondents, as shown in Figure \ref{figure:geographies}, although there is a clear bias towards Europe, almost certainly caused by our professional networks being centred in Europe.  55\% of respondents identified themselves as coming from Europe, 30\% from the Americas and only 7\% from Asia-Pacific and a single correspondent from the Middle-East and Africa.

\begin{figure}
\centering
\includegraphics[width=8cm,trim={2 2 2 2},clip]{Figures/prioritisation-regions}
\caption{Respondent Geographies}
\label{figure:geographies}
\end{figure}

To delve a little deeper, we checked the geographical location of the respondents' IP addresses using the well-known "geoiplookup" command line tool \footnote{http://geoiplookup.net}.  Of the four respondents who did not answer the question, the IP addresses they were using were in the UK (two), Germany (one) and Switzerland (one).  While this does not prove that the respondents work in those countries (they could have been travelling away from home) it does make it likely that they are from Europe, taking the European percentage to about 60\%.
 
\begin{figure}
\centering
\includegraphics[width=12cm,trim={2 2 2 2},clip]{Figures/prioritisation-iplocation}
\caption{Respondent IP Address Locations}
\label{figure:iplocations}
\end{figure}

The result of using the respondents' IP address geographical locations to find which countries the questionnaire was completed form is shown in Figure \ref{figure:iplocations}.  While we must be cautious in assuming that these geographical locations are necessarily the locations of the respondents' homes and workplaces, it is a useful cross-check on the data.  We can see that 20 countries are represented, with 6 countries having 4 correspondents or more (which is roughly 5\% of the survey size).  These larger 6 countries are the UK, Canada, the USA, Germany, the Netherlands, Sweden, Australia and Romania.  Ten of the countries (Brasil, China, Columbia, France, India, Israel, Switzerland, Turkey and Uruguay) only had one survey completed from one of their IP addresses.  Just over 55\% of the responses were from IP addresses located in four countries, the UK (20\%), Canada (15\%), the USA (12\%) and Germany (11\%).

We discuss the possible impact of geographical location more when we consider threats to validity in section \ref{sec:threats}, but we think it is fair to say that we achieved good cross-geographic participation, but still ended up with a strong bias to Western Europe and North America.

The final classification we asked our respondents for was the number of years of experience that they had.  We asked this to ensure that participants had a significant amount of professional practice upon which to base their evaluation of the model and to allow us to understand if there are differences in the value of the model to architects with different amounts of experience.  The degree of experiences for our respondents is summarised in Figure \ref{figure:yearsexp}.
 
\begin{figure}
\centering
\includegraphics[width=8cm,trim={2 2 2 2},clip]{Figures/prioritisation-yearsexp-detailed}
\caption{Years of Experience of Respondents}
\label{figure:yearsexp}
\end{figure}

The obvious first impression is that our respondents were overwhelmingly experienced people, with over half of them (55\%) having at least 20 years of post-graduation experience.  At the other extreme only one respondent had less than 5 years of experience.  About 7\% had 5 to 10 years of experience, 15\% had 10 to 15 years and about 17\% had 15 to 20 years.  Four of our correspondents did not answer this question.

\subsection{The Results}

As mentioned earlier, we structured the questionnaire into two distinct parts, the closed-ended questions that asked people to rate the usefulness of the model and the open-ended questions that asked whether we had missed important risk areas to use with the prioritisation heuristic or whether there were any significant aspects of prioritisation that were completely missing from the model.

In this section, we review and analyse the responses to the closed-ended questions in the survey.
The first question we asked was to find out if the model was similar to how experienced architects already focused their attention, which would suggest that the model was credible and, if we assume that experienced architects are probably effective, a useful guide for less experienced architects, early in their career.  The responses we received from all of the respondents are summarised in Figure \ref{figure:similarity}.
 
\begin{figure}
\centering
\includegraphics[width=8cm,trim={2 2 2 2},clip]{Figures/prioritisation-similarity}
\caption{How Similar the Model is to Existing Practice}
\label{figure:similarity}
\end{figure}

Considering all of the responses, the model validates quite strongly against the participants' existing practice, with 75\% of respondents stating that it was "very similar" or "quite similar" to their existing approach for focusing attention in their architecture work.  20\% said it was "somewhat similar", 5\% said it was not very similar to how they worked, and no respondents replied, "not at all similar".

Given that we were interested in attracting experienced architects to validate the model, we were also interested in finding out whether the number of years of experience altered their view of how similar the model was to how they worked.  The summary of this information is shown in Table \ref{table:similaritybyexp}.


\begin{table}
\caption{Similarity of the Model to Practice by Experience Level}
\label{table:similaritybyexp}
\footnotesize
% {l p{0.5cm} p{0.5cm} p{0.5cm} p{0.5cm} p{0.5cm} p{0.5cm} p{0.5cm} p{0.5cm} p{0.5cm} p{0.5cm}}
\begin{tabular}{l rrrrrrrrrr}
                & \multicolumn{2}{P{1.5cm}}{Not at All Similar} & \multicolumn{2}{P{1.5cm}}{Not Very Similar} & \multicolumn{2}{P{1.5cm}}{Somewhat Similar} & \multicolumn{2}{P{1.5cm}}{Quite Similar} & \multicolumn{2}{P{1.5cm}}{Very Similar} \\
1 to 5 Years       && &   &        &    &       & 1  & (100\%) &	&        \\
5 to 10 Years      && & 1 & (17\%) & 2  &(33\%) & 1  & (~17\%) & 2  & (33\%) \\
10 to 15 Years     && & 1 & (8\%)  & 3  &(25\%) & 7  & (~58\%) & 1  & (8\%)  \\
15 to 20 Years     && &   &        & 1  &(7\%)  & 11 & (~79\%) & 2  & (14\%) \\
More than 20 Years && & 2 & (4\%)  & 10 &(21\%) & 22 & (~47\%) & 13 & (28\%) \\
\end{tabular}
\end{table}

The table shows how many respondents, grouped by years of experience, rated the model at each similarity level, compared to their own approach to prioritising their work.  The 4 participants who did not indicate their experience level are excluded from this table.  The percentage values are the percentage of the participants in the current row that the value represents.

We are primarily interested in the top three groups, which is architects with at least 10 years of experience.  What we can see from this data is that the degree of similarity between the model and the architect's current practice does vary between these three groups, with 66\% of the 10 - 15 year group rating it as "quite similar" or "very similar", 93\% of the 15 - 20 years group rating it at this level and 75\% of the 20+ year group rating it in this way.  So, all three groups validate quite strongly, but it seems to reflect most strongly how the 15+ year groups work.

The second question moved on to try to establish, whether the respondents thought that the model would be useful in practice.  A summary of the answers to this question across all responses is shown in Figure \ref{figure:helpfulness}.
 
\begin{figure}[h]
\centering
\includegraphics[width=8cm,trim={2 2 2 2},clip]{Figures/prioritisation-helpfulness}
\caption{Helpfulness of the Model}
\label{figure:helpfulness}
\end{figure}

Across all of the respondents, 70\% said that it was definitely or probably useful, which we interpret as a strong overall validation of the model.  The remaining respondents mainly stated that they might "possibly" find it useful.  Only 3 respondents said "probably not" and none said, "definitely not".
We were interested in how the utility of the model might vary by the experience of the respondent, so we performed a similar analysis by experience group to that performed for the previous question.  The results are shown in Table \ref{table:helpfulness}.

\begin{table}
\caption{Helpfulness of the Model by Experience Level}
\label{table:helpfulness}
\footnotesize
\begin{tabular}{l P{1.5cm} P{1.5cm} P{1.5cm} P{1.5cm} P{1.5cm}}
 & Definitely Not & Probably Not & Possibly & Probably Yes & Definitely Yes \\
1 to 5 Years       & &          &           &           & 1 (100\%) \\
5 to 10 Years      & & 1 (17\%) & 3 (50\%)  & 2 (33\%)  & \\
10 to 15 Years	   & &          & 4 (33\%)  & 5 (42\%)  & 3 (25\%) \\
15 to 20 Years     & &          & 4 (29\%)  & 5 (36\%)  & 5 (36\%) \\
More than 20 Years & & 2 (~4\%)  & 10 (26\%) & 23 (43\%) & 12 (26\%) \\
\end{tabular}
\end{table}

Again, focusing on the architects with at least 10 years of experience, 67\% of the 10 to 15 year experience group believe the model is "probably" or "definitely" useful, 72\% of the 15 to 20 year group also rate it at this level, while 69\% of the most experienced, 20+ years of experience group, believe it is "probably" or "definitely" useful.  A strong majority of respondents in all three groups appear to find the model useful, with the strongest validation coming from the 15 - 20 years of experience group.

Finally, we wanted to check that the areas of risk we had identified as important within the "prioritise time according to risks" heuristic were valuable to a practising architect.  The results of the corresponding question in the survey across all respondents are shown in Figure \ref{figure:validationofareas}.

\begin{figure}[h]
\centering
\includegraphics[width=8cm,trim={2 2 2 2},clip]{Figures/prioritisation-riskareas}
\caption{Validation of Risk Prioritisation Areas}
\label{figure:validationofareas}
\end{figure}

In this case we did have one very strong negative opinion ("definitely not") but this was a single individual (an enterprise architect in the 10 - 15 years of experience group, who commented in the open-ended questions that he did not believe that it was possible to define general software development risks in a useful way).

Beyond this response, 80\% of respondents believe that the areas of risk were "definitely" or "probably" valuable, suggesting that this aspect of the model should be of value to many practitioners.

Again, we analysed the responses by the experience level of the respondent, which is shown in Table \ref{table:valueofriskfactors}.

\begin{table}
\caption{Value of Risk Factors by Experience Level}
\label{table:valueofriskfactors}
\footnotesize
\begin{tabular}{l P{1.5cm} P{1.5cm} P{1.5cm} P{1.5cm} P{1.5cm}}
 & Definitely Not & Probably Not & Somewhat & Probably Yes & Definitely Yes \\
1 to 5 Years       &         &          &          &           & 1 (100\%) \\
5 to 10 Years      &         & 1 (17\%) & 1 (17\%) & 2 (33\%)  & 2 (~33\%) \\
10 to 15 Years     & 1 (8\%) &          &          & 5 (42\%)  & 6 (~50\%) \\
15 to 20 Years     &         &          & 3 (21\%) & 6 (43\%)  & 5 (~36\%) \\
More than 20 Years &         & 2 (4\%)  & 9 (19\%) & 17 (36\%) & 19 (40\%) \\
\end{tabular}
\end{table}


Looking at the architects with more than 10 years experience, we still see a high degree of validation (92\% of 10 to 15 year experience respondents, 79\% of 15 to 20 year respondents and 76\% of respondents with more than 20 years of experience saying "probably yes" or "definitely yes") but there is a larger range of opinion than before.

We had the single individual who strongly disagreed with the risk factors being valuable in the 10 - 15 year group, 3 respondents in the 15 to 20 years of experience group only feeling that they were "somewhat" valuable and 11 respondents with more than 20 years of experience stating that the factors were "somewhat" or "probably not" valuable.

This said, we still feel that the degree of validation that the model received across all of the experience levels indicates that the model has a high possibility of being useful to at least a majority of practitioners.

We were also interested to investigate if the key question of how useful the model was would vary significantly between our different respondent groups, particularly by job family and geography, which might suggest that the model aligned better with certain types of architecture work or practice in certain parts of the world.

We have analysed the responses to the question "Would you find this model helpful in guiding architectural attention for maximum benefit?" by role type and the results are shown in Table \ref{table:usefulnessbyrole} and Figure \ref{figure:usefulnessbyrole}.

\begin{table}
\caption{Usefulness of the Model by Role Type}
\label{table:usefulnessbyrole}
\footnotesize
\begin{tabular}{l P{1.5cm} P{1.5cm} P{1.5cm} P{1.5cm} P{1.5cm}}
 & Definitely Not & Probably Not & Possibly & Probably Yes & Definitely Yes \\
Software Architect	 &  &  1 (~3\%) & 11 (35\%)  & 13 (42\%) & 6 (19\%) \\
Software Designer    &  &  1 (10\%) &  ~2 (20\%) & ~5 (50\%) & 2 (20\%) \\
Solution Architect   &  &           &  ~2 (25\%) & ~5 (63\%) & 1 (12\%) \\
Technical Architect	 &  &           &            & ~2 (50\%) & 2 (50\%) \\
Enterprise Architect &  & 1 (~4\%)  & ~4 (17\%)  & ~9 (39\%) & 9 (39\%) \\
\end{tabular}
\end{table}

\begin{figure}[h]
\centering
\includegraphics[width=8cm,trim={2 2 2 2},clip]{Figures/prioritisation-usefulness-by-role}
\caption{Usefulness of the Model by Role Type}
\label{figure:usefulnessbyrole}
\end{figure}

As can be seen, all of the respondent role groups validated the model as "probably" or "definitely" useful, with the lowest validation (interestingly) coming from the Software Architect group, where 61\% of the respondents indicated this level of agreement (although nearly all of the remaining respondents were neutral - "possibly" - rather than negative).  In the Software Designer group 70\% of respondents, in the Solution Architect group 75\%, in the Technical Architect group 100\% and in the Enterprise Architect group 78\% of respondents considered the model to be "definitely" or "probably" useful.

Turning to the possible influence of geographical location, we analysed the same question as to the usefulness of the model by the respondents, by the geographical location that the respondents told us they were from.  The results are summarized in Table \ref{table:usefulnessbygeo} and Figure \ref{figure:usefulnessbygeo}.

\begin{table}
\caption{Usefulness of the Model by Geography}
\label{table:usefulnessbygeo}
\footnotesize
\begin{tabular}{l p{1.5cm} p{1.5cm} p{1.5cm} p{1.5cm} p{1.5cm}}
% slightly experimental bit here - aligns headers centrally and leaves the
% others with default alignment from the definition above
% https://tex.stackexchange.com/questions/2924/how-to-align-table-headers-differently-than-all-other-table-cells
 & \centering Definitely Not & 
   \centering Probably Not & 
   \centering Possibly & 
   \centering Probably Yes & 
   \centering Definitely Yes \tabularnewline
Americas              & &         & ~6 (23\%)  & 10 (38\%) & 10 (38\%) \\
Asia-Pacific          & &         & ~3 (50\%)  & ~3 (50\%) & \\
Europe (inc. UK)      & & 3 (6\%) & 11 (23\%)  & 22 (45\%) & 11 (23\%) \\
Middle-East \& Africa & &         & ~1 (100\%) &           & \\
\end{tabular}
\end{table}
 
\begin{figure}
\centering
\includegraphics[width=8cm,trim={2 2 2 2},clip]{Figures/prioritisation-usefulness-by-geo}
\caption{Usefulness of the Model by Geography}
\label{figure:usefulnessbygeo}
\end{figure}

In this analysis, we have included all of the regions, but in reality, the data for the Middle-East and Africa and Asia-Pacific is difficult to use with confidence as there were very few respondents from these regions (1 from ME\&A and 6 from APAC).  Therefore, while recognising the importance of these regions and their contribution to contemporary software engineering, we focus on Europe and the Americas for the purposes of this specific analysis.

We had 26 respondents from the Americas and 69 from Europe (including the UK).  Both of these contributions are large enough to be significant, being 31\% and 82\% of the total responses, respectively.

Of these significant geographical groups, the group from the Americas validated the usefulness of the model more strongly, with 76\% of respondents indicating that the model was "definitely" or "probably" useful.  For the European group, 68\% of the respondents indicated that the model was "definitely" or "probably" useful.

We interpret this data as suggesting that the model validates well in both the Americas and in Europe and should be useful in both regions.  The data we have suggests that it should also be useful in Asia-Pacific as 50\% of respondents indicated it would be useful, while in contrast, it may not be useful in ME\&A as our single respondent from that region indicated it was "probably not" useful; However, we do not have enough respondents from these regions to draw meaningful conclusions.  To investigate the usefulness of the model in APAC and the Middle-East and Africa would require a further study.

In summary, having analysed the answers to the closed-ended answers in our survey, we conclude that our model is likely to be credible and useful in Europe and the Americas and broadly aligns with the prioritisation approach used by many experienced architects in those regions.
We view this as a successful validation of the model; However, we were also interested in how the model could be improved and so we used the responses to the open-ended questions in the survey to find themes which we could include in a refined model.

\subsection{Analysing the Open-Ended Responses}
\label{sec:openended}

As explained earlier, we asked two important open-ended, questions, Q4, to identify missing risk factors from the "prioritise time according to risks" heuristic ("are there other general areas of risk that should be added to "prioritise time according to risks" that would be applicable to most (information) systems and environments?") and Q5, to ask whether we had missed any important aspect of the model entirely ("are there any significant factors missing from the model which you use to focus your architectural work?").

We had 44 responses to Q4, about missing aspects of the model, and 51 responses to Q5, about missing areas of risk.

Given the nature of these responses, we again used grounded theory style analysis to analyse them, coding each one initially using straightforward, descriptive labels, directly reflecting the language used in the response, then refining this with further coding steps, to identify higher-level categories to allow the responses to be collected into meaningful groups.

For the first question, Q4, we initially coded the responses to 37 distinct categories, plus two null categories for the initial coding of "None" and "General Comment" for those responses which were present but did not specify a new risk area or just made a general comment.  The responses suggested a diverse range of possible risk areas, and when we refined the coding to find common concepts, this resulted in 24 higher level categories.  We attempted to refine this further but did not find further meaningful refinements as we tried further rounds of coding (we judged that we had reached "theoretical saturation" with the data).   In addition, we had a very long "tail" of concepts with only a single mention in the responses, leaving us with 5 categories that had 4 responses or more: Organisational Environment (11 occurrences), Stakeholders (6 occurrences), Cost (6 occurrences), Time (4 occurrences), External Environment (4 occurrences).  We chose to focus on categories with at least 4 occurrences as this represents approximately 5\% of the total respondents to the survey and we judged this to be high enough to warrant consideration.

In the context of a survey with over 80 responses, none of the categories of response for missing risk factors was universally viewed as important, but we felt that it was important to reflect the fact that these four factors had been independently identified by a number of people.  Therefore, we decided to integrate these new factors into the refined version of the model.

For the second open-ended question, Q5, on missing aspects of the model, we initially coded the responses into 43 distinct categories, again plus "None" and "General Comment".  As we continued with the process of refining the coding further, we ended up with 26 higher level categories.  As with the responses to Q4, many of the categories were only mentioned once and only four were mentioned 4 times or more: Team Effectiveness (10), Benefits (7), Stakeholders (6) and Time (5).  Of these factors, "Stakeholders" are already a significant factor in the model and the comments provided in these cases were suggesting a particular emphasis on certain stakeholders or method of dealing with stakeholders.  The specific suggestions were all different and stakeholder needs and priorities are already a significant part of the model, so we did not feel that a new model element was needed, but we simply need to review the detailed comments and ensure that they are reflected in the existing elements of the model.

In this context, we felt that adding a completely new aspect to the model was a significant step and so we only wanted to consider this for aspects which had been identified as important by a significant number of respondents to the survey.  On this basis, we decided to add a new element to the model to reflect the "Team Effectiveness" theme as it was the only additional candidate aspect that at least 10\% of the respondents had identified as important.

Finally, we also received 37 general comments in the open-ended question at the end of the survey along with another 14 responses to the other open-ended questions which we judged to be general comments rather than specific answers to those earlier questions, making a total of 51 general comments.  We do not view these responses as part of the validation of the model, but some of them did provide useful commentary on the work.  Again, we used a grounded theory style coding approach to analyse the data and this resulted in 23 categories of comment.  Like the other open-ended questions, most of these categories were not judged as significant because less than four respondents mentioned them.

The categories which had four or more respondents were general Positive Comments such as "nice and simple model" (14), comments on the How the Architect Should Work, such as "an architect must help implement what he/she helped to decide" (6) and comments on the Presentation of the Model, such as "this model [\ldots] does appear to be a rather linear, and distinct,  in reality it [the process] is quiet iterative and overlapping" (sic) (5).

We found all of the general comments interesting and potentially useful, but most of them did not lead us to conclude that further changes were needed to the model.  The exception was the group of comments categorised as Presentation of the Model.  These 5 comments suggested that the respondents interpreted our graphical presentation of the model as indicating a linear "upfront" process.  In fact, we had meant to communicate exactly the opposite through our diagram, and indicate that the process is iterative and continuous, happening right through the project's lifecycle.  We took this as an important indicator that we needed to change the graphical representation of the model and also describe its intended iterative and continuous nature more clearly in the supporting text.

\section{Stage 4 - Refined Model for Prioritising Architectural Effort}
\label{sec:refined-model}

We took the outputs of the open-ended question analysis described in section \ref{sec:openended} and used them to add missing features of the model, improve the list of risks to suggest for time prioritisation and improve the model using the advice provided in the general comment responses to the survey.
The result of these additions is a refined model for prioritising architectural effort, with an additional feature of the model, "Team Effectiveness" and a refined list of risks for time prioritisation.  As mentioned earlier, we also decided to alter the graphical presentation of the model to try to emphasise that it is not a linear "process" but rather a set of activities to be performed throughout the project lifecycle.  The refined model is shown in Figure \ref{figure:refinedmodel}.

\begin{figure}
\centering
\includegraphics[width=12cm]{Figures/prioritisation-refined-model}
\caption{Refined Model for Prioritising Architectural Attention}
\label{figure:refinedmodel}
\end{figure}

The model is comprised of 4 aspects, Stakeholder Needs and Priorities, Prioritise Time According to Risks, Delegate as Much as Possible and Team Effectiveness.  Each aspect is a theme which our initial interviewees and the later survey respondents find useful when considering how to prioritise their architectural work.  The details of each theme are described in the subsections below. 

The idea of the model is to provide a guide for new architects, or an aide-memoir for experienced architects, on how to prioritise their architectural work in order to maximise its effectiveness.  It is not a process or a step to be followed in an architectural "method" but rather these are themes for effective effort prioritisation that should be repeatedly considered during the lifecycle of a project.
As with any set of heuristics, they can only be a generalised starting point and need to be considered, interpreted and applied in a context-specific way by the architects and teams who use them.  However, they have validated well against a reasonably broad survey of experienced, practising architects and so we believe that they are a useful starting point upon which to build a personal approach for prioritisation.

\subsection{Stakeholder Needs and Priorities}

The first theme which emerged strongly in our study was focusing on the needs and priorities of the stakeholders involved in the situation.  The principle that architecture work involves working closely with stakeholders is widely agreed \cite{rozanski2011-ssa2e, bass2012-sainp}, and this theme reinforces that. Architects need to focus significant effort to make sure that stakeholder needs and priorities are understood to maximise focus on the critical success factors for a project and maximise the chances of its success.  Three specific heuristics to achieve this which emerged from the study are:

\begin{itemize}
	\item \emph{Consider the whole stakeholder community}. Spend time understanding the different groups in the stakeholder community and avoid the mistake of just considering obvious stakeholder groups like end-users, acquirers and the development team.  As the architecture methods referenced above note, ignoring important stakeholders (like operational staff or auditors) can prevent the project from meeting its goals and cause significant problems on the path to production operation.
	\item \emph{Ensure that the needs of the delivery team are understood and met}.  Spend sufficient time to ensure that the delivery team can be effective.  What is the team good at?  What does it know?  What does it not know?  What skill and knowledge gaps does it have?  These areas need attention early in the project so that architecture work avoids risks caused by the capabilities of the team and that time is taken to support and develop the team to address significant weaknesses.
	\item \emph{Understand the perspective and perceptions of the acquirers of the system}.  Acquirers are a key stakeholder group who judge its success and usually have strategic and budgetary control, so can halt the project before delivery if they are unhappy.  Specifically addressing this group's needs, perceptions and concerns emerged as an important factor for some of the experienced architects in our study.  Acquirers are often senior managers and so may be distant from the day-to-day reality of a project and need regular, targeted, clear communication to understand their concerns and ensure that they have a realistic view of the project. 
\end{itemize}

\subsection{Prioritise Effort According to Risks}

During a project, an effective approach to prioritising architectural attention is to use a risk-driven approach to identify the most important tasks.  If the significant risks are understood and mitigated, then enough architecture work has probably been completed.  If significant risks are open, then more architecture work is needed. The specific heuristics to consider for risk assessment are:

\begin{description}
	\item \emph{Risks from external dependencies}.  Understand your external dependencies because you have little control over them and they need architectural attention early in the project and whenever things change.
	\item \emph{Risks from novel aspects of the domain, problem or solution}.  Another useful heuristic, from the experience of our study participants, is to focus on novelty in your project.  What is unfamiliar?  What problems have you not solved before?  Which technology is unproven? The answers to these questions highlight risks and the participants in our study used them to direct their effort to the most important risks to address.
	\item \emph{Risks in the organisational environment}.  Each organisation is different and there are nearly always risks specific to an environment such as the internal political situation, what is possible in the organisational culture, and the maturity of the organisation with respect to architecture, change and risk.  Different organisations also have different cultures and capabilities for funding change, which can create risks.  The speed which different sorts of risk and difficulties can be addressed can also be affected by organisational factors and so may cause you to change where you focus attention.  Participants in our study noted the importance of "situational awareness" \cite{wikipedia-sitawareness} that allows the architect to find and address the risks specific to their organisational environment.
	Risks from the external environment.  Nearly all organisations exist in a complex ecosystem of interacting partners, customers regulators, competitors and other actors and they can be a source of risk for many systems, as can general trends and changes in the industry that the organisation exists within (such as changing regulatory environment or industry-wide pressures such as reducing margins on products or services). 
	\item \emph{Risks related to cost and time}.  Most architects will report that they are often expected to achieve challenging goals in unrealistic timescales or with unrealistic cost estimations.  Many of our study participants reported that they needed to focus significant attention on risks resulting from cost and time.
	\item \emph{Identify the high impact decisions}.  Prioritise architecture work that will help to mitigate risks where many people would be affected by a problem (e.g. problems with the development environment or problems that will prevent effective operation) or where the risk could endanger the programme (e.g. missing regulatory constraints).
\end{description}

\subsection{Delegate as Much as Possible}

Delegation was an unexpected theme that emerged from our study. The architects who mentioned this theme viewed themselves as a potential bottleneck in a project and delegation and empowerment of others was a way to minimise this.  Delegation was also seen as a way of freeing the architect to focus on the aspects of the project that they had to focus on rather than all the other aspects that they could possibly get involved in.

The general message of this theme is to delegate as much architecture work as possible to the person or group best suited to perform it, to prevent individuals becoming project bottlenecks, allow architects to spend more time on risk identification and mitigation, and to spread architectural knowledge through the organisation.  The heuristics that were identified to help achieve this are:

\begin{description}
	\item \emph{Empower the development teams}. To allow delegation and work sharing, architects need to empower (and trust) the teams that they work with.  This allows governance to become a shared responsibility and architecture to be viewed as an activity rather than something that is only performed by one person or a small group.  This causes architectural knowledge, effort and accountability to be spread across the organisation, creates shared ownership, reduces the load on any one individual and prevents reliance on a single individual from delaying progress.
	\item \emph{Create groups to take architectural responsibilities}.  A related heuristic is to formalise delegation somewhat and create groups of people to be accountable for specific aspects of architectural work.  For example, in a large development programme, an architecture review board can be created to review and approve significant architectural decisions.  Such a group can involve a wide range of expertise from across the programme and beyond, so freeing a lead architect from much of the effort involved in gathering and understanding the details of key decisions, while maintaining effective oversight to allow risks to be controlled and technical coherence maintained.  Similarly, a specific group of individuals could be responsible for resilience and disaster recovery for a large programme, allowing them to specialise and focus on this complex area, and allowing a lead architect to confidently delegate to them, knowing that they will have the focus and expertise to address this aspect of the architecture.
\end{description}

\subsection{Team Effectiveness}

A theme that emerged when we validated our initial model with a wider group was the need to focus attention on making sure that the overall development team was as effective as possible.  The participants who indicated the importance of this factor were concerned with the need to develop the individuals in the team and to ensure that the team was as diverse as possible, to allow it to use a range of skills and perspectives when innovating and solving problems.

Other aspects of this theme that participants were concerned with were the importance of architecture work being used to quickly unblock the team when it hit difficulties and the importance of technical leaders, like the architect, to step in when needed to make sure that the team was functioning well and to address any dysfunctional behaviour observed.

The heuristics that the participants identified as being important for focusing architectural attention to achieve team effectiveness were:
\begin{description}
	\item \emph{Develop the team through mentoring}.  Every team should be on a collective journey towards improvement and hopefully, every individual in a team is on a similar personal journey to be the best that they can be.  People doing architecture work tend to be some of the most experienced people in a team and so a valuable and important area to focus attention, in order to achieve a highly effective team, is to spend time developing the individuals in the team, and the team as a whole, through thoughtful, intentional mentoring.
	\item \emph{Achieve team diversity for better innovation and problem-solving}.  In order to innovate and identify good solutions to problems, it is valuable to have a range of experience, perspectives and skills in the team.  Our study participants indicated that a valuable use of architectural time is to spend time building diverse teams that can achieve this.
	\item \emph{Remove blockers preventing team progress}.  Development and support teams often tend up blocked by technical or organisational factors, so spending time resolving these problems is a valuable focus for many architects.
	\item \emph{Address dysfunction in teams}.  Sometimes teams do not work well and it requires someone who is close to the team and respected by them, but outside the team structure, to identify the problem and suggest solutions.  Some architects work directly in individual teams and are not well placed to do this, but people doing architecture work are often close to the teams but outside their structure, and have the respect, soft-skills and experience to resolve team problems.  This use of architectural time can have huge benefits when dysfunctional behaviour is observed in teams.
\end{description}

\section{Threats to Validity}
\label{sec:threats}

We designed and conducted this study carefully to provide us with a useful model and a reliable evaluation of it, avoiding bias as much as possible.  Specific steps we took to produce a reliable evaluation of the model included focusing on the practitioner community (as they are the intended users of the model), focusing on experienced respondents who have the experience to evaluate the model, finding a reasonably large, geographically distributed group to validate it for us, structuring the questionnaire to allow disagreement as well as confirmation, and analysing the results in a careful, structured manner to allow the data to lead us to the conclusions, to avoid the danger of us subconsciously using it to validate an opinion we already held.  However, we acknowledge that there are potential limitations to any qualitative study, including ours, which can result in threats to our study's validity.

There are four main types of threat to the validity of a study like this, namely construct, internal, 
external and conclusion validity as defined in \cite{matt1994-threatstovalidity}. 

Construct validity is concerned with the relationship between theory and observation.  A commonly recurring threat when using questionnaires is the phrasing of the text, the questions and the responses to the closed-ended questions.  A second threat is where too many closed-ended questions are used and respondents cannot find suitable responses in the available set.  We addressed potential problems with wording by keeping the amount of text in the questionnaire as small as possible and using simple language that directly referenced the model.  The model itself was derived from the language and concepts that emerged from the semi-structured interviews.  We also tested the questionnaire ourselves and on a small number of other architects that we knew, to ensure that their interpretation of the questionnaire was as we expected and we refined the language slightly as a result.  We mitigated the possible problem of participants being unable to express their opinions through the closed-ended questions by limiting the closed-ended questions to being simple ratings and then providing open-ended questions for the participants to explain, expand or clarify their answers.

Internal validity is concerned with the validity of the causality relationship between the observations and the outcomes of the study.  We addressed this by using very straightforward analysis approaches, both for the statistical data analysis and for the analysis of open-ended answers, so the threats to the correctness of the analysis we performed are minor.  We also reviewed all of the answers from each respondent to ensure that they formed a credible and consistent set (which all did), so validating that the respondents understood the process and were expressing a coherent opinion.  To avoid possible misunderstanding we provided a clear definition of the model, links to additional information, trialled the questionnaire with people we knew, and included open-ended questions to allow respondents to express opinions that could not be easily captured using the closed-ended questions.

External validity is concerned with the generalisability of the results of the study.  The key risk we identified relating to external validity is an unrepresentative respondent population or a respondent population who lack competence in software architecture and so cannot validate the model effectively.  We mitigated these risks by finding a relatively large respondent population, who are distributed geographically, although as noted earlier, nearly all of the respondents came from Europe and the Americas.  So a residual risk we continue to have is the lack of representation from Asia, in particular, countries like India, China and Singapore, with significant software engineering populations and the potential for significant cultural differences from Europe and the Americas.  We mitigated the concerns around experience and competence by targeting experienced architects and architects in our extended professional network, whom we knew to be experienced and highly competent.  We know a significant percentage of the respondents at least slightly and through some informal sampling of employing organisations and job titles, have a high degree of confidence in the ability of our study participants to validate or critique the model correctly.  This leaves us with a residual risk that our extended network may be more likely to think similarly than a truly random sample, but anecdotally we do not believe that they are significantly different to most practitioners we have met over the years and we believe this trade-off to ensure credible study participants is the correct one.

Conclusion validity is concerned with the validity of the relationship between the data obtained in the study and the conclusions that have been drawn from it.  The threats that we identified in this area are whether we asked the right questions at each stage of the study, whether we made mistakes in analysing the data, and whether we introduced unconscious bias into the study which could invalidate our conclusions.  We mitigated the possibility of asking the wrong questions by using a semi-structured interview in the first stage and providing extensive opportunity for open-ended responses in the third stage.  We acknowledge that we could have made mistakes in our analysis and processing of the data, but we mitigated this by reviewing and cross-checking our work and using a simple, repeatable process which was straightforward to follow.  The largest risk to conclusion validity is probably the chance of introducing unconscious bias, as much of the study involved interpreting open-ended responses from the interviews and the questionnaire.  We attempted to mitigate this risk through the use of the grounded theory process, which helped us to be led by the data rather than trying to fit the data into a pre-existing theory.  We also reviewed our conclusions several times and repeated parts of the analysis if we felt that there was any danger of an alternative outcome being more representative.  We also did not have any preconceived ideas about the likely outcome of this study at the beginning, so did not have an underlying theory we were trying to validate.  Overall, we do not feel that we have been likely to introduce unconscious bias in the study, but we accept that it is hard to be certain that this did not occur at all.

In summary, we have designed and executed the study carefully, but do acknowledge that there are a number of threats to its validity which could threaten the generalisability of our results.  Probably the most severe threat to the global applicability of the model is the lack of study participants from Asia.  However, this threat does not suggest that the model will not be useful in Europe and North America, which would still be a valuable outcome.

\section{Conclusions}

Our experience and informal discussion with architects over many years suggested that they find it difficult to decide how to focus their effort to maximise their effectiveness and allow time to focus on architectural qualities, such as energy effiency, that were not immediate priorities of key commercial stakeholders but were clearly important to the long-term success of the system they were working on.  We were interested in how experienced practitioners solved this problem and whether they used common heuristics.  To investigate this, we used a four-step process of investigation.

We started with a semi-structured interview process with eight experienced practitioners.  The conclusion of the initial study was that there are some shared heuristics which practitioners use, but that the community of practising architects is not aware that the heuristics are common and shared.  We found that the heuristics clustered into three groups: focus the architect's attention on stakeholders, use their time to address specific risks and delegate as much as possible, in order to give them as much time for architecture work as possible.

We then created a simple structured model to capture and explain the heuristics that emerged from the initial study and we published this via Internet social media channels.  In the next step, we asked practitioners to complete a survey to comment on the usefulness of the model and whether anything had been missed.  84 responses were received to the survey, mainly from European and North American software, solution and enterprise architects with over 10 years of professional experience.

When we analysed the survey responses we found that the model validated well, as 70\% of the practitioners who responded to the survey think it would probably or definitely be useful, but we found that we had missed several important risk factors which are commonly used for prioritisation and we had missed an entire element of prioritising effort, which is the need to spend time to ensure overall team effectiveness.  We added these missing elements to the model.

These findings are not completely unexpected and many of the heuristics in the model are familiar.  However, neither the participants or ourselves knew that these were the important and shared heuristics before we undertook the study, so we believe that the model we have created will have value as a teaching aid and as an aide memoir for experienced practitioners.

Given the validation that we have achieved, the model should also be an effective technique to help architecture practitioners understand how to organise their priorities in a way that allows them to address quality properties, like energy efficiency and security, that key stakeholders often ignore when they prioritise the work for the development team.

In summary, this work contributes a useful, validated, model to help architecture practitioners to prioritise their effort effectively, but more specifically, it has the potential to guide practicing software architects to prioritise their workload such that they can address energy efficiency as a key system quality property.




 % Ch4  - Focusing Architectural Attention

\chapter{Design Principles for Energy--Efficient Applications}

%Use IEEE Software article plus some from "What Got Us Here" and expand both, then answer RQ.
%[RQ3] What design guidelines can we create to guide architects to improve energy efficiency of their systems?

\section{Introduction}

Digital-transformation initiatives have led to major efficiencies and cost savings, including the transition from paper-based processing to electronic documents and the use of traffic-routing algorithms for vehicle navigation. However, the software performing this remarkable work consumes nearly 10 percent of the world's electricity \cite{mills2013-digital-energyusage}. Today's cloud-based applications span multiple continents, consuming energy in servers, networks, cooling and power facilities, storage, and user devices.

In this chapter we present three simple design principles software architects can use to address system-level energy efficiency. A case study illustrates the energy savings possible with a holistic approach.

Over the past decade, researchers have been studying IT infrastructure energy consumption, working to increase datacenter, network, and hardware efficiency. Datacenter energy efficiency has improved considerably. For example, in the US, public-sector datacenters are now expected to operate at a power usage effectiveness (PUE) of less than 1.5, whereas a PUE of 2 was considered normal only a few years ago.

PUE is a datacenter's total energy consumption divided by its IT energy consumption, usually measured over one year. A PUE of 1.5 indicates that for every 1 KWh of IT load, a datacenter requires an additional 0.5 KWh.
Hardware has experienced a similar trend; computations per joule of energy have doubled every 1.57 years over the past two decades \cite{koomey2011-trends-energy-efficiency}. Yet, limited progress has been made in addressing the entire software system's energy efficiency. However, software engineering research is now focusing on a system-wide approach.

\section{The Challenge for Software Architects}

We suspected that software architects might find it difficult to prioritize energy efficiency for three main reasons. 

First, we have little understanding of how design decisions affect energy efficiency or other system qualities such as user experience, reliability, and performance. Without this knowledge, analyzing tradeoffs to elucidate the benefits or costs of improving energy efficiency is difficult. Minor system design changes could yield substantial benefits, such as avoiding unnecessary polling or eliminating redundant housekeeping tasks that prevent equipment from entering lower power states. However, a lack of relevant design tools and frameworks makes it difficult for architects to achieve more sophisticated optimizations that consider contextual information about the runtime environment.

Second, to achieve the next order of magnitude in energy efficiency, architects must think beyond traditional design boundaries. This will require that people from different specializations and departments work together. Such collaboration is challenging given current organizational software governance structures, wherein teams might have competing objectives, and human dynamics and political barriers. Moreover, existing technologies provide few mechanisms to allow communication across different technology layers (the application software, middleware, hardware, network, cooling, power infrastructure, and so on), which would enable cross-layer optimization.

Finally, end users rarely require or worry about energy efficiency. This is partly because of split incentives. System operators such as administrators or datacenter managers don't pay for the energy bill directly - the budget for energy tends to be included in a separate facilities budget owned by a different manager. This means that they would see little or no direct personal benefit from any energy savings that they achieve. However another important factor is that, given current energy prices, information and communications technology energy costs constitute less than 3 percent of a typical organization's budget. So, an organisation may not view pursuing savings in this area a priority, if there are larger costs to target. Exacerbating this problem is the lack of benchmarks, metrics, and reliable data that would allow realistic comparisons of different energy efficiency opportunities and their returns.

A previous survey attempted to understand architects' perspectives on energy efficiency [REF] and surveyed 12 representative, experienced architects from various organization types. They were asked whether they had encountered energy efficiency as an architectural concern in the previous five years and whether they believed that they had the right tools to address energy related challenges. The survey also asked whether the participants believed that energy would be an important architectural concern over the next five years.

The survey, while small, did include participants from a number of relevant sectors including 7 from technology consultancies, 1 from an Internet company, 2 from banking and finance and 2 from other instrustries.  Of these respondents, most of them (83\%) hadn't had to deal explicitly with energy concerns during the previous 5 years although interestingly 66\% of them thought that energy was an important concern that they would need do deal with over the next 5 years.  Given the state of the art, predictably, only 25\% of the respondents thought that they had the right tools to deal with energy as an architectural concern.  These findings are consistent with our industrial experience, where energy is rarely discussed as an architectural concern, when it is discussed it is usually seen as a hardware and data centre concern, and when application architects are concerned with it, they lack the methods and tools to allow them to understand and compare the energy implications of their architectural decisions.

\section{State of the Art}

To increase efficiency, we must be able to measure it. That is, we must be able to measure the useful work our software applications produce and the amount of energy this takes and then optimize the ratio between the two. However, although the datacenter world has metrics such as PUE, no comparable metrics exist for software.

A further complication is that modern applications run across multiple platforms (user devices, networks, computers, storage, and so on). Optimizing energy consumption across all these platforms will require a range of specialists to collaborate across traditional design boundaries.

Optimization must also consider key quality properties such as resilience (because redundancy in system designs is usually a major contributor to energy consumption), usability, and performance. In reality, however, we have no design tradeoff tools that let us conduct such analyses \cite{bashroush2016-datacentreenergy}.

Despite these challenges, energy efficiency has been gaining traction in software engineering. Much of the early research focused on measuring applications' energy consumption \cite{islam2016-energysoftwarefeatures} and tried to define useful work so as to allow the creation of useful metrics (for example, the DC4Cities project; www.dc4cities.eu). In parallel, other researchers have explored compiler optimization to decrease energy consumption or have evaluated design patterns' energy efficiency.

All these efforts have helped us begin to understand and optimize software applications. However, improving today's Internet-scale systems will require a more radical approach that considers the whole system. Such an approach is inherent to software architecture work.

\section{The Three Principles}

On the basis of early experiences and research in the field, we propose three simple architecture principles for achieving energy-efficient systems:

\begin{itemize}
\item \emph{Principle 1}. Energy efficiency metrics must relate business transactions to energy consumption in a meaningful way to key system stakeholders.
\item \emph{Principle 2}. Identifying sources of energy waste at the system level produces the biggest savings.
\item \emph{Principle 3}. Addressing the energy optimization problem requires a \\ cross-disciplinary team.
\end{itemize}

We now examine these principles in more detail.

\textbf{TODO} these can be expanded significantly

\subsection{Relating Business Transactions to Energy Consumption}

Energy efficiency must be measured in a way that system stakeholders can understand. Ensuring that the metrics are meaningful is necessary to convince senior management to sponsor optimization projects. Ultimately, suitable metrics can help achieve holistic system tuning and drive revenue and cost optimization.

- Tendency to view energy as "technical concern" so you never get attention from decision makers

- Energy needs to be seen as a cost for the business transactions

- Stakeholders care in different ways (acquirers, assessors, systems administrators, end users but only for , developers \& architects but only if transparent and has some cost to them - e.g. energy "budget") and the organisational view needs to be integrated to get a full picture (e.g. administrators don't pay for energy)

- Energy has a number of organisational impacts including ethical, reputational, environmental, cost, agility (do more with less); move to cloud focuses this more on ethical, environmental and cost.

- If energy efficiency is considered through organisational impact rather than as a technical concern or just a general overhead, we'll never improve.  We need to make organisational impact clear and this involves transparency of usage, impact and cost in the broad sense.  This needs to be translated for different stakeholder groups. 

\subsection{Energy as a System Level Concern}

- Current energy approaches tend to be at micro level (code procedure) or macro level (data centre) [mentioned earlier somewhere].  Neither of these is useful to improving application energy efficiency.  Micro level is too detailed for an architect to use at any scale or to have a big impact on a system as many components combine for a scenario and it's not clear which of the micro-level improvements have the biggest impact at system level.  Macro level also doesn't help as the only metrics available are maximums, minimums and averages on sections of the infrastructure estate (at best) and the application architect can't see how their decisions make a difference.

- Correct level to have an impact on systems is to consider the application as the system and then to analyse usage patterns (scenarios) rather than individual components.  Scenarios are how systems are used so understanding their energy characteristics reveals the applications real runtime energy characteristics.

- Once scenarios are understood, then impact can be understodd allowing effort to be focused on where it will be most effective.

Focus effort where it will be the most effective. For example, redundancy is a commonly overlooked source of energy consumption. To support resilience, redundancy is usually applied at all levels, including facilities, hardware, and software. Without system-level evaluation of resilience requirements, redundancy might be applied too generously. So, matching redundancy to actual requirements is a huge opportunity to achieve energy savings that would be difficult with local optimizations.


\subsection{Employing Cross-Disciplinary Teams}

Energy optimization requires design work across traditional design boundaries. For example, optimizing the design of resilience requires collaboration among infrastructure engineering, application development, and business teams. Without such collective efforts, improvements will be restricted to local optimizations, which often miss the bigger opportunities for savings.

- Related problem to the micro/macro problem is the "not my problem" situation where it's always someone else's problem - in production it's an "inefficent application", from the application perspective it's "ineffcient infrastructure with arbitary overheads".

- Information needed to address energy concerns is often scattered over organisational silos, with hardware energy consumption in one place, operating system counters in another, application tracing and monitoring somewhere else again, and data centre efficiency and overheads being yet another group.

- Skills needed to understand an address energy concerns also tend to be scattered over different teams as does the authority needed to make meaningful change and effectively evaluate the impact of a change, as this will be at multiple levels of abstraction and ranges of authority.

\section{Case Study: Online Auction Site}

The online-auction company eBay used principles such as those we have outlined to achieve significant energy savings.

As part of eBay's commitment to reducing its environmental footprint while decreasing costs and increasing performance, it introduced the Digital Service Efficiency (DSE) initiative \cite{ebay2013-digitalefficiency}. DSE relates business metrics such as customer buying and selling transactions to their energy consumption and environmental impact. DSE provides a set of easily understood metrics to help eBay understand, communicate, balance, and tune its energy consumption (principle 1). These metrics include: Buy Transactions / kWh; Sell Transactions / kWh; Revenue / MW; and CO2 emissions / Million users to name a few.

eBay identified reducing infrastructure redundancy as one of the main opportunities to save energy. To explore this opportunity, eBay rethought its entire system architecture (principle 2), taking into consideration its business needs and redundancy costs. eBay realized that no matter how resilient its back-end system was, if the end-user system failed, the whole session failed. This meant that responsibility for system resilience could be moved to the weakest link: the end-user system.

So, the company introduced a new system architecture that includes a service request proxy in the end-user system. The proxy identifies when a transaction such as a search request exceeds a timeout, perhaps owing to a service failure. When this occurs, the proxy transparently reissues the request to another service location, without notifying the user. Thus, noncritical services can be less resilient because the proxy's operation can mask their failures.

Furthermore, on the basis of business analytics, eBay estimated that only about 10 percent of its transactions were critical, such as payments. Payment handling has specific regulatory needs and requires a highly available infrastructure. On the basis of this insight, eBay processes payments at a specific, highly resilient datacenter that's a fraction of the capacity of its original datacenter \emph{[Dean Nelson's talk "How eBay's I and O Organization Is Supporting Business Initiatives" from Gartner Data Center Conf. 2013] - not available?} while processing the remaining workload in a cheaper, less resilient infrastructure. This significant improvement in energy efficiency required cooperation across eBay's engineering, operational, and business teams (principle 3).

Figure \ref{figure:styles} depicts eBay's original and resulting architectures. The new design allows for reduced datacenter redundancy while maintaining overall system performance and resilience.

\begin{figure}
\centering
\includegraphics[width=\textwidth]{Figures/principles-styles}
\caption{eBay's (a) original system architecture and (b) new design}
\label{figure:styles}
\end{figure}

eBay has achieved major capital and operational expenditure savings by adopting this new architecture. The new low-redundancy site's simpler requirements have substantially decreased the infrastructure, which in turn has significantly decreased datacenter build-out and fit-out costs and time scales. Redundancy costs are significant. For example, according to Steven Shapiro, the cost of building a Tier III datacenter is double that of a Tier II datacenter \cite{shapiro2015-datacentre-mythsrealities}. (The Tier Classification System is a widely used rating system for datacenter availability, with Tier I being the least available or redundant facility and Tier IV the highest \cite{uptime2015-tierclassification}.)

Even more important (from our perspective), eBay has reduced energy consumption by approximately 50 percent because the low-redundancy site requires fewer infrastructure components (for example, N + 1 rather than 2N + 1 redundancy).5 This has resulted in not only significant energy cost savings but also reduced maintenance and hardware refresh requirements, further lowering environmental costs.

\section{Conclusion}

There has been increased interest in reducing the significant energy costs of running large IT systems. However, software architects lack suitable tools and methods to address energy concerns when designing systems. With this challenge in mind, we've suggested our three principles, which architects can follow to make energy-related tradeoffs during system design even with today's limited knowledge and technology.

Despite these principles' simplicity, eBay's experience shows that they can yield significant cost and energy savings when applied to large-scale production systems. Savings of this scale are difficult to achieve through local optimizations, so we must rely on software architects' skills to lead our efforts in this emerging area




 %Ch5 - Design Principles for Energy Efficient Applications

\chapter{Monitoring Application Energy Usage During Operation}
\label{chapter:monitoring}

\section{Introduction}

As we discussed in \cref{chapter:introduction} the energy usage of information and communications technology systems is starting to receive much more attention due to its sharp increase in recent years and the predicted continual growth for the forseeable future.  Researchers from a number of domains have been working on the problem of ICT energy efficiency for some time and have made good progress in a number of areas including data-centre level efficiency \cite{dayarathna2016-dcenergy} and micro-level code analysis, allowing more energy efficient implementation options to be identified \cite{noureddine2015-hotspots}.

However as we also discussed, the problem for the software architect is that their interest sits between these two extremes as they need to understand the energy consumption at the overall application level rather than at the individual software module level or at the data centre level where many applications are consolidated.  As we saw during our literature review of this area \sref{sec:litreviewenergy}, some progress has been made in providing tools and techniques for application energy monitoring, but much of the technology developed is immature, relatively unproven, and unavailable for general use.

The other limitation of the existing research, from the architect's perspective, that we identified is that it monitors application energy consumption in terms of the energy properties of architectural components, operating system processes or even server machines.  An architect cannot gain immediate insight from these measurements as they are not related to the workload that was running on the system at the time. Therefore the architect has to run careful benchmark exercises in highly controlled conditions to allow them to understand the energy properties of the different types of workload that their system processes. This information is needed in order to allow them to decide which usage scenarios to focus their attention on.

The modern trend towards microservice-based systems \cite{wikipedia_microservices} makes things more complicated for the practitioner.  In principle it is possible to use or adapt the code-level energy estimation approaches to be useful with monolithic applications.  But this quickly becomes overwhelming with a microservice-based system, due to the number of system components and the complex ways in which they combine to process an incoming request.

In this chapter, we present the logical (i.e. technology independent) aspects of the design for a piece of technology that we have designed to address this problem by fairly allocating the energy usage of a host machine to the application elements running on it.  Using modern microservice and operating system technology including containers, tracing and resource monitoring, combined with energy statistics for a machine, we can provide the software architect, and also the host operator, with reliable estimates of the fair energy allocation of a machine's total consumption required to process requests for an application running on it.  This allows cost estimation but, more importantly from the software architect's perspective, the evaluation and exploration of architectural alternatives to minimise energy consumption.

The logical design in this chapter then forms the specification for our implementation of the ideas which we describe in Chapter \ref{chapter:implementation}.

\section{Motivation}

There are several reasons to seek a method of estimating energy usage by software applications, but two immediate motivating examples in our case are cost estimation and architectural evaluation.

Today, the energy consumption of an application is not taken into account when considering its cost to operate and so there is little motivation for the software architect to understand and minimize their application's energy footprint.  This prevents large possible reductions in energy usage and its associated resources of environmental impact and cost.

Even where the architect is interested in the energy usage of their application, no mainstream and practical approach for estimating the energy usage of software at the application level is available.  This means that for applications where a significant number of system elements are used to process a request, such as in a microservice-based system, it is theoretically possible to use program-level approaches to estimate the energy usage, but is complex enough that we do not believe that it would ever be done in practice.  This means that architects cannot evaluate the energy usage qualities of different architectural options that they are considering.

The specific advance achieved by this piece of work is the novel combination of scenario (application request) level resource usage statistics, total host resource utilisation statistics, and host energy consumption to create an approach to fairly allocate the energy consumption of the host to the workload that ran on it at particular points in time.  This allows a fair, reliable and useful estimate of the energy usage that should be allocated to of specific requests made to the application.

The goal of this is to provide a practical approach for software architects to estimate the energy impact of their applications and to evaluate different architectural design and deployment options in terms of their likely energy usage.

\section{Microservice-Based Systems}

The microservice architectural style \cite{richardson2018_microservices} is rapidly becoming a mainstream approach for building industrial software systems and it is systems build using this style that we are specifically interested in for this work.

A microservice-based system is made up of many small, encapsulated, network-connected services, rather than the more traditional approach of having a small number of large servers that aggregated many services (a so-called "monolith" \cite{mazlami2017_microservices_monoliths}).

For our purposes, the important characteristics of a microservice-based system are:

\begin{itemize}
\item The business logic in the system is implemented as a group of small, focused services, each performing one task, typically implementing a "bounded context" in Domain Driven Design terms \cite{evans2006_ddd}.
\item The services are as independent as possible and have well defined service interfaces and only interact through these interfaces.  Resources such as databases are owned by a specific service and are not accessed by other services (hence a microservice-based system will have many independent data stores rather than a single consolidated database used by many services).
\item Handling an incoming request for a system is likely to involve a set of cooperating services, with one handling the initial request and then calling other services in order to fulfil the request and provide a response.  Microservice-based systems often separate request-handling and domain services but we do not assume that this is necessarily the case.
\end{itemize}

Well designed microservice-based systems typically share other important characteristics \cite{newman2015_microservices}, such as independent build, test and deploy for each service, and interfaces defined using machine readable formats such as OpenAPI \cite{openapi2018} and RAML \cite{raml2018} but these other characteristics are not significant in the context of this work, and so we do not assume that they are present.

\section{Estimating Energy Usage}

As we investigated the problem of how to provide software architects, and other interested parties like data centre operators, with estimates of application level energy usage we identified a number of possible approaches, as described in \sref{sec:litreviewenergy}.  

\begin{itemize}
\item Other researchers have tried to create entirely \emph{model based approaches} (such as \cite{seo2008-energystyle}) which can allow relative energy usage between different architectural structures to be estimated, but sidestep the problem of calculating energy values, do not provide runtime measurement and require significant amounts of effort to create the models for non-trivial applications.

\item Other research projects have attempted to estimate architectural characteristics through \emph{event based simulation} \cite{grahn1998-energystyles} but creating event based simulation models is an unfamiliar process to many practitioners and again does not provide runtime monitoring and is significant amounts of effort for non-trivial systems.  It is also the case that existing research in this area has not yet investigated how to estimate energy consumption, but rather has focused on other architectural qualities such as performance.

\item Some researchers have used \emph{regression models} to provide an estimation of runtime energy consumption based on a number of resource consumption measurements for an operating system process \cite{phung2017-agnosticpower}.  As noted in section \sref{sec:litreviewenergy} these approaches have had little validation beyond small laboratory experiments and we have concerns about the amount of training and re-training that regression models are going to need to retain their predictive power in real environments.

\item Other researchers working on runtime energy estimation have used \emph{cost-based models} to provide an estimation of energy consumption based on the specifications of the hardware in use and key resource consumption measurements for an operating system process (typically CPU utilisation and in some cases network or memory utilisation) \cite{noureddine2015-hotspots, acar2016-beyondcpu}.  Most of these research projects have been focused on micro-level measurement rather than architectural measurement, but where the code is available for inspection, provided us with considerable inspiration for our approach.
\end{itemize}

However, as we mentioned earlier, all of these existing approaches suffered from the same limitation from our perspective, namely that they measure or estimate energy at the architectural component or operating system process level.  This of course can be useful, but it means that to get meaningful results the architect has to run tightly controlled benchmark scenarios and perform the measurement during the time period bounded by the start and end of the scenario.  Without this discipline, the energy usage of the components is not particularly useful, as the architect has no context as to the workload that caused the energy consumption.

That said, some of the research projects investigating code level measurement at the \emph{micro-measurement level} have created sophisticated approaches and tools to estimate energy consumption for individual algorithms or programs using cost-based models  \cite{noureddine2016-jolinar} with a high degree of success.  However we discovered that these approaches rely on a highly controlled execution environment for the code being measured and access to CPU-specific low-level hardware state metrics, such as processor frequency statistics, in order to deal with the complex power characteristics of modern computing hardware.

We explored whether we could combine the results of several of these projects who open sourced their tools \cite{noureddine2016-jolinar, bourdon2013-powerapi} with application-level resource usage measurements to produce meaningful energy estimates for the application-level workload.  However we found that their reliance on access to CPU-specific low-level hardware state metrics meant that their use isn't practical in large distributed execution environments such as those used by microservice-based systems.

Having considered these options and realised that each of them had practical limitations, we shifted our focus slightly and reconsidered the goal of the work.  The problem we aimed to solve was to provide software architects and other interested parties with information on how to improve the energy efficiency of their applications.  The solution we identified to this problem was to shift our goal from precisely estimating the \emph{actual} energy usage of a distributed application to estimating the \emph{fair allocation} of energy consumption to the service components of an application.  This is a useful goal because it allows service providers (such as cloud or hosting operators or data centre operators) to understand how to fairly allocate the cost of energy consumed by their infrastructure and it allows software architects to understand the energy consumption implications of their design decisions as a proportion of real energy cost, taking into account their deployment decisions as well as software design decisions.

Our approach assumes that the energy consumption of the execution platform hosting the application is available to the energy calculation process and uses this, along with the total resource usage consumption of the execution platform and the resource usage of the application components, to allocate the platform energy consumption to the different application elements running on it.  By tracing the execution of inbound requests across application elements, this allows us to allocate energy usage to specific workload scenarios and so compare the energy efficiency of different parts of a system, different workloads and different system design options.

Allocating energy usage at the application level is a complex process and so it is important to be clear how the calculation will be performed before trying to design an implementation of it.

There are five quite distinct parts to the problem:

\begin{itemize}

\item \emph{Identification of service elements involved in processing a request}.  Processing a request in a modern distributed system will often involve a chain of service calls between the services that comprise the system logic.  We assume that the implementation of the services is under the control of those wishing to estimate energy consumption.

\item \emph{Identification of the processing periods attributable to a request}.  Once we know the system elements involved in processing a particular request, we need to identify when the element was performing work on behalf of the request.

\item \emph{Resource usage of the request}.  Given the system elements involved in processing a request and the periods when they were active, on behalf of the request, we need to estimate the runtime resources consumed by the system elements during these periods.  The resource consumption we need to estimate is primarily the amount of CPU consumed (for reasons we will explain later) although the amount of active memory in use, the number of network i/o bytes sent and received, and the number of disk i/o bytes read and written are also typically available from modern operating system and container platforms.

\item \emph{Estimation of resource and energy usage of the underlying host machine}.  Our aim is to fairly allocate the energy used by a machine during a time window across the requests that were active during that period.  Hence we need to estimate the overall resource usage of the machine and the energy that it consumed as a result.  

% TODO - removed description of the PUE for the moment. Add back in
% if I get time to introduce this aspect.
%We also need a estimate of the PUE of the execution environment \cite{iso30134-pue}, to allow us to "load" the host's energy consumption to reflect the additional cost of the infrastructure (such as power transmission and cooling) that supports its operation.

\item \emph{Energy allocation of the request}.  Once we have the resource consumption metrics for a particular request, we then need to translate these into an estimate of the share of energy consumption that they imply.  We do this by establishing the percentage of overall machine resource utilisation that can be attributed to the request and then allocating the same percentage of energy usage of the underlying host to the request.

% TODO - more on PUE removed for now
% and then adjusting this by the PUE value for the execution environment (data centre) at the time of day that the request occured

\end{itemize}

Each of these aspects of the problem is reasonably complex to solve, but luckily they can largely be solved independently, and once resolved, combined to achieve a reliable and fair energy allocation for each inbound request for a microservice-based system.  In the following sections of this chapter we discuss the "logical" design for a solution to each of these aspects of a problem.  This provides us with a largely technology independent design of the solution, which we then use as a specification for the technology specific design that we implemented, as described in \cref{chapter:implementation}.

\section{Logical Design of an Energy Allocation System}

The functional design of a system to estimate energy allocation for application requests (which we have dubbed "Apollo" - the Greek God of Prophesy, amongst other things) is shown in the informal block diagram in \fref{figure:logicaldesign}.

\begin{figure}
\centering
\includegraphics[width=\textwidth]{Figures/estimating-energy-logical}
\caption{Logical Design of the Energy Estimation System 'Apollo'}
\label{figure:logicaldesign}
\end{figure}

The elements in the diagram with solid fill are the underlying runtime platform and services, the diagonally hatched elements are the application under study, the densely-dotted elements are data, while the sparsely dotted elements are the Apollo energy estimation elements.

The elements of the system are briefly described below:

\begin{itemize}
\item \emph{Application Elements} are the architectural elements of the application which is being studied for their energy consumption characteristics.  These are the main functional processing elements of the system, running in their own address space, providing a network interface to their services, invoked as part of request processing, with the implementation being under the control of the system owner who wants to estimate energy usage.  An example would be a request handling microservice to create an order.
\item \emph{Service Platform} is the system software which hosts the application components and can provide detailed measurements of their resource usage.  An example might be a PaaS platform like Cloud Foundry \cite{cloudfoundry2018}, a virtual machine, a modern operating system like Linux or a container platform like Docker \cite{docker2018} or Rkt \cite{rkt2018}.
\item \emph{Host Platform} is the hardware and operating system platform that provides the general computing platform that hosts the service platform and provides runtime statistics on the host's execution and energy consumption.  This is typically an Intel based server machine running a Linux or Windows operating system, or a virtualised version of them.
\item \emph{Tracing Mechanism}, which is key to our approach.  We require the ability to reliably and efficiently trace the architectural elements that were involved in handling a particular request for the system.  This tracing mechanism provides that ability, reporting the sequence of invocations of architectural elements and the time and duration of each.  Technology to allow this originated at Google \cite{sigelman2010-dapper} and has been implemented in open source projects such as Zipkin \cite{zipkin2018} and Jaeger \cite{jaeger2018}.
\item \emph{Resource Usage Records} is a database of resource usage statistics for all of the functional elements of the system, fed from the Service Platform and the Host Platform.  An example could be a regular database or a more specialised time-series database like Prometheus \cite{prometheus2018} or InfluxDB \cite{influxdb2018}.  The database is typically populated using a specialised statistics collection server like Telegraf \cite{telegraf2018} or cAdvisor \cite{cadvisor2018}.
\item \emph{Trace Records} is a database containing the request invocation traces from the Tracing Mechanism, which is likely to be a simple relational or document-oriented database, which the Tracing Mechanism writes its trace records to.
\item \emph{Apollo Energy Estimator} provides the key element of the energy estimation process, a calculator that works through the Trace Records and for each one, calculates which elements were active over which time periods, and uses the resource usage records to estimate the resource consumption of each request across the elements that were involved in processing the request.  These values are then compared with the overall host platform resource usage and the ratio of the values used to allocate the energy consumption of the host platform during the time period that the requests were active.

\end{itemize}

There are three key data structures in the design, which are critical to achieving the energy calculation, namely Trace Records, Resource Usage Records and Energy Usage Records.

\begin{itemize}

\item \emph{Trace Records} are created by the tracing mechanism and are used to record the invocation of system elements in processing a request.  There are two types of trace records, commonly referred to as Traces and Spans \cite{opentracing2018-traces}.  Common features of the two are the identity of the runtime element writing the record, the start time of the record and the end time of the record.  A "trace" record is written by the first element to handle an inbound request and its start time is when the request is received and its end time is when the response was dispatched.  A "span" record is written to record the invocation of another system element (i.e. service) by the original element handling the request and it has a "parent" attribute which indicates the trace it is part of.  Should an element \emph{e1} handle an inbound request and as part of processing it cause another element \emph{e2} to be invoked, then a trace record is written to record the execution of \emph{e1} and a span record is written, recording the execution of \emph{e2}, with the trace record as its parent.  Hence traces and spans are organized into a tree that mirrors the invocation structure of the request handling.  An example trace is shown in \fref{figure:span}.

\begin{figure}
\centering
\includegraphics[width=0.75\textwidth]{Figures/estimating-energy-trace}
\caption{Example Execution Trace}
\label{figure:span}
\end{figure}

The figure shows two traces, representing requests, one after the other.  The traces are distinguished by not having parent spans, both traces have child spans, as indicated by the dotted arrows.  The y axis indicates invocation from top to bottom, the x axis indicates the passage of time.  The first request has two child spans, indicating that it invokes two other services.  Child span "span1" in turn invokes a third service, which invokes a fourth.  The second request invokes a single service.

\item \emph{Resource Usage Records} are generated by the runtime platform and the host platform and stored in a timeseries style database to allow the metrics for a particular runtime element during a specified period to be retrieved (e.g. metrics for process \emph{p1} for the 3 second period from 20171105T084512.500 to 20171105T084515.500 or metrics for the host \emph{h1} for the 12.5 second period from 20171105T084512.500 to 20171105T084525.000).  The metrics will be generated by the underling platforms at regular but arbitrary intervals and so this data store will need to interpolate between the available measurement points to get the resource usage for the requested time periods.  The metrics that both the runtime platform and the host platform will be assumed to create are CPU usage (in a hardware based measure such as "ticks" or a time based measure such as milliseconds), memory usage (in KB), disk IO (in KB) and network IO (in KB).  Absolute or cumulative measurements are both usable for our purposes.  At this stage, we assume that we can obtain a reliable measure of the resource utilisation of each architectural element of interest, without being concerned with how this is provided.  Mechanisms which could provide this data include a container system like Docker, an operating system's process statistics, an internal application monitoring system or even a virtual machine manager with each architectural element in a separate virtual machine.

\item \emph{Energy Usage Records} provide a history of the estimated energy usage of the host platform (that is the virtual or physical hardware and its operating system).  There are two varieties of energy usage record source that could be available to us depending on the situation.  In some cases, there will be physical energy usage records available from data centre infrastructure sources such as DCIM data collection platforms, like Sunbird \cite{sunbird2018} or Nylte \cite{nlyte2018}, which extract data from enterprise class hardware devices through a protocol like IPMI \cite{ipmi2013} (and increasingly its more secure and capable replacement, Redfish \cite{dmtf2018-redfish}).  These records can be fed into a timeseries database (similar to the one used for resource usage records) and queried directly.  In other cases, a model based approach can be used to simulate real power readings.  A model based approach can be used to produce workable estimates of energy usage at different points in time by using accurate power ratings for different hardware devices such as those reported through the use of the SPEC Power benchmark \cite{lange2009-specpower}.  These benchmark results provide accurate power consumption rates for specific pieces of hardware at different degrees of utilisation.  If we know what host platform is in use and have accurate host platform resource usage records (for CPU usage specifically) then these can be combined with the benchmark results to create a usable model-based estimate of energy usage of the underlying host at a point in time.

\end{itemize}

\section{Utilising Resource Usage Records} \label{sec:utilisingresourceusage}

Modern computing platforms like Linux and Windows provide detailed and comprehensive resource utilisation statistics that describe CPU utilisation (typically in terms of milliseconds or nanoseconds of CPU time used), disk i/o (in terms of read and write requests, blocks and bytes performed), memory utilisation (as KB of memory used over time) and network i/o (in terms of receive and transmit requests and bytes) \cite{unix_sar_command, windows_performance_monitor}.

When estimating energy characteristics of a workload, it is possible to measure the power consumption of a server via DCIM monitoring equipment or alternatively estimate power consumption of a server using a model based on reliable benchmarking.  However it is difficult to attribute the overall energy consumption of a server to the specific components within in.  While it is possible to obtain representative power specifications \cite{hitachi_drive_data_sheet} for individual components of enterprise class computing hardware, interpreting such specifications to produce accurate power consumption estimates is extremely difficult as they require a detailed understanding of the physical operation of the hardware element concerned during the period of interest (such as the frequency of the CPU or the amount of time that disks spend spinning at full speed during a particular measurement period).

As a result, we concluded that combining CPU, disk, network and memory utilisation statistics in a cost based model was not a practical approach for anything other than the smallest laboratory examples and so was not suitable for the large scale enterprise use that we aspired to apply our work to.  While a statistical regression model might have been able to perform this task, given the right training set, as outlined in section \sref{sec:litreviewenergy} we had significant concerns about the usefulness of such regression models in practice beyond the laboratory.

Therefore in common with a number of other researchers \cite{amsel2010-greentracker, yuki2014-compilingenergy, kim2014-fullstackenergy, mobius2014-powerest} we decided to simplify our approach and focus our attention on CPU utilisation as a proxy for overall server power consumption. Previous research \cite{bashroush2018_hardwarerefresh} has shown that CPU utilisation is directly proportional to the power consumption of the server as a whole and supporting this assertion for memory usage, Chen and his colleagues found that "research on the power consumption of memory reports that the power consumption of memory remains constant regardless of the workloads" \cite{chen2014-automatedanalysis}.  From practical experience we are also confident that CPU varies roughly linarly with network IO given the large amount of serialisation and deserialisation work that characterises modern RPC protocols and libraries.  We investigated the relationship between disk IO work and CPU and in our testing we found that a linear relationship holds for that resource type too (see \cref{chapter:validation}).  Hence we have a high degree of confidence that this simplifying assumption is correct and that CPU utilisation is a good proxy value for the relative power consumption of a server computer.  

This assumption allows us to use CPU utilisation of the server host as a whole, compared to the CPU utilisation of the architectural elements used to process application requests, as a reliable proxy for the ratio of the energy consumption of the server to the energy consumption required to process application requests.

In the next section of this chapter, we explain how we calculate estimations of energy allocation for application requests.

\section{Calculating an Energy Allocation Estimate}

\subsection{Our Approach for Estimating Energy Allocation}

The key novel element in the system is clearly the Energy Estimator, which implements the processing required to fulfil the purpose of the system.  This element is a numerical calculator which combines data from the trace records, resource usage records and energy usage records in order to produce a reliable estimate of the amount of the energy of the underlying host which should be allocated to a specific request which the application under consideration has processed.

We illustrate the approach to creating an energy allocation estimate using the graph in \fref{figure:cpuusage}, which shows the cumulative CPU usage over time for two architectural elements \emph{E1} and \emph{E2} along with the corresponding cumulative CPU usage for the host they are executing on.

\begin{figure}
\centering
\includegraphics[width=1.0\textwidth,trim={2 2 2 2},clip]{Figures/estimating-energy-cpuusage}
\caption{Cumulative CPU for Request Handling}
\label{figure:cpuusage}
\end{figure}

The graph shows how CPU usage accumulates for an individual architectural element when it handles requests and how the CPU usage of the individual elements combine, along with other workload on the machine, to affect the host's CPU usage.

Starting from the left, the host and element \emph{E1} start up and use CPU during this process before becoming largely idle for a short period and so their CPU usage curves flatten.

Next element \emph{E1} receives a request, which we name \emph{r1}.  As can be seen, processing this request causes \emph{E1} to consume CPU rapidly, hence the CPU consumption curve becomes much steeper for a short period and the host's CPU usage curve follows a similar shape, reflecting the effect of \emph{E1}'s CPU usage on the host.  After request \emph{r1} has been processed, the CPU utilisation curves flatten again, reflecting the idle state that the application elements and the host are in.

The next part of the graph shows the arrival of another request, which we name \emph{r2}, processed by element \emph{E1}.  Similar to the previous request, this results in a short burst of CPU consumption by \emph{E1}.  However in this case \emph{E1} calls a second architectural element, \emph{E2} as part of the processing of the request.  At this point, this CPU consumption curve for \emph{E1} flattens as it is no longer consuming CPU, but \emph{E2}'s CPU consumption starts increasing rapidly for a short period.  Once \emph{E2} has completed processing, the host CPU curve flattens, indicating that request \emph{r2} has been processed.

Finally, element \emph{E1} receives a third request, which we name \emph{r3} and its CPU usage graph again becomes steep for a period as the request is processed.  Note that the CPU usage for processing request \emph{r3} is significantly greater than that which was recorded for processing \emph{r1} (and in fact \emph{r2}).  The host's CPU usage can be seen to increase in step, reflecting the impact of processing the request on the CPU usage of the host.

By comparing the shape of the host CPU usage graph to the usage graphs of the application elements it is possible to assess the amount of the host's CPU consumption which is attributable to the application elements.  In this example, the host CPU usage graph has quite a similar shape to application elements beneath it.  However careful inspection reveals that its slope is nearly always steeper than that of the application elements, reflecting the other workload that the host is undertaking in addition to that of the architectural elements.  

In this case, the fairly close match of the host usage curve to the application elements suggests that relatively little other workload is active on the host.  In the most extreme case, the host utilisation curve would be flat at the top of the graph, indicating close to 100\% utilisation and a very signifiant amount of other work being executed on the host in parallel with the application workload which we are studying.

It is this host CPU to application element CPU ratio which is key to our approach to allocating energy usage.  As explained above, we view CPU usage as a good proxy for the energy consumption of a host.  Therefore if an application's elements are consuming (say) 65\% of the host's CPU usage for a period, then we know that it is consuming roughly 65\% of the energy of the host during that period.

We use this insight as the key to designing an algorithm to allow us to estimate energy allocation for individual applcation requests, as explained in the next section.


\subsection{A Specification for Energy Allocation Calculation}
\label{subsection:calculation-specification}


We can express the specification for this algorithm as a series of simple equations as shown below.

As explained earlier usage records are \emph{Traces} which contain \emph{Spans}.  A \emph{Trace} represents an inbound request to the system and can be considered to be a tuple of the form 
\begin{equation}
Trace :: (tid, st, et, S)
\end{equation}
where $tid$ is a unique identifier for the trace record, $st$ is the start time of the trace, $et$ is the end time of the trace and $S$ is the set of spans that the trace contains.  

A span is a single invocation of an internal system element and so every trace contains at least one span (i.e. $\forall t : Trace \cdot t.S \neq \emptyset$) but most traces will contain a number of spans, related to each other, as illustrated in \fref{figure:span}.  A span can be considered to be a tuple of the form
\begin{equation}
Span :: (tid, sid, st, et, a, pid)
\end{equation}
where $tid$ is the identifier of the trace this span is contained within, $sid$ is the unique identifier of this span, $st$ is the start time of the span, $et$ is the end time of the span, $a$ is the address of the architectural element that the span is invoking and $pid$ is the unique identifier of the parent span of this span, if there is one (this is an optional element in the tuple).

For both traces and spans times are assumed to be recorded to millisecond precision and are conventionally represented as the number of milliseconds since an agreed point in time, by convention 00:00:00 on the 1st January 1970 \cite{josey2004-ieee1003}.

The address of the architectural element can be any unique identifier which can unambigously identify an architectural element and can be captured reliably by the tracing system in use.  In practice, this is usually a network address of the form $ipaddr:port$ where $ipaddr$ is an IP version 4 address (e.g. 156.76.45.32) and $port$ is an IP network port (e.g. 9745).  However our approach does not require addresses of this form and other addressing and identification schemes could be accomodated if required.  We discuss this important detail further in Chapter \ref{chapter:implementation}.

Given these structural definitions we can now define how the energy estimation process should work.

Given a trace $t$ we first define our period of interest $p$ to be the period between $t.st$ and $t.et$ and the set of spans of interest $S$ to te $t.S$.

Next we define the set $R$ of runtime application elements which were involve in processing the requests described by the set of spans in $S$, which are identified by their network addresses.  By "runtime application element" we mean any execution container for which resource usage statistics can be reliably obtained, such as a container, a process or a thread.

Let us assume we have a function $ator : (A,\mathbb{Z}) \to R$ which can map a network address from set $A$ to its corresponding runtime element at a point in time represented by an integer timestamp.  Such a mapping can be assumed to be available from the runtime platform in use.

We can now define the set $R$ to be the set of runtime elements corresponding to the network addresses found in the spans within the trace:
\begin{equation}
R = \{ ator(s.a) \mid s \in t.S \}
\end{equation}

Let us now consider how we estimate the CPU usage of a runtime element during period $p$.  

We make a key but reasonable simplifying assumption that a runtime element will only process workload on behalf of a single trace during a trace's execution period.  While this would be a significant constraint with traditional monolithic architectures, it is common practice in microservice-based deployments to deploy instances of microservices for the purpose of monitoring and testing in the production environment \cite{hilton2017-darklaunch} and microservice infrastructure, such as Istio is specifically designed to allow this \cite{istio2018-mirroring}.  Hence, temporarily dedicating a small part of the deployed application estate to processing requests for energy testing is not difficult or onerous in most modern microservice based systems.

For each runtime element, we have a series of resource usage samples at fixed intervals and we need to use interpolation between these points to estimate the CPU usage of runtime elements at arbitary points in time. The process is illustrated in Figure \ref{figure:graph}.  The figure shows an example CPU usage statistic for a particular operating system process, captured as a cumulative usage value over time.  Points in time $t_{1}$, $t_{2}$ and $t_{3}$ are sample points, when the CPU usage statistic is reported.  Points $st$ and $et$ are the start and end times of the period $p$ we are interested in.  We indicate the value of the CPU usage statistic at a point in time using subscript notation where $C_{t_{1}}$ is the value at point $t_{1}$ and $C_{st}$ is the value at point $st$.

\begin{figure}
\centering
\includegraphics[width=0.75\textwidth]{Figures/estimating-energy-graph}
\caption{Cumulative CPU Usage Graph}
\label{figure:graph}
\end{figure}

Given $r$ is a runtime element then we can describe the CPU usage of the element using the equations below, where $C_{r,t}$ indicates the cumulative CPU usage for the element $r$ at time $t$.

Let $C_{r, st}$ be the cumulative CPU usage for the element $r$ at point $st$ defined by
\begin{equation}
C_{r,st} = C_{r,t1} + \left( \frac{C_{r,t2} - C_{r,t1}}{t2 - t1} \times (st - t1) \right) 
\end{equation}

Let $C_{et}$ be the cumulative CPU usage at point $et$ defined by
\begin{equation}
C_{r,et} = C_{r,t2} + \left( \frac{C_{r,t3} - C_{r,t2}}{t3 - t2} \times (et - t2) \right) 
\end{equation}

Then the CPU usage of runtime element $r$ can be defined as
\begin{equation}
C_{r} = C_{r,et} - C_{r,st}
\end{equation}

Given $h$ which is the host that a runtime element was executed on, and $st$ and $et$, which are the start and end times of a period respectively then we can describe the CPU usage of the host in a similar manner.

Let $C_{h, st}$ be the cumulative CPU usage for the host at point $st$ defined by
\begin{equation}
C_{h,st} = C_{h,t1} + \left( \frac{C_{h,t2} - C_{h,t1}}{t2 - t1} \times (st - t1) \right) 
\end{equation}

Let $C_{et}$ be the cumulative CPU usage at point $et$ defined by
\begin{equation}
C_{s,et} = C_{h,t2} + \left( \frac{C_{h,t3} - C_{h,t2}}{t3 - t2} \times (et - t2) \right) 
\end{equation}

Then the CPU usage of host $s$ between times $st$ and $et$ can be defined as
\begin{equation}
C_{h,st,et} = C_{h,et} - C_{h,st}
\end{equation}

Having estimated the CPU usage of the runtime element during its execution period and the CPU usage of the host during the same period, we can use the ratio of the two, combined with the model-based or physical measurement estimates of the host's energy consumption over the same period.  Power indicates the rate of energy consumption at a moment in time and is a constantly and rapidly fluctuating value. We rely on averages of its value over time in order to find a representative value to use for our calculations.

In the situation where we have physical power metrics from the data centre infrastructure, then it is normal to find a Data Centre Infrastructure Management (DCIM) software platform deployed and we can extract power consumption metrics from this platform directly, using the API that such products expose. Different DCIM platforms offer different facilities and in some cases we might need to perform our own aggregation of device power readings in order to produce power consumption metrics suitable for our needs.  In other cases, the DCIM platform will have performed this aggregation and normalisation of the data already and we will be able to retrieve it via the API.  In ether case, the approach would be to produce a cumulative power usage metric for the hosts of interest in the managed environment, to allow a power usage average to be calculated in a similar way to the treatment of the cumulative CPU usage metric.

If a model based approach needs to be used then a slightly different it is calculated as follows:

Let $T_{h,st,et}$ be the total possible CPU consumption of host $h$ between start time $st$ and end time $et$.  Let $N_{h}$ be the number of CPUs that the host contains.

\begin{equation}
T_{h,st,et} = N_{h} * (et - st)
\end{equation}

Let $U_{h,st,et}$ be the percentage CPU consumption of host $h$ between start time $st$ and end time $et$.
\begin{equation}
U_{h,st,et} = \frac{C_{h,st,et}}{T_{h,st,et}}
\end{equation}

The power consumption estimate in watts, $P_{h}$, can now be extracted from the relevant SPECPower results for the model type of the host machine using $U_{h,st,et}$.

Let $E_{h,st}$ be the energy consumption (in joules) for the host between start time $st$ and end time $et$.
\begin{equation}
E_{h,st,et} = (et - st) \times 1000 \times P_{h}
\end{equation}

Given these supporting definitions we can now define the specification for the energy estimation process to be

\begin{equation}
C_{t} = \sum_{r \in R} C_{r,t.st,t.et}
\end{equation}
\begin{equation}
E_{t} = \left( \frac {C_{t}} {C_{h,t.st,t.et}} \right) \times E_{h, t.st, t.et}
\end{equation}

The energy allocation for a trace is the sum of the CPU usage for the runtime elements that processed its workload during its execution time, as a percentage of host resource usage during the trace execution period, multiplied by the estimated host energy usage during the trace execution period.

An algorithm which implements this specification will therefore allocate the energy used by the underlying host fairly across the application execution traces that it is executed for, so allowing fair comparisons to be made between different application requests to motivate design for energy efficiency and allowing recharging of energy costs in a fair and systematic manner.

\section{Implementing Apollo}

In this chapter we have explained the problem of providing energy estimates to a software architect and explored how this can be achieved in a realistic enterprise-computing environment through the use of a combination of resource consumption statistics, host energy consumption statistics and data centre efficiency factors.

So far our discussion has been largely implementation agnostic.  We have assumed the availability of the information we need, which presupposes a reasonably modern, mainstream enterprise application development and deployment environment (such as .NET on Windows or Java on Linux), but we have not limited or constrained our design by choosing specific technologies to use for data collection and analysis.  This means we have a specification for the software we need to build, which we could now implement using a choice of technologies for each part of the solution.

Realistically, when we choose technologies for each part of the solution they will solve parts of the problem for us and also constrain the solution and bring limitations and difficulties.  This is why presenting the implementation independent form of the design is important so that the underlying ideas can be explored and applied in a range of technology environments.  However we are aware that the design, and possibly its effectiveness, will be altered by the specific implementation choices we make when we build it.

In the next chapter we explain how we built a proof-of-concept version of Apollo, the technology choices we made, the problems we encountered and how we solved the problems to create a realistic and viable application energy calculator.

 %Ch6 - Monitoring Application Energy Usage During Operation

\chapter{Implementation of Application Energy Monitoring}
\label{chapter:implementation}

\section{Introduction}
The logical design of the Apollo energy estimation calculator was presented in Chapter \ref{chapter:monitoring} and showed how, in principle, we can build a useful energy calculator to fairly allocate energy usage of a collection of host machines, on the basis of the individual application requests processed in a period of time (rather than just the total resource usage of application elements over a period).   Such a calculator can provide a tool for a software architect to understand the energy consumption implications of their architectural decisions and can guide them towards higher energy efficiency for the applications.

The logical design of the calculator is independent of specific technologies and does not specify the details of the implementation, simply the operations that must be performed.  Hence, while it is a type of design, as we are now to implement the calculator it actually acts as our specification and our proof of concept implementation - Apollo - can be implemented in a number of ways using a number of different technology choices and detailed design decisions.

In this chapter we explain how we went about the task of implementing our proof-of-concept calculator, the problems we set out to solve, the choices we made, some of the problems we had to solve and some of the tradeoffs that were necessary.

\section{Initial Decisions}

In order to move forwards efficiently, we needed to narrow the design space for our problem to allow us to focus on the key decisions specific to the problem we were investigating rather than being distracted by the generic decisions that all software projects need to make.

These basic decisions were as follows:

\begin{itemize}
	\item As mentioned in the previous chapter, We will focus on the domain of \emph{microservice-based information systems}, such as those found in large enterprises, Internet-facing systems and Internet-oriented startup companies.  We will not specifically exclude the approach being used with other architectural styles, but where a design decision is required, we will assume that the approach will be used with a microservice-based system.
	\item We will assume that the primary technology "stack" used to implement the system will be \emph{enterprise Java}, meaning software written in Java, running on the JVM, using common open source frameworks and libraries like Spring Boot, Hibernate, Dropwizard, Spring Data, Apache Commons and so on.
	\item We will assume that the primary \emph{execution platform is Linux on Intel}, as this is something of a defacto standard for running large-scale enterprise Java systems.  Where it is possible we will try to make the approach execution platform agnostic (in particular to allow for Windows, another common server execution platform in the enterprise) but again, where a decision needs to be made, we will assume a Linux on Intel host.
\end{itemize}

All of these decisions align us with mainstream industry practice and provide practical options for applying our approach to industrial systems, while narrowing our design decisions to those related to our research problem, rather than generic concerns.

\section{Tracing Application Execution}

Our aim is to provide the application architect with insight into the energy consumption implications of their design decisions, and so simply measuring the energy of the infrastructure hosting the application does not provide enough information for this purpose.  The architect needs to understand the energy implications of the execution of different parts of their application, and most importantly, the energy consumption of certain types of workload.  This will allow them to understand the energy intensive parts of their application and workload and focus on improving the energy characteristics of these application elements.  Comparing the energy characteristics of different elements and workloads will also allow them to see the implications of particular design choices.

This requirement means that we need some way of tracing the execution of workload through the application to produce data equivalent to the traces and spans that we saw in Chapter \ref{chapter:monitoring}.  We considered a number of ways of achieving this.

\emph{Application specific tracing} could be provided by an application as part of its implementation and write special purpose log files or database entries to record how an application request is processed through the application's elements.  While straightforward from our perspective, this is a complex and potentially time consuming feature to add to an application and would be quite complicated to add to an existing system.  We think that this approach would be unlikely to be adopted in practice.

\emph{Application Performance Management} (APM) tools, such as AppDynamics, New Relic and Dynatrace \cite{appdynamics2018, newrelic2018, dynatrace2018} already perform application request tracing to allow them to measure and estimate application performance characteristics.  Initially using a tool like this as the basis of our approach was our preferred option.  In practice though, while attractive to practitioners who were already using the particular tool we would choose, it is a significant barrier to everyone else due to the cost and complexity of deploying these tools.  While we see our work as a potential extension of APM tools, perhaps providing them with a new dimension to their facilities, we decided against basing our approach on one of them.

\emph{Microservice tracing systems} like the open source Zipkin and Jaeger \cite{zipkin2018, jaeger2018} projects were a third alternative that we considered.  These systems are used by application developers to provide standardised trace data about the execution of their applications and provide collection and analysis infrastructure to allow the trace data to be easily used.  When we initially investigated them, they appeared to solve part of the problem of implementing application specific tracing, but still left the application developer with significant work to do.  As we investigated these open source products further we found that they are supported by or integrated into many commonly used application frameworks (for example Zipkin is already integrated into libraries for about 10 languages, including Java, Python, C\# and JavaScript, and just considering Java, it is integrated into more than 15 well known application frameworks including Spring Boot, Dropwizard, Google RPC, Apache HTTP Client and Jersey).  If using a pre-integrated framework, using these tracing systems is very straightforward from an application developer's perspective and normally simply involves starting the data server to receive the trace data and setting some configuration parameters in the framework configuration.

After some experimentation we found that the Zipkin tracing system worked very well for a set of Java microservices and its database was easy to query to extract the trace information we needed.  We concluded that the reliability, easy availability, low implementation overhead, ease of use and large number of existing integrations with widely used application frameworks made Zipkin a good choice for our work.

\section{Estimating Resource Usage of Application Workload}

Once we can reliably trace execution of application requests through the application elements involved in processing them, we can move to considering how to estimate the resource usage of those application elements in order to work out the resource consumption of the requests processed by the application.  The key requirement here is to be able to collect reliable samples of the resource usage of the application elements on a very frequent and predictable basis (e.g. every couple of seconds).  Ideally the samples will be in terms of cumulative usage rather than usage at that point in time, as these are much easier to use for our purposes.

The only practical source of application resource usage statistics is the application execution platform.  There are a number of sources of statistics that we could use, each with slightly different characteristics.

The simplest option is to use \emph{native operating system tools} such as \texttt{sar} and \texttt{pidstat} on Linux and \texttt{procmon} and \texttt{perfmon} on Windows.  In principle these tools can collect resource usage statistics for application processes on the machine.  However in practice, our industrial experience and recent investigation for this work, suggests that they are really intended for collecting host-level statistics or for interactive investigation of a performance problem on a machine, by a skilled administrator.  They do not provide an easy way to get a reliable stream of samples of cumulative usage over time written into an accessible form.

An alternative is to bypass the tools and access the \emph{operating system performance counters} directly.  Like most modern operating systems, Windows and Linux both implement a set of performance counters in their kernels, which are used for monitoring performance and throughput of workload executed by the machine.  These counters are used by the operating system tools to provide the data they need to operate and so by accessing the counters directly, we can avoid any limitations in the tools and still achieve consistent results.  Linux provides access to its performance counters via the very convenient \texttt{{/proc}} file system, which exposes all of the kernel's counters for global and process-specific metrics, via a pseudo file system interface (which can be read using standard text processing tools or through the standard file system API).  Windows provides access to its performance counters via the \texttt{perfmon} tool or through an API which is accessible via PowerShell (the modern Windows scripting language) or a conventional programming language.  In principle we can build any collection tool we want using these interfaces or we can use metrics collection servers such as \texttt{telegraf} or \texttt{collectd} \cite{telegraf2018, collectd2018} to read them automatically and store them in a suitable database for us.  During initial research, this approach was our preferred option, however in practice we found it quite difficult to get a usable set of statistics for our application.  The main problem we faced was that the collector does not know which workload on the machine belongs to a particular application.  Therefore we had to collect everything at operating system process level, which potentially generates huge amounts of unnecessary data.  We also found that the business of building a reliable collector was more difficult than initially assumed and that linking the trace data to the dataset we could collect easily from operating system counters was quite difficult to do reliably.  These difficulties were not insurmountable, but led us to consider whether there were other options we could consider.

The third option we investigated was to use \emph{Docker} \cite{docker2018} as a packaging technology for the application elements and to allow us to collect resource usage statistics.  Docker is an operating system virtualisation technology which uses operating system mechanisms (such as "cgroups" and "kernel namespaces" on Linux) to isolate processes from each other, providing the illusion that each is running on a separate machine.  Docker also provides a packaging convention that allows software to be packaged into reusable packages called "images" which are combined at runtime to form runtime environment known as a "container".  Docker also provides a set of management APIs and tools to allow containers to be interrogated, managed and controlled.  Docker is rapidly becoming a de-facto standard in the industry to package, deploy and manage microservices both on general host computers (where Docker becomes an execution environment on top of the operating system) and more abstract, container based platforms like Kubernetes \cite{kubernetes2018} where Docker forms part of a sophisticated platform providing demanding quality properties like scalability and resilience, across a cluster of host computers.

Our interest in Docker lies in its ability to provide a low-overhead, isolated environment for running application elements (in our case, microservices specifically).  If we can assume that all of the microservices within our system are packaged and then run as Docker containers then it provides us with the ability to extract accurage resource utilisation statistics for the microservice running in the container.  Another benefit of using Docker is that each container has its own network (IP) address which can be found via the runtime metadata available via Docker's management API.  This greatly simplifies the process of relating the resource usage data to the request traces from Zipkin as the network address is a shared piece of information between the two.

We believe that requiring application elements (the microservices) to be packaged and run as Docker containers is realistic and reasonable, given its wide and growing adoption in industry, particulary for microservice-based systems.  Therefore this combination of factors resulted in us deciding to use Docker as the basis for collecting application resource utilisation statistics.

The second part of collecting utilisation statistics is how they are extracted and stored from the execution platform.  In our case, our experimentation with Docker quickly revealed that a number of open source projects, including \emph{cAdvisor} \cite{cadvisor2018} and \emph{Telegaf} \cite{telegraf2018}, provide close integration with Docker and can extract and store the utilisation statistics data in different database systems.

After some experimentation we chose to use Docker with Telegraf, storing utilisation statistics in the \emph{InfluxDB} \cite{influxdb2018} open source time-series database to provide us with reliable application element resource utilisation statistics gathering.

\section{Estimating Resource Usage of the Host Platform}

Estimating resource usage of the underlying host platform is simpler than gathering the same information for the application elements, as host-level statistics gathering is a mature and widely used technology, provided by all major operating system platforms.

In our case we need to achieve reliable resource utilisation sampling of the host platform - the underlying Linux operating system - and have them stored in a database that allows us to extract them through a query interface to support the calculation process.  Ideally if the statistics are in a similar form to the statistics for the application level resource utilisation (e.g. the same timestamp convention and data types used), then this is likely to make the implementation of the calculator easier.

Our earlier investigation of the \emph{Telegraf} data collection server to extract and store application-level utilisation statistics revealed that it can be configured to extract and store host-level utilisation statistics too, and that this facility is provided as part of the standard distribution of the (open source) product.  When we tested the host resource utilisation statistics feature of Telegraf we found that it was straightforward to use and reliably stored accurate statistics in the same database as the application-level statistics.  The host-level statistics used the same basic conventions as the application-level statistics (e.g. they were both cumulative utilisation statistics using the same timestamp conventions).

Therefore we were able to solve the host-level resource utilisation statistics gathering problem by simply extending the configuration of the Telegraf server and the extension of the data set it stored to include host-level statistics.

\section{Estimating Energy Usage of the Host Platform}

As explained when we discussed the motivation for this work in Chapter \ref{chapter:introduction} the field of energy estimation is relatively young and reliable approaches for energy estimation of individual devices and applications are only just emerging.  Our work does not intend to address the problem of estimating energy consumption for a single host computer, but rather requires a reliable energy consumption metric to be available.

As explained in Chapter \ref{chapter:monitoring} there are two main approaches available to us that can provide energy usage of our underlying host platform, physical energy consumption metrics made available via a DCIM platform and model-based energy consumption estimation using machine utilisation levels and published benchmark results.

In many industrial situations a DCIM platform will be available and energy consumption metrics for all or a significant subset of host machines will be available through it.  However in our research environment we did not have access to such a platform and in many industrial situations this will be the case too.  While the state-of-the-art in organisation design is to integrate development and operations groups, they are frequently still separate and so even when a DCIM platform is available, software architects may well not be able to access it easily.

Therefore we decided to use a model-based approach to estimate the energy consumption of the host, but to ensure that it was easily replaceable with alternative models or with queries to a DCIM platform if one was available.

The approach used to estimate the energy usage of a host was explained in section \ref{subsection:calculation-specification} and relies upon published power consumption benchmark results associated with the SPECpower\_ssj 2008 benchmarks \cite{lange2009-specpower}.  An example data set for power consumption for a specific model of server host is shown in \ref{table:powervalues}.

\begin{table}
\centering
\caption{SPECPower 2008 Benchmark Results for \\Dell PowerEdge R730 (Intel Xeon E5-2699 v4 2.20 GHz)}
\label{table:powervalues}
\footnotesize
\begin{tabular}{|c|c|c|}
\hline
Machine Load & Power Consumption (W) & W / \% \\
\hline
\hline
Active Idle  &  44.6 & -    \\
10\%         &  84.8 & 8.48 \\
20\%         & 102.0 & 5.10 \\
30\%         & 120.0 & 4.00 \\
40\%         & 136.0 & 3.40 \\
50\%         & 150.0 & 3.33 \\
60\%         & 163.0 & 2.64 \\
70\%         & 181.0 & 2.58 \\
80\%         & 205.0 & 2.56 \\
90\%         & 238.0 & 2.64 \\
100\%        & 272.0 & 2.72 \\
\hline
\end{tabular}
\end{table}

We use this style of benchmark data combined with the machine type executing our application workload and the utilisation level metrics of the server executing the load to estimate host server energy consumption at a particular point in time, as explained in section \ref{subsection:calculation-specification}.

The third column in the table is a derived value we have added to show the power consumption per percentage point of server utilisation at each level of utilisation.  This clearly illustrates the need to keep servers busy from an energy efficiency perspective as it can be seen clearly that 1\% utilisation when the machine is quiet is three times more expensive in energy consumption terms than 1\% utilisation when the machine is busy.  We investigate this observation further in Chapter \ref{chapter:validation} when we describe how we validated the implementation of Apollo.

\section{Implementing the Calculator}

\subsection{Architecture of the Calculator}

The design of the Apollo calculator is shown in the simple block diagram in Figure 3. The system element filled with the fine dotted pattern represents the architectural elements of the application (i.e. an application microservice), the system elements filled with the fine cross-hatching are data elements, while the Apollo Energy Estimator is filled with the light solid fill.  The unshaded elements are the third party technologies which are reused unchanged as part of the implementation.

\begin{figure}
\centering
\includegraphics[width=0.5\textwidth]{Figures/implementation-design}
\caption{Implementation of the Apollo Energy Estimator}
\label{figure:implementation}
\end{figure}

The design elements of the calculator and their responsibilities are summarised in Table \ref{table:designelements}

\begin{table}
\centering
\caption{Apollo Energy Calculator Design Elements}
\label{table:designelements}
\footnotesize
\begin{tabular}{|l|p{10cm}|}
\hline
\textbf{Design Element} & \textbf{Responsibilities}  \\
\hline
\hline
Application Service & This element represents the regular microservices that comprise the application under investigation.  The implementation of these services are under the control of the development team and they have the responsibility to generate Zipkin trace records (via the Zipkin Client library) to record their activity (although this will usually be achieved automatically through use of an application framework like Spring Boot). \\
\hline
Zipkin Client & A trace of the invocations to and between application elements is required and as explained above, the Zipkin tracing system is used to achieve this.  The Ziplin Client is a client programming library used by Application Services to generate trace records and forward them to the Zipkin Server for storage.  The application code may invoke this library directly or it may be invoked automatically by an application framework like Spring Boot or Drop Wizard. \\
\hline
Zipkin Server & The Zipkin server receives and stores the trace records from the Application Services.  One Zipkin Server is used for all of the Application Services in a monitoring context. \\
\hline
Trace Records & The Zipkin Server persists the trace records in a well defined schema in a database.  In our case we used MySQL as the database for the trace records. \\
\hline
Docker Container    & All application elements need to run within Docker containers.  This allows metadata about the elements to be retrieved and resource usage statistics to be gathered.  This container contains the Application Service (each service is packaged in a separate container to allow it to be monitored separately).\\
\hline
Docker Runtime     & The Docker Runtime is part of the Docker system software package and provides the control and monitoring of the Docker containers in the application and provides runtime statistics for the containers, which in our situation are streamed to the Telegraf Server for storage. \\
\hline
Linux/Intel Host   & The application runs within Docker on the underlying host machine(s) and we have chosen to use an Intel host running the Linux operating system, due to the maturity of both Java and Docker on this platform. \\
\hline
Telegraf Server & The Docker platform produces a stream of resource usage statistics for the containers that it is executing.  The Telegraf open source metrics collection agent collects these metrics and stores them in a timeseries database for easy retrieval. \\
\hline
Resource Usage Records & The Telegraf Server generates a stream of resource utilisation statistics by constantly querying the Docker Runtime and the Linux Host.  These statistics records are persisted to a database for later use.  The datastore used for the Resource Usage Records is InfluxDB, an open source timeseries database. \\
\hline
Docker Network Map & The Zipkin traces and the resource usage records identify the runtime elements of the system in different ways; the Zipkin traces are collected at the network level and so identify elements by IP address and port number, while the Docker resource usage statistics are identified by Docker container ID.  Hence metadata is needed to link the two together and this is the purpose of the Docker Network Map which is metadata available from the Docker Runtime (through the \texttt{docker network inspect} command) which allows us to find the IP address(es) in use by each container during the execution of the application. \\
\hline
Apollo Energy Estimator & The Apollo module collects data and implements the energy allocation algorithm described in Chapter \ref{chapter:monitoring}.  This module is the primary software that we have implemented as part of this research (along with an example application we use for validation, which we describe in Chapter \ref{chapter:validation}). \\
\hline
\end{tabular}
\end{table}

The significant piece of custom software developed for this investigation was the Apollo Energy Estimator module.  We describe its software design in the next section.

\subsection{The Design of the Apollo Energy Estimator}

TODO




\subsection{The Calculation Algorithm}

TODO

\section{Limitations of the Calculator}

The design described in this paper has been validated via proof-of-concept implementations of each of its significant decisions and mechanisms, but not yet built.  Therefore the next step in the work is a full, robust implementation which can be utilised and validated to create a useful energy estimation mechanism.
The design described here is the first development of the ideas and has a number of limitations, which can be addressed in future work.

Our approach estimates energy usage for individual requests, by estimating the energy usage of each container as the request is processed.  This allows some degree of isolation as other workload which does not use these containers can be running in the application without it distorting the results.  For Application Service elements this is useful, as they are usually replicated for scalability reasons and so workload can run in the replicas that are not processing the request of interest.  However, care must be taken not to have workload running through shared services (such as databases) which would distort the energy estimation.  This is a limitation of the current version of the design. In practice, we believe that architects are likely to expect this constraint, as it is similar to the situation for other non-functional tests like performance and scalability.

The use of the architectural description data is somewhat unsatisfying as the rest of the inputs to the process are automatically generated.  The creation of the architectural description is a burden for the architect and could be performed incorrectly.  A more sophisticated tracing system, which can trace the activity of “black box” components could go some way to addressing this and could potentially be achieved by lower-level tracing in the operating system and network layer.

The system does not provide the architect with any visualisation or analytics view of the data in order to help draw insights from it.  While not the focus of this work, there are a number of well-known tools (such as Graphana) that could be used to create a visual interface for understanding the results of the analysis.

Third, the energy estimates for each trace and span are derived from sampled resource usage statistics for the corresponding container.  The sample times will very rarely align with the start and end times of the traces and spans and so estimates will need to be made, based on the samples available.  Depending on the sampling interval of the resource monitoring and the timing of the traces and spans, it is possible that the simple estimation approach we use could prove to be inaccurate (perhaps missing peaks and troughs between sample intervals).  The sample intervals we achieve in practice are short (TODO - HOW LONG?) and so we do not believe that this is a significant limitation, but does need to be monitored in practice.

\section{Summary}

TODO


 %Ch7 - Implementation of Application Energy Monitoring

\chapter{Validation of Application Energy Monitoring}
\label{chapter:validation}

%%
%% TODO
%%
%% Possible additional testing:
%%   - Run multiple service type tests (e.g. mixed data and CPU)
%%   - Run multiple service instance tests (e.g. two CPU Hogs)
%%   - Run multiple machine tests
%%

\section{Introduction}

As described in \cref{chapter:implementation} we have implemented a proof-of-concept version of the Apollo Energy Allocation system, to prove the usefulness of our approach for allocating the energy consumed by underlying host systems to individual application requests executing on them.  The software was tested during development to ensure correctness with respect to expected results through unit and integration tests, but now needs to be validated by using it in realistic test cases.  This process is described in this chapter.

In order to validate Apollo with realistic test cases, we need to define the kind of validation we are interested in achieving.  There are three main varieties of validation that we wish to achieve.

Firstly, we wish to validate the \emph{calculation correctness} of Apollo's results, by running the calculator in one or more controlled scenarios where we can also gather additional runtime statistics that allow a separate independent calculation of a fair energy consumption and manually perform these calculations and use them to check the correctness of Apollo's results in the same scenarios.

We also need to validate the \emph{consistency} of Apollo's results across a range of scenarios, to ensure that energy is allocated consistently with respect to the workload in the execution scenarios and host utilisation levels that prevail during them.

Finally, another important area of validation is the \emph{allocation} of Apollo's results when a specific application scenario is run on a host machine with different amounts of competing workload.  As the competing workload rises, the energy allocation to the application scenario should fall, in proportion to its use of the machine.

In addition, another area of validation we wish to perform is to confirm that \emph{CPU is a valid proxy for overall resource usage} when performing energy allocation calculations.  For this validation we focus on how CPU usage varies for disk IO intensive workloads.

Throughout this validation testing we aim to achieve consistency of results to within 5\% tolerance.  Long experience teaches us that a high degree of consistency is very difficult to achieve in any performance or resource utilisation testing due to the number of factors that can affect the results of a test on a modern multi-cpu, multi-core server machine.  Our experience in previous testing work is that a 5\% tolerance in most cases is equivalent to equality plus the variation caused by factors outside our control.

\section{Testing Approach}

\subsection{The Test Application}

To allow us to test Apollo, we needed to reliably and reproducably generate application workload that we could monitor, collect data for and run Apollo on the results.  We briefly considered using a real application such as an existing enterprise application or an open source business domain applications.  However it quickly became clear that using a real application would make the validation of the Apollo calculator almost impossible as it would be very difficult to make the application workload highly predictable and repeatable, to allow validation of the calculations performed.

We therefore decided to create a simple application, specifically designed to allow predictable application workloads to be generated and reproduced on demand.  The application contains four microservices implemented in Java:

\begin{itemize}
	\item A \emph{Gateway Service} that provides a simple entry point to the application for a benchmarking client to call and acts as an API gateway for the application.  The Gateway Service contains configuration to define repeatable application scenarios that can be invoked by name and this service then calls the other microservices as required for a particular scenario.
	\item A \emph{CPU Hog Service} that will consume a specified amount of CPU time, specified by a URL parameter to its service call.  The service can consume CPU time constantly for a timed period (e.g. 5 seconds) or it can consume a specified number of milliseconds of CPU time before returning.  It times itself for a timed period using the system clock (via the \texttt{java.lang.System.currentTimeMillis()} method) and measures its CPU time using the Java JMX monitoring facilities.
	\item A \emph{Memory Hog Service} that will consume a specified number of megabytes of (heap) memory, then wait for a specified number of seconds before attempting to release the memory.  The amount of memory and holding period are specified as URL parameters to its service call.  Given the design of the Java Virtual Machine (JVM) care must be taken when using a program to consume and release memory reliably, as the combination of JVM garbage collection and virtualisation of memory access mean that the operating system process often does not increase or decrease its memory usage predictably in response to Java code increasing or decreasing memory usage.  The service is useful though for applying increased memory pressure on the machine at specific times.
	\item A \emph{File IO Hog Service} that will perform a specified amount of file IO when its service entry point is called.  The amount of IO to perform is specified in megabytes as a URL parameter.  This service works in a very specific way, to avoid possible IO system optimisations reducing the real IO performed, while also avoiding unnecessary CPU consumption.  The problem we could have is that if we repeatedly write blocks of predictable data (e.g. blocks containing the same value throughout or repeating short data sequences repeatedly) then there is a danger that the IO subsystem will avoid performing some of the IO operations by performing some sort of write compression.  On the other hand, if we generate a large number of blocks of truely random data, this will involve quite a lot of CPU usage, which distorts the operation of the service and could make it CPU rather than IO intensive.  Therefore, the service generates a set of resuable random blocks of data when it starts up and selects a random sequence of these blocks, using a cheap pseudo-random number generator (\texttt{java.util.Random}) when writing a file.  This minimises CPU usage during the writing of the file, while also ensuring enough randomness in the file to avoid obvious IO system optimisations.
 \end{itemize}

 These services were developed using Java 1.8, using the widely-used Spring Boot 1.5.7 as an application framework (to provide application services such as HTTP request handling, configuration, database access abstraction and dependency injection).  This minimised the amount of application code that had to be written, allowing us to focus on the code needed to implement the core purpose of each service.  The microservices were organised into separate source trees, built using the Gradle build utility.

 Once developed, and unit and integration tested, the services were packaged into Docker containers to allow easy versioning, deployment and monitoring via the Docker runtime system, which was required by our proof of concept version of Apollo.

\subsection{The Test Software}

In order to reliably run a large number of validation tests for Apollo, the process needed to be automated so that the execution of the tests and collection of the results was standardised, efficient and reliable.  This involved a number of different types of automation and tools.

We started with a Linux server machine as the test environment.  The specification of the machine and the specific version of Linux are not very important, but we selected a 4 CPU machine with Intel Xeon 2.3GHz CPUs, 16GB of memory and 50GB of SSD storage running Ubuntu 16.4.04, one of the Ubuntu stable, long term support releases.  This is a reasonably large machine, but we selected this specification as it is representative of the sort of mid-range server class machine that forms the backbone of the server fleet of many large organisations today.  Microservice systems often utilise many smaller machines, but we quickly realised that while the Apollo system runs equally well on many smaller machines (due to the features of te open source tools it uses) consistency, repeatability and control were going to be reduced when using a number of hosts.  Therefore it was decided to do most of the testing using a single host machine that had sufficient capacity to run all of our services without resource contention.

The \emph{machine configuration} was automated using the open source Ansible system configuration tool, with a number of custom configuration files (or "playbooks" as they are known in the Ansible ecosystem) and some custom shell scrips defining the basic system configuration needed for the test environment, including updating the operating system and installing security patches, creating users and installing public keys, installing basic tools like Python and Git, installing Docker, and setting up some basic security mechanisms.

Once the machine configuration was complete, \emph{application configuration} and \emph{service configuration} were both automated using the Docker Compose tool, which is part of the Docker toolset.  Docker Compose allows a group of cooperating services to be defined in terms of the Docker container images they utilise, the specific configuration that each service applies to the base image and the dependencies between the services (such as one service needing to call another).  We defined a single Docker Compose configuration that included the application services and the data collection services we needed to provide the data for Apollo (Zipkin, MySQL, Telegraf and InfluxDB as discussed in \sref{section:resusageofappworkload}).  A service configuration is defined as a YAML text file.  This approach allowed a simple operating system script to be used to call Docker Compose with this configuration to reliably start or stop all of the services via a single command.

The process of \emph{running tests and collecting outputs} was automated using a range of operating system utilities and custom shell scripts.  The process can run one or more test scenarios, which were defined in the configuration of the Gateway Service as explained in the previous section.  The process of running a test scenario involved:
\begin{itemize}
	\item Clearing the MySQL and InfluxDB metrics databases so that the results of the test could easily be exported to files.
	\item Starting the \texttt{mpstat} and \texttt{pidstat} operating system utilities as "background" processes, logging their outputs to files, so that operating system and service process resource utilisation could be analysed after the test had executed.
	\item Invoking the Gateway Service to run the requested scenario (and in turn it would call the other services as the scenario definition defined).
	\item Stopping the operating system utilities cleanly to terminate resource utilisation statistics collection at a defined point.
	\item Exporting the execution traces, host and application resource utilisation metrics and Docker network metadata from the MySQL and InfluxDB databases and Docker, into files that could be used to re-populate databases at a later point in time for Apollo to use.
	\item Exporting the host level resource utilisation statistics from the \texttt{sar} utility into a text file to allow host level resource utilisation to be analysed after the test had executed.
	\item Gathering all of the outputs of the test into a \texttt{tar} archive file for easy storage and transfer.
\end{itemize}

When a particular test scenario needed competing workload on the host machine, the \texttt{openssl} package's \texttt{speed} command was used to generate load on one or more of the host's CPUs.  This was a scripted process, invoked manually before test cases that required predictable competing workload on the server host.

Once test cases were available, the output archive data files were transferred to local machines and a scripted process to \emph{run Apollo on test scenario output} was executed. This cleared local MySQL and InfluxDB database of their data, loaded the trace and resource utilisation data from the scenario into these databases, unpacked the other data files from the scenario archive, and ran Apollo on these data sets.  The output of the Apollo calculator was then inspected to analyse the characteristics of the scenario and the energy allocation that it had been allotted.


\section{Validating Calculation}
\label{sec:validatingcalculation}

Our first point of validation is to ensure that the allocation calculation performed by Apollo can be reproduced by hand, following the algorithm presented in the previous chapter, with an acceptable level of consistency between the two approaches.

We chose one of the simpler standard scenarios that we had used during the testing process, the \texttt{simple-cpu-x50} scenario, which calls the CPU intensive service 50 times.  We decided to use a scenario based on just two services, the Gateway service and the CPU intensive service, both running on a single machine.  This was to keep the manual calculation process tractable and avoid inconsistencies and mistakes complicating the process.  The Apollo calculation algorithm is simply a process of aggregating results from individual application elements and so we were confident that if the result of calculation for a small number of application services was correct then this would result in a correct calculation process for more complex cases that were simply aggregations of the lower level calculations.

To perform the calculation process, we decided to use a different set of metrics data sources from the set used by Apollo, to validate the approach to data collection as well as the implementation of the algorithm.  The set of data that the calculations was based on was:


\begin{itemize}
	\item the \emph{Zipkin traces} from the scenario being studied, as there was no credible alternative to this data;
	\item the \emph{Docker network configuration} for the Docker subsystem that the application elements were running within, to allow the IP address to container mapping to be retrieved, and again this was the only source of this information;
	\item a set of \texttt{pidstat(1)} metrics for the application, showing the CPU utilisation every second for the application processes;
	\item a set of \texttt{mpstat(1)} metrics for the host machine which allowed us to assess how much CPU was being used by the host during the duration of the test trace;
	\item the \emph{SPECPower benchmark results} for a respresentative server model, similar to the host in use; and
	\item the output of the \texttt{ps(1)} command from the host machine with all of the application elements running, to allow the process IDs for the application elements to be retrieved.
\end{itemize}

The important point about the data sources we used for this calculation is that the key data sources for resource utilisation metrics were sourced from different mechanisms than those used by Apollo, to allow a broader degree of validation than a simple recalculation using identical data would have allowed.  While Apollo gets its resource utilisation metrics from the Docker subsystem's statistics mechanism (gathered via the Telegraf metrics server), for our calculations we used the metrics from mpstat(1) and pidstat(1) as explained above.

The first step was to inspect the Zipkin trace data to allow the start and end points of the request to be identified.  The Zipkin trace data had been saved from the execution of the test scenario and was loaded into a MySQL database to allow it to be queried easily.

The root traces were extracted from the Zipkin database, by selecting all of those rows in \texttt{zipkin\_spans} where the \texttt{trace\_id} column was the same as the \texttt{span\_id} column:

\lstset{language=SQL}
\begin{lstlisting}
select * from zipkin_spans where id=trace_id
\end{lstlisting}


From the row returned, we extracted a trace ID of \texttt{5896569591426873811}, a start time of \texttt{1534185668229000} (nano seconds since the "Epoch", equating to 20180813T184108.229Z), a duration of \texttt{126292740} (nano seconds - 126.292 seconds)	which imply an end time of \texttt{1534185794521740} (being equivalent to 20180813T184314.522Z).

Using the trace ID we could then run a more complex query to extract the trace's spans and their attributes from the Zipkin database:

\lstset{language=SQL}
\begin{lstlisting}
SELECT hex(s.trace_id) as trace_id, hex(s.id) as span_id,
       hex(s.parent_id) as parent_id,
       start_ts as start_time_usec, 
       start_ts+duration as end_time_usec,
       inet_ntoa(endpoint_ipv4 \& conv("ffffffff", 16, 10)) as ipv4_address,
       endpoint_port
FROM zipkin_spans s, zipkin_annotations a
WHERE s.trace_id = 5896569591426873811
AND s.trace_id = a.trace_id
AND s.id = a.span_id
AND a_key = 'sr'
ORDER BY start_ts
\end{lstlisting}

This allowed us to identify the spans making up the processing of the request and the IP addresses of the two application elements that handled the invocations as being \texttt{172.18.0.7} and \texttt{172.18.0.8}, which were the IP addresses of the Docker containers containing the microservices involved in handling the request.

The next step was a simple lookup of the IP addresses in the Docker network configuration which resulted in the container IDs of the two Docker containers that our application elements ran within.  The short form IDs returned were \texttt{c42b3d3bf4b2} for the Gateway container and \texttt{b75d29577eb9} for the CPU intensive container.

Having the container IDs allowed us to reliably identify the Linux process IDs of the Java virtual machines running our application elements.  By using the \texttt{ps} output we saved during the test execution, we could use the container IDs to find the Docker \texttt{docker-containerd-shim}  processes running the JVMs for our services, which were process 3278 for the CPU intensive service and 3436 for the Gateway service.

Having the process IDs allowed us to inspect the \texttt{pidstat} output that we had collected during the test execution.   This utility captured CPU percentage usage statistics for our processes every second during the test period.

For process 3278 (the CPU intensive service) we found that its average CPU consumption was 25.42\% with a very low standard deviation of 0.26 across the test set.

For process 3436 (the Gateway) we found that its average CPU consumption was 0.3\% of the machine (although this did have a maximum of 0.75\% for short periods, leading to a standard deviation of 0.11 for the test set).

We therefore decided to use 25.75\% as the container CPU usage during the test period.

The next step in the process was to estimate the energy usage of the underlying host machine.  We collected \texttt{sar} and \texttt{mpstat} resource utilisation metrics for the machine during the test period to provide host CPU utilisation metrics.  The \texttt{sar} statistics were of less use as this utility collects statistics every minute and so we had a limited number of samples, however it appeared to indicate a total machine CPU usage of about 25\% during the test period.

The \texttt{mpstat} statistics were more useful as they contained a metric measurement for CPU usage percentage per CPU every second during the test period.  When we analysed this data set we found that the machine's utilisation was a very constant 27\% right through the test period, apart from one second (10 seconds into the test) when it jumped to 75\% usage and then back down to 27\% (for reasons we weren't able to identify).

The CPU usage percentage allowed us to estimate the energy consumption of the machine using a representative set of SPEC Power benchmark results for a similar server model.  The benchmark results were 102W at 20\% and 120W at 30\%.  Therefore interpolation resulted in the value $((120 - 102) \times (7/10)) + 102 = 12.6 + 102 = 114.6$.  That is, the machine's power consumption during the test period was about 114 J/second.

Given that the test period was 126.30 seconds, this resulted in a power consumption value for the host machine of $126.30 \times 114 = 14473.98$J, which we rounded to 14474J.

Given the host CPU usage, the CPU usage of our containers and the host power consumption we could now calculate the power allocation to our application trace.

Given a test length of 126.3 seconds and a 4 CPU host machine this means a total possible CPU resource available of:

\begin{equation}
126.3 \times 4 = 505.2 ~\text{CPU seconds}
\end{equation}

Our test data indicates that the host was busy for 27\% of the time during the test execution, therefore our total host CPU time consumed during the test period was:

\begin{equation}
505.2 \times 0.27 = 136.40 ~\text{CPU seconds}
\end{equation}

Our application processes (containers) consumed 25.75\% of the host's CPU time during the test period therefore their CPU usage is calculated as:

\begin{equation}
136.40 \times 0.2575 = 130.09 ~\text{CPU seconds}
\end{equation}

Given our CPU usage and the host energy consumption we can then calculate the energy allocation to our application request as being:

\begin{equation}
130.09 / 136.40 \times 14474 = 13804\text{J}
\end{equation}

We then ran the Apollo proof of concept implementation on the test scenario data set to compare our results, which are shown in \tref{table:calculationresults}

\begin{table}
\centering
\caption{Manual Calculation Compared to Apollo Calculation}
\label{table:calculationresults}
\footnotesize
\begin{tabular}{|c|c|c|c|}
\hline
Value & Manual Calculation & Apollo Calculation & Difference \\
\hline
\hline
Trace CPU msec       & 130090 & 124257 & 4.5\% \\
Host CPU msec        & 136404 & 132891 & 2.5\% \\
Host Energy J        & 14474  & 14213  & 1.8\% \\
Application Energy J & 13804  & 13290  & 3.9\% \\
\hline
\end{tabular}
\end{table}

As can be seen from the results in the table, using a manual calculation technique that attempts to mirror the algorithm used in Apollo, while using different data sources for the critical resource utilisation metrics, has resulted in a very close match between the results, with the manual calculation of the energy allocation for the application request trace being within 4\% of the automated Apollo value.

The difference in the results will be familiar to anyone who has been involved in performance testing, where repeatability of results is often very difficult to obtain (as evidenced by the subtle but constant inconsistencies in the numerical results from tools such as \texttt{sar}, \texttt{mpstat} and \texttt{pidstat}).  In our situation, as well as the normal difficulty of repeatability, we believe that most of the difference in these results is likely to be the result of Apollo using the finer grained Docker metrics, while the manual process relied on the coarser grained metrics from the standard system administration tools. 

This result is well within our target tolerance for energy allocation results and so validates the Apollo proof of concept energy allocation implementation's accuracy.

\section{Validating Consistency}
\label{section:validatingconsistency}

To validate consistency of energy allocation, our strategy is to run a known control workload under fixed host utilisation conditions (no other workload being the simplest case) and to run a range of other workloads that we know contain an equivalent amount of computational work but are structured differently.  The energy allocation should be the same for each case.

In our first test, we structured a workload into three cases, each of which involved the same amount of CPU workload but in three different scenarios.  The first scenario invoked a short service call (involving 50msec of CPU work) 1000 times, the second scenario invoked a longer service call (500msec of CPU work) 100 times and the third ran a long service call (5000msec of CPU work) 10 times.  Each scenario was called 6 times to ensure a consistent result.

Our first question was whether the energy allocations per request were consistent across the three cases.  Our analysis of this question is shown in the scatter graph in \fref{figure:validation-energycpu}, which plots the energy usage against CPU workload for each of the scenarios executed, using logarithmic scales.

\begin{figure}
\centering
\includegraphics[width=0.75\textwidth]{Figures/validation-energycpu}
\caption{Energy Allocation per Request for Small, Medium and Large Services}
\label{figure:validation-energycpu}
\end{figure}

The graph shows the three sets of scenarios (short, medium and long) clearly clustering very closely, showing that the energy allocation per trace is extremely consistent across the three sets, suggesting that the allocation calculation is working consistently as designed.  More precisely, the correlation coeficient between the two sets of values is 0.999, indicating a very high degree of correlation between CPU consumed and energy allocation performed.

Our second test involved investigating how energy allocation was affected by a constant amount of CPU workload but in scenarios of different lengths.  To test this we ran repeatedly ran two scenarios, both of which contained the same amount of CPU workload (a trace that took 2,500 msec, run 50 times), with one scenario having pauses inserted into it, to cause the scenario to take longer to execute but consume no more CPU resource during the extended execution time.  In addition, for reasons which we explain below, we ran the first set of scenario tests with no additional load on the machine and the second set with synthetic workload consuming 50\% of the machine's CPU.  The results of this experiment are shown in the line graph in \fref{figure:validation-scenariolength}

\begin{figure}
\centering
\includegraphics[width=0.75\textwidth]{Figures/validation-scenariolength}
\caption{Energy Allocation Across Different Scenario Lengths}
\label{figure:validation-scenariolength}
\end{figure}

This graph plots the CPU usage per trace (the top line) and the energy usage per trace (the bottom line) for sample executions of the scenarios.  As indicated by the dashed boxes annotating the graph, the first 6 executions were performed with no additional workload on the machine, the second 6 executions were performed with the machine having 50\% of its CPU capacity consumed by other synthetic workload.

As can be seen from the graph, the estimate of CPU usage by trace is consistent, within a maximum of 2\% variation from the mean (min 2467, max 2556, mean 2502, standard deviation of 23.09, which is less than 1\% of the mean).  This is well within our target consistency.

When we analysed the power allocation by trace we saw an interesting development which was the higher power allocation for the longer scenarios when no additional load was on the machine, an in constrast, a more constant power allocation when 50\% additional load was running on the machine.  While unintuitive initially, when we investigated the data, as explained below, we found that this is exactly as expected and is an important energy usage insight for the software architect investigating the power characteristics of their software in production.  

When no additional load is executing on the host, there is no other workload apart from our traces to allocate power consumption to.  In which case if the scenario takes longer, you would expect a higher energy allocation even with constant resource usage, as the host machine is consuming energy, even when not actively running our workload and if there is no other workload to allocate this "background" energy consumption to, then it will be allocated to our workload.  The graph shows that this is exactly what happens; when no other workload is executing on the machine, our longer scenarios (the "single-cpu-x50-pause" scenarios) are allocated more power than the shorter scenarios (the "single-cpu-x50" ones) even though they all consume very similar amounts of CPU time.

In contrast, when there is additional workload on the machine, the energy allocated to each trace is much more even, within a maximum of 4\% variation from the mean (min 157, max 164, mean 160, standard deviation of 2.66 which is 1.6\% of the mean).  This is also an expected result as during the execution of our test workload, there is other workload running on the machine which shares the allocation of the host's energy consumption.  As our workload runs longer, but is not using CPU during part of the period, it is allocated correspondingly less of the host's energy as there is other active workload on the machine which is allocated more of it.  This is the allocation we would expect and it is within our target consistency and again suggests that this is a consistent energy allocation mechanism.

\section{Validating Allocation}
\label{section:validatingallocation}

Another aspect of energy allocation is how a fixed workload is allocated energy when the underlying host has varying amounts of other workload running on it concurrently.  We tested this aspect of allocation by running a single workload type under 5 different host utilisation conditions, namely when there was no other workload on the host, and when the host was 25\%, 50\%, 75\% and 100\% utilised before our workload started.  The results of this test are shown in the line graph in \fref{figure:validation-machineload}.

\begin{figure}
\centering
\includegraphics[width=0.75\textwidth]{Figures/validation-machineload}
\caption{Energy Allocation Under Different Host Load Conditions}
\label{figure:validation-machineload}
\end{figure}

As with some of our other testing, the results initially look somewhat counter intuitive, but on further analysis are validation of consistent energy allocation by workload.

%  (Trace J at 25/50/75/100 prior load: 276, 183 [93], 164 [19], 169, 167)

The first point in the graph shows our workload running on an otherwise idle host and it is allocated quite a large amount of energy per trace (of 276J) because the host is relatively inefficient at lower levels of utilisation and there is no other workload on the host to allocate its energy to.

The second point in the graph shows a sharp reduction in energy per trace (to 183J), which is caused by the host's utilisation now being about 50\% which is considerably more efficient than 25\% and the fact that the host's energy is being split between two roughly equivalent workloads.

The third point on the graph shows a further reduction in energy per trace (to 164J), which is a considerably smaller reduction than the previous step.  This is due to our share of the machine workload falling less significantly than in the previous step (from 49\% to 33\% whereas the previous step was from 96\% to 49\%).

The fourth point on the graph, at 75\% of other utilisation, actually goes up slightly (to 169J).  This is due to two factors.  Firstly, once again, our utilisation percentage drop decreases, this time from 33\% to 25\% (only 8\%) but secondly, the underlying machine becomes less efficient as utilisation moves beyond 75\% and so there is more energy to allocate between the different workload items.  This is an important insight for the application architect so that they consider the potentially non-linear energy consumption curve of the underlying host.

Finally at the fifth point on the graph, with 100\% other utilisation, our workload is competing with existing workload to be scheduled for execution.  This results in our execution duration extending slightly (by 5\%), and our CPU utilisation percentage to drop slightly (to 23\%) with the result being a slight reduction in energy utilisation (to 167J).  This is the result of a relatively small increase in the host's energy utilisation (as it was already running close to 100\% utilisation at the previous sample) and there is now further workload to allocate the energy of the host across, so reducing our workload's allocation slightly.

This phase of testing was an interesting process because it illustrated the usefulness of investigating energy allocation using practical testing and a quantitative data-based allocation mechanism like Apollo.  It would be quite possible to make a simplistic assumption that fair energy allocation would keep falling linearly as load increased, but our tool can be used to provide a more sophisticated analysis that reveals how a complex interaction of a number of factors (including load, scenario length and host energy characteristics) can result in a correct allocation that is more complex.  This is a useful insight for the application architect as they investigate the energy characteristics of their application.

\section{Validating CPU as a Resource Usage Proxy}

When we explained how the energy allocation process for an application's elements was to work, in \cref{chapter:monitoring}, part of the design of the allocation approach was to make the simplifying assumption that CPU utilisation is a good proxy for IO activity (\sref{sec:utilisingresourceusage}).  This allowed  the approach to rely on CPU usage to allocate energy fairly.  While this assumption is based on previous research work \cite{bashroush2018_hardwarerefresh} , we were interested to test this assumption for ourselves by comparing IO activity with CPU utilisation.

To test the assumption that CPU utilisation acts as a good proxy for IO activity, we wanted to find whether the two values correlated well during an application workload.  To test this we ran IO intensive workloads of varying known sizes and measured the CPU utilisation of each one.  The results of this exercise are shown in the line graph in \fref{figure:validation-traceenergydatasize}.  The x-axis of this graph shows the test scenarios, the left-hand y-axis is the amount of CPU measured for each scenario, the right-hand y-axis is the amount of energy allocated to each scenario.

\begin{figure}
\centering
\includegraphics[width=0.75\textwidth]{Figures/validation-traceenergydatasize}
\caption{CPU Utilisation and Energy Allocation Scenarios}
\label{figure:validation-traceenergydatasize}
\end{figure}

The graph shows the results of running a number of data-intensive application scenarios, one group that called a service to write 100MB of data to file 4 times, one that called a service to write 500MB of data to file 4 times, and one that called a service to write 1GB of data to file 4 times.  Hence the total data written by the first group of scenarios was 400MB, the second group 2GB and the third 4GB.

When we plot the CPU usage and the Apollo energy allocation for the other test scenarios on the graph, we can clearly see that CPU usage is directly related to the amount of data written and the correlation coefficient between the values for data written and CPU utilised is 0.9965, indicating that CPU is a good proxy for the amount of data written by a process.

For completeness we also plotted the data-size of each scenario compared to the energy allocated by Apollo to each, which is shown in \fref{figure:validation-energybydatasize}.  

\begin{figure}
\centering
\includegraphics[width=0.75\textwidth]{Figures/validation-energybydatasize}
\caption{Energy Allocation by Data Size}
\label{figure:validation-energybydatasize}
\end{figure}

This graph looks similar to \fref{figure:validation-traceenergydatasize} but is illustrating a slightly different point in that while the x-axis (test scenarios) and right-hand y-axis (energy per trace) are the same, the left-hand y-axis shows the amount of data written by each scenario, rather than the CPU usage.  This can be seen in the flat horizontal sections of the graph for each type of scenario.  

The strong grouping of points around scenarios shows that this graph validates that the energy allocation for these data-intensive services correlates well with the amount of data each is writing, albeit with some variation in some test cases. When we calculated the correlation coefficient between energy allocation and data written, it was found to be 0.9967, again showing a very high degree of correlation between energy allocation and the amount of data written by the application elements within the scenario.

Finally, to provide some additional validation of the relationship of IO workload to CPU usage independent of Apollo, we performed some detailed manual tests, calling the microservice directly and measuring CPU usage before and after service invocations for different amounts of IO.  The CPU usage was read directly from the \texttt{/proc/PID/stat} operating system statistics.

\begin{figure}
\centering
\includegraphics[width=0.75\textwidth]{Figures/validation-cpubydatasize}
\caption{Energy Allocation by Data Size}
\label{figure:validation-cpubydatasize}
\end{figure}

The result of these manual tests is shown in the scatter plot and trend line shown in \fref{figure:validation-cpubydatasize}.  The y-axis shows the amount of CPU used by the service per request, the x-axis the amount of data written by each request and each point on the graph is a single service invocation.  As the graph shows there is a strong correlation between the amount of data written and the CPU consumed for the service call and when we calculated the correlation coefficient, it was found to be 0.9998, confirming this finding.  By using the difference in data size and CPU utilisation between samples, we were also able to calculate that 1 MB of file IO write activity appears to require approximately 2ms of CPU time on our test machine.

The combination of these test results shows that we were correct in our assumption that CPU is a good proxy for other resource utilisation by a process.

\section{Summary}

In order to validate our proof-of-concept implementation of the Apollo Energy Allocator system, we identified four aspects of validation that we needed to perform.

\begin{itemize}
	\item \textit{Calculation Correctness} which involves running the calculator in a realistic, but reasonably simple, scenario, under controlled conditions and replicating its calculation process as independently as possible.
	\item \textit{Consistency} which involves running a number of test scenarios with different characteristics and comparing the energy allocation results provided by Apollo, to ensure consistency between different types of scenario.
	\item \textit{Allocation} which requires us to run identical scenarios in situations with different amounts of controlled competing workload on the host machine(s) to allow us to validate that Apollo allocates energy fairly across these workload profiles.
	\item \textit{CPU and Data Validation} which confirms that the previous research result that suggested that CPU usage was a valid proxy measure for overall resource utilisation, and in particular for disk IO activity.
\end{itemize}

As stated earlier, our aim was to achieve consistency of results to within 5\% tolerance as practical experience has taught us that results within this tolerance level are effectively equal in performance and resource usage testing, given the number of dynamic factors outside our control.

The testing presented in this chapter has specifically addressed each of these aspects of validation.

We performed a manual data collection and calculation exercise in order to validate \textit{Calculation Correctness}, as described in \sref{sec:validatingcalculation}.  The result of this exercise, based on independent resource utilisation data sources, separate from the data used by Apollo, was an energy allocation value within 4\%  of the value calculated by Apollo.

We then investigated \textit{Consistency} by performing a series of CPU intensive workload tests, as described in \sref{section:validatingconsistency}, that ran controlled application workloads for different levels of resource consumption, in order to confirm that Apollo allocated energy correctly and consistently for all of the test cases.  These test cases proved that there was a very high degree of correlation between our different levels of application workload and the energy allocation that each received, so proving that the energy allocation was consistent across varying workload.

The next step was to validate \textit{Allocation}, as described in \sref{section:validatingallocation}, by running controlled test workloads on host machines that had carefully controlled competing workload running on them already. These tests proved that when competing workload was present on a host machine, energy was allocated correctly between our test workoad and the competing workload and was well within the 5\% level of consistency that we were aiming for.

Finally, we investigated whether \textit{CPU usage is a good proxy for resource usage} for an application trace.  We were aware of previous research results suggesting that this was the case, but we decided to investigate it empirically in our specific situation too.  We focused on file IO during this part of the investigation, given its very large contribution to energy consumption [REF], and we ran a number of tests to investigate its relationship to CPU usage.  These tests confirmed that CPU usage is a very good proxy for file IO usage too, with a correlation coefficient greater than 0.9 for the two values.

In summary we have investigated four different aspects of the correctness and utility of the Apollo Energy Allocation System's proof-of-concept implementation of our energy allocation approach for application-level energy monitoring.  All four of the testing exercises confirmed a different aspect of the correctness of the implementation.  Therefore, we conclude that the proof-of-concept implementation is sufficently consistent and correct to validate the application energy allocation approach that we propose.

 %Ch8 - Validation of Application Energy Monitoring

\chapter{Conclusions and Future Directions}

\section{Summary and Conclusions}

The research presented in this thesis has been a journey from abstract design languages to practical runtime tools with the goal of providing software architecture practitioners with better support for considering energy as an architectural concern than exists today.  On the way it has involved tools, design guidance and the working practices of effective architects.

This journey has resulted in a number of research contributions to the fields of software architecture research and energy efficiency research.

Firstly, we have performed a comprehensive systematic survey of 25 years of research in the field of architectural description languages, resulting in a thorough characterisation of the field.  This then led to a published case study \cite{woods2012-adlcasestudy, woods2015-adlcasestudy} that reported the experience of creating a large-scale industrial architectual description and the shortcomings of existing architectural description languages in such environments.

The question of how architects can prioritise energy efficiency work led to an interview based investigation of how expert architects prioritise their effort and the development of a model that distills the common advice into an accessible form that can be used to guide less experienced practitioners \cite{woods2017-archeffort}.  This was then validated and refined through a large scale survey of software architecture practitioners from across the world.

We considered what tangible advice was available to software architects who want to improve the energy characteristics of their systems and found little in the research literature that most architects could directly apply.  This led us to identify a small number of architectural design principles \cite{bashroush2017-archprinciples} based on a successful industrial case study of a large organisation that improved the energy characteristics of some of their application services, through architectural changes.

Finally, we identified the need for a practical tool that architects could use to measure the energy characteristics of their applications when running different scenarios and designed an approach to achieve this.  We implemented a working proof-of-concept version of the tool and validated it with practical test cases.

This research was undertaken in the context of the four research questions that we introduced in \cref{chapter:introduction} and having now completed the work we can provide answers to them.

\textbf{RQ1} \emph{What architecture description languages exist and can they be used to reason about the energy properties of a system?}

To answer this research question, we performed a thorough review of the research literature over the last 25 years and considered whether the ADLs we found could be used in an industrial context.  We performed a significant case study project and created a large architectural description of an industrial system, which was then used for a variety of purposes.  However the conclusion we reached during this work was that existing architectural description languages are not suitable for mainstream adoption due to their narrow focus on functional structure, the lack of industrial validation, the high adoption cost of most of the languages, and the lack of mature tool support available.  This led us to define a lightweight graphical notation, supported by graphical and textual templates for documentation creation, which was ultimately successful in the case study project.

Many of the ADLs we surveyed are extensible, although relatively few of them (4, 7\% of those surveyed) provide direct support for capturing system qualities in the language, however we judged that half of the ADLs we analysed could capture system qualities via some mechanism provided by the language.  Therefore, in principle it should be possible to use these languages as the basis of a system to allow reasoning about a system's energy qualities.  However the practical adoption problems we encountered mean that we do not believe that they can be used to support reasoning about energy properties in practice.

We make a number of constructive suggestions for how to make architecture description languages a more practical proposition for practitioners in
 \sref{section:adlvalidation}.

\textbf{RQ2} \emph{How can architects prioritise energy efficiency as an architectural concern?}

When considering how our work might be used by practitioners we realised that the first challenge was how to persuade architects to priortise the energy characteristics of their systems.  Architects have a very wide range of concerns to address and often complain that it is difficult to know where to focus their attention.  Anecdotally, this seems to be particularly acute for less experienced practitioners.

We noted that many experienced practitioners manage to focus their effort very effectively and seem to be able to deal with a wide range of concerns during the lifetime of a project.  When we asked people informally, we did not find anyone using a formal approach for this, it just seemed to be something they knew how to do.

An initial literature review did not find any generally applicable approaches that provided sufficient guidance to make a focus on energy properties likely, so we investigated how experienced software architecture practitioners focus their attention.  We found that some strong themes emerged, which we used to create a model to guide less experienced practitioners.  This model was validated via an online survey questionnaire, completed by over 80 practitioners from all over the world.  The results of the survey validated the model strongly and also provided input to allow it to be refined into a more effective model.

We found that there are four aspects to the approach that experienced practitioners use to focus their attention, namely:
\begin{enumerate}
	\item Stakeholder needs and priorities
	\item Priorising time according to risks
	\item Delegating as much work as possible
	\item Ensuring team effectiveness
\end{enumerate}

This work provided an answer our research question.  From these four areas, the aspect of prioritisation that will cause architects to focus on the energy qualities of their systems is to ensure that the energy efficiency of the system is high in the list of stakeholder needs and priorities.

\textbf{RQ3} \emph{What design guidelines can we provide to assist architects to improve the energy efficiency of their systems?}

When we considered what software architecture practitioners needed to allow them to confidently address the energy properties of their systems, we quickly identified the importance of accessible and reliable technical guidance.  The two forms of guidance that architecture practitioners are already familiar with are architecture principles and architectural tactics, so we investigated the principles and tactics available to them.

The initial literature review revealed that while this field is relatively immature, there was material in the research literature that could be of use to architecture practitioners, notably an architectural perspective for energy efficiency.  However we found a lack of generally applicable tactics and principles.  There were several sets of architectural tactics in the literature \cite{lewis2015-foragingtactics,procaccianti2013-cloudenergyefficiency} but one is aimed more at those building cloud platforms than applications and the other is specifically aimed at architects building applications utilising cyberforaging to offload work from mobile devices.

In response we decided to try to identify some architectural principles that could be generally applied by software architecture practitioners who were trying to improve the energy efficiency of their applications.  We did this by identifying a published industrial case study of a large organisation who improved the energy efficiency of a number of their application services through software architecture changes, and extracting and generalising the principles that had guided their work.

This resulted in a set of three initial principles which had proved of value in the organisation that performed the case study.  The principles we identified were:
\begin{enumerate}
	\item Energy efficiency metrics must relate business transactions to energy consumption in a way that is meaningful to key system stakeholders.
	\item Identifying sources of energy waste at the system level produces the biggest savings.
	\item Addressing the energy optimisation problem requires a cross-disciplinary team.
\end{enumerate}

This work allowed us to answer the research question with this initial set of energy related architecture principles which we believe can be extended further in the future through the study of other successful industrial energy efficiency improvement projects.

\textbf{RQ4} \emph{How can we make architects aware of the runtime energy characteristics of their systems?}

Having considered how to enable software architects to focus attention on the energy properties of their system and identified some initial principles that could guide the development of more energy efficient systems, it became clear that architects also need to be able to measure the energy properties of their systems.

A literature review revealed that there have been a number of attempts to create software systems that can measure the energy characteristics of software applications.  However as we reported in \sref{sec:litreviewenergy} there were a number of limitations with most of the work that had been reported in these publications.  

Firstly a number of the projects used linear regression models to establish the relationship between resource consumption and energy consumption, but had not validated how robust or reusable these models would be without constant re-training.  Training these models in an industrial setting is complicated to achieve and models that required retraining for different workloads or after every change to the environment would not be a practical propostion.  

The other concern with the existing research is the focus on measuring the energy consumption of operating system processes (or individual pieces of code) rather than execution scenarios.  This means that the architect needs to set up very specific benchmark scenarios under controlled conditions in order to gain any insight from the results, which again is difficult to do in a real project.  Instead, we wanted a scenario based approach that measured the energy consumption of a single execution scenario, as this would allow the approach to be used with synthetic workload in existing test or production environments.  

Finally, we also found that most of the research projects have not made their prototype systems available for inspection or use, meaning that many of the details of the work are unclear and there is no scope for reuse by other researchers.\footnote{The E-Surgeon researchers \cite{noureddine2015-hotspots} are the notable exception as they have helpfully open sourced all of their tools, which we did inspect for deeper insight into their work.}

In order to progress this area of research, we designed a model for estimating the energy characteristics of individual architectural scenarios (inbound requests) to a microservice based system.  We dubbed our approach "Apollo" and implemented a proof-of-concept version of it and then validated this with practical testing.  The result is a reliable and practical tool for calculating the resource utilisation of a specific inbound request to a microservice system and using this to allocate the energy consumption of a server machine to the workload running on it. This encourages the architect to minimise the resource utilisation of their software and also to consider the most efficient deployment options for it.

This work provides us with an answer to our research question, which is that we can provide architects with tools that calculate a context specific energy consumption estimate for their software application executing different scenarios.  This will allow architecture practitioners to use the tool with synthetic workload in suitable configured production and test environments, alongside other workload.  This will allow them to monitor the application energy consumption over time and understand the energy implications of their architectural design decisions, so allowing energy to be treated as a first class architectural concern.

\section{Future Directions}

Much of the work reported in this thesis has promising future directions to further increase the scope, applicability or sophistication of the research results reported here.

\subsection{Architectural Prioritisation}

The refined model is now ready for dissemination to the practitioner community to see if it proves as useful in practice as our survey of the preliminary model suggests.  To reach the practitioner community, we will publish the model in a less formal style via posts on mainstream Internet sites (such as medium.com, LinkedIn and Twitter).  We will also try to publish a summary of it in practitioner-oriented publications and publicise it through conference sessions at practitioner conferences, if it proves to be of interest to programme selection committees.

After practitioner oriented publication, we are also interested in extending the Stage 3 questionnaire to architects in other geographical locations to compare and explore whether they react in the same way to the model as their colleagues in Europe and the Americas

Beyond this, it would be interesting to survey practitioners who have used the model in the future, after they have been using it for some time.  This would allow us to understand whether its usefulness was borne out in practice and to find out what the practitioners are actually using it for (for example, whether it is used more as a training aid or as a personal aide memoir) and which industries and architectural job types are using it.

\subsection{Architectural Design Guidance for Energy Efficiency}

By analysing a successful industrial energy reduction project we have identified a small set of useful architectural design principles to guide architects in their consideration of energy as an architectural concern.  

There is great potential in this area to identify other industrial work that is attempting to reduce the energy consumption of real software systems and from the successes and failures of those project identify the principles and tactics that allow architects to actively manage the energy consumption of their applications as an architectural concern.  

We also believe that further industrial and academic cooperation (of the sort we observed in a case study from the Netherlands \cite{jagroep2016-comparingreleases}) could lead to the identification and validation of more principles and tactics for energy aware architecture, as could focused academic work to propose likely principles and tactics and to validate them in both laboratory and industrial settings.

Once we have a larger proven set of principles and tactics then they would form a valuable addition to the energy architectural perspective created by Utrecht University and Centric Netherlands BV \cite{jagroep2017-energyperspective}, which would make them available to architecture practitioners in an accessible form.

\subsection{Runtime Application Energy Monitoring}

The Apollo model and proof-of-concept implementation presented in this thesis is a research prototype that is still relatively immature, as its validation has been limited to controlled testing.  There is significant scope to continue research in this area with the aim of creating a practical tool which can be applied in a general industrial setting.

There are a number of interesting avenues to explore in the area of data aquisition, including experimentation with hardware event counters and event based direct data collection from operating system resource counters (rather than the sampling based approach used in today's implementation).  Extending the cost based energy model beyond CPU usage to include network, disk and memory resource usage measurements would also be an interesting area to explore to see whether the increased accuracy is valuable enough to justify the additional complexity.

In a related area, the current implementation requires dedicated microservices for the monitored workload, to simplify the collection of statistics.  This is a reasonable simplification because modern microservice infrastructure makes it straightforward to add additional container instances dedicated to specific workload.  However an interesting future research direction would be to utilise thread-level resource consumption statistics rather than process level ones and investigate whether this would allow us to relax this constraint.

The current model and implementation do not take the energy overhead of the data centre environment into account in the energy consumption estimates.  This would be relatively straightforward aspect to add to the model provided that a reliable source of PUE data for the environment(s) that the software is running in were available.  PUE varies over time, but many DCIM products provide estimates of the PUE of an environment and so could be used as a source for this data.  This would allow an allowance for the data centre's infrastructure to be added to the energy estimates, so highlighting the implications of deployment options to the architect.

The current system is batch based and as we described in \cref{chapter:validation} we tested it by running tests and collecting data sets from them, which were then processed by Apollo.  However there is nothing in the model or in fact the current implementation that would prevent Apollo being used to analyse data in "mini batches" as soon as it is available.  A potentially fruitful avenue of research would be to create an event-driven data collection system that generated a data set for Apollo whenever a scenario (i.e. an inbound request) completes, which can be observed from the Zipkin database.  The data set could then immediately be processed by Apollo (which takes a couple of seconds), providing a near realtime view of the application's energy characteristics.

Another possible avenue for research is the outputs of the tool.  The current software reports the energy and resource utilisation measurement values as text messages written to logs or the console.  While perfectly functional, this means that the user has to take process this data themselves to perform and analysis on it (using text processing tools and spreadsheets, as we did during the validation process reported in \cref{chapter:validation}).  An interesting research topic would be to apply modern analytical and visualisation techniques and tools to the output of Apollo in order to provide the architect with insight into the results and perhaps automated guidance on how to improve the energy characteristics of the application.

Finally, the software needs to be made available as open source software via Github to allow other research groups and interested practitioners to access it.  It has been developed "in the open" on Github but does not have the supporting materials to allow others to understand and use it at present.

\section{Concluding Remarks}

This research was motivated by the urgent need to reduce the ever increasing amount of energy required to support the world's burgenoning digital transformation.  This is needed for both environmental and cost reasons, to allow a sustainable transition to the next phase of the information age, particularly as developing countries become digital economies.

The journey has taken us from architectural description languages to the prioritisation of architecture work, through design guidance for energy efficient applications, to the creation of a novel model and tool to provide architects with insight into the runtime energy characteristics of their systems.

During the journey we have understood the state of the art in architectural description languages and tried to apply them, investigated how expert architects balance concerns to focus their attention to be most effective, identified architectural design principles for energy efficient systems and designed, built and validated a novel and practical tool for architects to use to measure the energy efficiency of their applications.  The overall conclusion from the work is that it is now entirely possible to start treating energy efficiency as a first class architectural concern in software architecture, although a significant amount of work is needed to mature the field to the point where it can become part of mainstream practice.

As we progressed through the work we have answered our research questions, some positively and some negatively, but beyond those relatively narrow topics we have been exposed to the huge amount of intellectural effort being expended across a fascinating range of topics related to the architectural design, analysis and energy efficiency of complex systems.  Sadly much of this thinking, while creative and innovative, fails to have significant impact due to a lack of validation, industrial alignment and accessible communication to practitioners.

Surely now, with the environmental imperative of controlling the energy usage of our digital economy, we can summon the motivation to realign our research and industrial communities in a united effort to address this problem?  The world needs us to.


 % Ch9

%% ----------------------------------------------------------------
% Now begin the Appendices, including them as separate files

%\addtocontents{toc}{\vspace{2em}} % Add a gap in the Contents, for aesthetics

\appendix % Cue to tell LaTeX that the following 'chapters' are Appendices

\chapter{Architectural Description Languages} \label{appendix:adl-list}
In this appendix, we list the characteristics of all of the architectural description languages that met our inclusion criteria for the literature review described in \sref{sec:adl-lit-review}.

Due to the amount of information needed to characterise the ADLs, it is presented in three tables, \tref{table:adl-basics} that contains the basic characteristics of the languages, \tref{table:adl-concepts} that describes the architectural concepts that appear in each language, and \tref{table:adl-mechanisms}, which lists the architectural description mechanisms provided by both.

%
% This table formatting is really black magic.  The careful structure of
% commands and data in the heading section comes from Stack Overflow, the
% longtable manual and experimentation
%
% The formula below (caption + label then hline then headers, then caption, then 
%    \endfirsthead, then the same structure with the caption modified not to
%    include each page in the contents - the "[]" - finshed with \endhead) appears
% to work!
%
\begin{landscape}
\singlespacing
\footnotesize
\begin{longtable}{|c|p{6cm}|p{3cm}|c|p{3cm}|p{2cm}|c|} 
\caption{General Characteristics of the ADLs} \label{table:adl-basics} \\
\hline
\textbf{ADL} & \textbf{Description} & \textbf{Institutions} & \textbf{Year} & \textbf{Domain} & \textbf{Application} & \textbf{Reference} \endfirsthead
\caption[]{General Characteristics of the ADLs} \\
\hline
\textbf{ADL} & \textbf{Description} & \textbf{Institutions} & \textbf{Year} & \textbf{Domain} & \textbf{Application} & \textbf{Reference} \endhead
\hline
AADL & An architectural description language with industrial roots, having come from work in the avionics industry based on key concepts from MetaH and ACME.  It is a rich ADL targeted at embedded systems, supporting a range of architectural views and having good tool support available.  It has been standardised by the Society of Automotive Engineers (SAE). & Over 20 industrial and academic organisations & 2006 &  Embedded Systems & Industrial Projects & \cite{feiler2006-aadl} \\
\hline
ABC/ADL & Developed with the aim of improving the link to implementation from the architectural description and as part of this, providing good support for composition.  It focuses on the functional structure of a system and prototype tools have been developed for it. & Peking University, Beijing, China & 2002 & General & Experiments & \cite{mei2002-abcadl} \\
\hline
AC2-ADL & It is an aspect-oriented ADL that allows aspects on both components and connectors to be integrated into the architectural description. & Wuhan University Wuhan, China; University of California Irvine, USA & 2008 & General (AO) & Examples & \cite{jing2008-ac2adl} \\
\hline
ACDL & An ADL that represents the centralized-mode architectural connection in which all components are linked by a single connector. The connectors described in ACDL are structurally flexible in the sense that protocols implemented in them have no restriction on the numbers of attached same-type components. & University of Technology, Sydney; Tsinghua University, Beijing, China & 2010 & General & Examples & \cite{su2010-acdl} \\

\hline
ACME & Developed primarily as a mechanism for the interchange of architectural information and as such is extremely flexible, intended as a base for more specific ADLs rather than being used directly. & Carnegie-Mellon University, USA & 1997 & General & Case Studies & \cite{garlan1997-acme} \\

\hline
ADLARS & Developed with the aim of creating an ADL with first class support for embedded systems product lines.  It supports multiple views of the system and variation points in the architecture. & Queens University Belfast, UK & 2005 & Embedded Systems, Software Product Lines & Experiments & \cite{bashroush2005-adlars} \\

\hline
ADML & The primary focus of ADML is dynamic behaviour rather than structure and it is based on a dynamic description logic called DDL(SHON(D)). & Four research groups in Beijing, China & 2012 & Concurrent and Distributed Systems & Examples & \cite{wang2012-adml} \\
\hline

Aesop & Aesop was developed to allow the description of architectural styles, rather than just architectures.  Aesop is a tool and the language that it implements, which allows style-specific architectural tools to be created. & Carnegie-Mellon University, USA & 1994 & General (Styles) & Experiments & \cite{garlan1994-aesop} \\
\hline

ALI & Developed with the aim of producing an industrially relevant language, which would be usable for product lines as well as individual systems.  It supports first class components, connectors and configuration but also includes explicit features for variation and reuse. & Queens University Belfast, UK & 2008 & General & Examples & \cite{bashroush2008-ali} \\
\hline

AO-ADL & An aspect-oriented ADL is a descendent of DAOP-ADL.  It allows aspects to be used as first class architectural constructs, in this case to assist in isolating parts of the system that address cross-cutting concerns (such as security).  & University of Malaga, Spain & 2011 & General & Research Projects & \cite{pinto2003-daopadl} \\
\hline
Archface & A component and connector based language.  It aims to bridge the gap between architectural description and code and does this by compiling the ADL into partially complete AspectJ code for the developer to complete. & Kyushu Institute of Technology and University of Tokyo, Japan & 2010 & General & Examples & \cite{ubayashi2010-archface} \\

\hline
Aspectual-ACME & An extension to the ACME language discussed above.  It adds aspect support to the language by extending ACME's connector element type. & Universities of Lancaster (UK), Bahia, Rio Grande do Norte and PCU of Rio de Janeiro (Brazil) & 2006 & Aspect-Oriented & Experiments & \cite{garcia2006-aspectualacme} \\

\hline
Backbone & An ADL developed with the concept of "resemblance" that provides a  modelling construct that allows an inheritance-like concept to be applied to components at all levels, providing uniform reuse and evolution support. & Imperial College, UK & 2006 & General & Examples & \cite{mcveigh2006-backbone} \\

\hline
Breeze/ADL & Breeze provides a means of describing a component and connector structure by means of an XML encoding of a graph formalism. & Shanghai Jiao Tong University, Shanghai, China & 2013 & General & Experiments & \cite{li2013-breeze}] \\

\hline
byADL & byADL's name is a contraction of "Build Your ADL" because it was created based on the premise that it isn't possible to create an all-purpose ADL and so it is better to create an extensible base upon which domain specific ADLs can be created. It has formal underpinnings and semantics and uses model driven development (MDD) technology to automatically generate software to manage and transform ADL descriptions. & Universita dell'Aquila, Italy & 2010 & General & Case Studies & \cite{ruscio2010-byadl} \\

\hline
C2SADEL & It was created to simplify the definition of architectures following the Chiron-2 ("C2") style. The language is a more sophisticated and complete ADL for the style including software tool support for it. & University of California at Irvine, USA & 1999 & Concurrent distributed systems & Experiments & \cite{medvidovic1999-c2sadel} \\

\hline
C3 & An architecture-centric approach that gives a new structure for connectors in which attachments are encapsulated within the definition of connectors. It defines and manipulates configurations as first classes entities. Also, a description of architectures from two different views, a model architecture view (logical architecture) created by the architect and an application architecture view (physical architecture instances of the logical architecture) generated automatically which serves as support to maintain the consistency and the evolution of the application architectures.  & University of Nantes, France and University Center of Souk Ahras, Algeria& 2009 & General & Examples & \cite{amirat2009-c3} \\

\hline
CBabel & An ADL developed to support the description applications that are implemented using the concurrent CR-RIO framework. It was developed at Brazilian universities and is a formal ADL, focusing on correctness, with semantics defined in rewriting logic. & Universidades Federal Fluminense and Estado do Rio de Janeiro, Brazil & 2005 & Concurrent distributed systems & Experiments & \cite{rademaker2005-cbabel} \\

\hline
CLARA & An ADL dedicated to real-time system design, which is part of REACT project. It describes the functional architecture of reactive systems and also some support for the description of the behaviour of the components and for the expression of real-time requirements (timing constraints) and properties (time budgets). & University of Nantes, France & 2004 & Real-time systems & Research Projects & \cite{faucou2005-clara} \\

\hline
DAOP-ADL & The predecessor to AO-ADL and was created to allow applications written on the DAOP platform to be easily described and the ADL is directly interpreted by the platform, so retaining the architecture of the system at runtime.  Like AO-ADL, aspects are first class architectural entities at design and runtime. & University of Malaga, Spain& 2003 & Distributed CBS & Case Studies & \cite{pinto2003-daopadl} \\

\hline
Darwin & Darwin was created to allow the creation and analysis of design specifications for distributed systems and its features include hierarchical decomposition and static and dynamic structures.  It has formal semantics, specified in the $\pi$-Calculus.  It has influenced many ADLs since, although it has not seen significant industrial usage. & Imperial College, UK & 1996 & Distributed systems & Research Projects & \cite{magee1996-darwin} \\

\hline
DiaSpec-ADL & DiaSpec-ADL addresses the needs of the pervasive systems domain.  It supports a fairly specific architectural style based on sensors, controllers and actuators, provides simulation of the architectural model and generates framework applications for developers to complete to create the application. & INRIA & 2009 & Control Systems & Research Projects & \cite{cassou2009-diaspec} \\

\hline
DPD-ADL & Aimed at the description of systems in the areas of data collection, analysis and reporting (although in fact it is still a generic component and connector language). & Tsinghua University, China & 2010 & General & Examples & [\cite{zheng2010-dpdadl} \\

\hline
DSOPL & ADL with the concept of SOA allows describing three types of information: architecture's structural elements, variability elements and system's configuration. Furthermore, it introduces context elements on which service reconfiguration is based. & University of Montpellier, France & 2015 & SOA & Examples & \cite{adjoyan2015-dsopl} \\

\hline
EAST-ADL & EAST-ADL was originally developed for the avionics industry but has broadened its scope across embedded systems and has been developed by a range of academic and industrial research centres and standardised by the SAE in the USA.  It is a component and connector based language, specialised for the needs of the embedded systems industry with strong support for product line concepts like variability.  It has been used in a number of significant industrially based research projects. & Continental Automotive, ETAS, Mentor Graphics, Volvo, University of Hull, TU Berlin, Mecel, CEA, KTH, Carmeq & 2010 & Embedded Automotive & Research Projects & \cite{cuenot2010-east} \\
\hline

FuseJ & Created to support a component model, which unified components and aspects. The language describes architectural structures following this model, which are then executed in a novel container runtime also created as part of the project. &Vrije Universiteit Brussel, Brussels, Belgium & 2005 & General & Examples & \cite{suvee2005-fusej} \\

\hline
Grasp & Created with the intent of adding traceable design rationale to architectural descriptions.  The language is a component and connector based language and was supported by quite sophisticated tooling based on Microsoft's "Oslo" project. & University of St Andrews, UK & 2011 & General & Examples & \cite{desilva2011-rationale} \\

\hline
I3 & I3 was an early attempt at adding functional semantics to a component and connector language, by using coloured petri net (CPN) semantics in the language. & University of Illinois at Chicago, USA & 1999 & General & Examples & \cite{chang1999-i3} \\

\hline
KADL & A formal architecture description language based on the Korrigan formal specification language. It was created to allow the definition of component-based systems with clear semantics, to abstract away from the details of individual component platforms like JEE and .NET. 7 & Universite d'Evry and INRIA, France & 2006 & General (for CBS) & Case Studies & \cite{poizat2006-kadl} \\

\hline
Koala & Koala is an extension of Darwin, in order to support the development of embedded systems for consumer electronics.  The primary motivation for the approach was to allow widespread reuse of components across many product configurations and so it contains product line features as well as the component based structures to describe an individual system & Phillips, Netherlands & 2000 & Embedded consumer electronic systems & Industrial Projects (100+ developers) & \cite{vanommering2000-koala} \\

\hline
LEDA & LEDA was created to try to address perceived shortcomings of earlier ADLs, such as refinement, validation and analysis, particularly for dynamic systems.  It attempts to address these shortcomings by using process algebras for the semantics of the description. & Universidad de Malaga, Spain & 1999 & General & Examples & \cite{canal1999-leda} \\

\hline
MetaH & MetaH was developed to support the development of systems in the guidance, navigation and control (GN\&C) domain.  It was paired with Control-H, a domain specific language for GN\&C, and provides the generic embedded software constructs to allow the system structure to be defined.  It has been used quite widely on industrial projects and proof-of-concepts, particular in the US military domain. & Honeywell, USA & 1996 & Embedded Systems & Industrial Projects & \cite{binns1996-metah} \\

\hline
MoDeL & MoDeL aims to provide a detailed "blueprint" for a distributed system, linking directly to the code structure and so it is closer to a module interconnection language than most of the other ADLs described here. & RWTH Aachen, Germany & 2010 & General & Case Studies & \cite{klein2000-model} \\

\hline
MontiArcHV & Developed with the aim of integrating variability management into a hierarchical component model, in an attempt to reduce the complexity of managing software product lines. & RWTH Aachen, Germany & 2011 & Interactive distributed and Cyber-Physical systems & Examples & \cite{haber2011-montiarchhv} \\

\hline
PrimitiveC-ADL & PrimitiveC provides a means to describe systems defined using the PCOM "context aware" component model, which was prior work of the same researchers.  It is a component and connector based language, using XML as its notation, with a number of novel features needed for PCOM, including contextual conditions that can affect the runtime architectural configuration. & Trinity College of Dublin, Ireland & 2010 & General (context oriented systems) & Examples & \cite{magableh2010-primitivec} \\

\hline
PRISMA AOADL & An aspect oriented ADL, to allow the description applications being built on their PRISMA research platform and it allows aspects to be used as first class architectural constructs.  PRISMA is based on a formal underlying language (OASIS), extending it with concepts like systems, components, connectors, aspects and architectural configuration. & Polytechnic University of Valencia, Spain & 2006 & General & Research Projects & \cite{perez2003-prisma} \\

\hline
Rapide & An ADL specifically for event based systems, which allows the architecture of an event based system to be rapidly defined and prototyped.  Rapide defines systems as components with well-defined interfaces that exchange events and as such does not provide first class connectors.  It has been influential in the event driven systems domain. & Stamford University, TRW Research, USA & 1995 & Event driven systems & Case Studies & \cite{luckham1995-rapide} \\

\hline
SADL & The focus is on the functional view of a system and its provably correct refinement to implementation, by allowing the formal definition of architectural structures and the relationships between them.  In particular SADL aims to provide support for the definition of architectural hierarchies. & SRI Computer Science Laboratory, USA & 1995 & General & Case Studies & \cite{moriconi1997-sadl} \\

\hline
Service-ADL & Service-ADL provides modelling elements for interaction patterns defining services, as well as for mapping sets of services to target component configurations. The language describes a comprehensive software development process that considers services as first class modelling elements. By decoupling the modelling of services from their implementation on target component configurations this process enables exploration of multiple architectures implementing the same set of services. The view of services as cross-cutting architectural aspects is substantiated by providing a mapping from services to aspects in AspectJ. & University of California, San Diego, USA & 2004 & SOA & Experiments & \cite{kruger2004-serviceadl} \\

\hline
SKwyRL-ADL & The motivation for this language's development was the development of secure multi-agent systems for distributed information systems and the lack of an ADL that met their unique needs.  The language is based on a computation model called "believe-desire-intention" that views the world as a set of independently executing autonomous agents which are each trying to achieve their goals.  Agents execute by reacting to events which are generated as a result of a change of state, goals or messages from the environment (such as other agents). & University of Louvain, Belgium & 2003 & Multi-agent systems & Experiments & \cite{Faulkner2003} \\

\hline
SOADL-EH & SOADL is an ADL for service-based systems and specifies the interfaces, behaviour, semantics and quality properties for services and allows modelling and analysis of a service-based architecture.  As the 'EH' in its name suggests, it explicitly provides error handling constructs in the ADL. & Wuhan University, China; China University of Geoscience, China & 2012 & SOA & Experiments & \cite{wang2012-soadl} \\

\hline
TADL & TADL was developed to extend a component and connector style language to include security concepts as first class constructs.  As such as well as the usual architectural primitives, it includes concepts like safety contracts and security mechanisms. & Concordia University, Canada & 2008 & General & Experiments & \cite{mohammad2008-tadl} \\

\hline
UniCon & This language implements an early component and connector model of software architecture and is one of the early attempts at introducing connectors as first class language elements.  The aim of the language was to create a useful, pragmatic and extensible test-bed that would allow the architectural abstractions used by practitioners (such as pipes, filters, objects, clients and servers) to be captured and reasoned about in a systematic manner. & Carnegie Mellon University, USA & 1995 & General & Experiments & \cite{shaw1996-unicon} \\

\hline
vADL & vADL is an ADL for product lines.  It is a component and connector based language that uses the $pi$-Calculus to define its semantics and provides explicit variability support within architectural elements. & School of Computer Science, Northwestern Polytechnic University, China & 2005 & Product Line Architecture (PLA) & Examples & \cite{zhang2005-vadl} \\

\hline
Weaves & A visual ADL developed to investigate how visual programming could be applied to large scale problems for an architectural style that constructs a software system from a directed graph of separate computational elements communicating by transferring typed objects over lightweight message queues. Weaves is particularly suited to domains that involve stream processing, such as satellite telemetry and satellite ground station prototyping. & The Aerospace Corporation and The University of Hawaii, USA & 1991 & Distributed stream processing & Case Studies & \cite{Gorlick1991} \\

\hline
Wright & A formally defined ADL to allow automated analysis of the architectural description.  Wright is a component and connector language that was developed to have well defined semantics and a set of reasoning techniques to allow architectural analysis.  It allows both architectures for individual systems and architectural styles to be described. & Carnegie Mellon University, USA & 1998 & General & Case Studies & \cite{allen1997-wright} \\

\hline
xADL & An XML based component and connector language defined as a set of XML schemas. Designed to use standard XML infrastructure and be easily extensible using standard XML-Schema extension mechanisms.  xADL was developed over an extended period, from about 2001 onwards.  The current version of xADL is 3.0 and the language tool set it still being actively enhanced at the time of writing. & University California Irvine and The SEI, USA & 2005 & General & Research Projects & \cite{dashofy2005-xadl} \\

\hline
XYZ/ADL & A formal ADL which was developed as part of a wider programme of research in the area of service oriented architecture (SOA). XYZ/ADL is a component and connector based language with formal semantics to allow it to be used as the basis of analysis techniques. Most of the literature on the language is in Chinese. & Chinese Academy of Sciences, China & 2011 & General & Examples & \cite{zhang2009-xyzadl} \\

\hline
ZETA & An ADL that focuses on component interactions, with the intent of it being used to define the composition of complex components in order to create systems from them.  The key concepts ZETA provides are components (and interfaces) and the concepts and semantics of messaging to link components. & University of Savoie at Annecy, France & 2002 & Distributed Software-intensive system & Case Studies & \cite{alloui2001-zeta} \\

\hline
$\pi$-ADL & A formal architecture description language for describing distributed and mobile systems. The language principles are that it should be a formal language, that it will focus on the runtime aspects of a system, that it should be executable and that it should be user-friendly (meaning that a number of syntaxes should be available for different uses).  As its name suggests, its semantics of the language are based on an extension of $\pi$-calculus & University of Savoie at Annecy, France  & 2004 & Distributed systems, particularly with a mobile aspect & Case Studies & \cite{Oquendo2004} \\

\hline
$\pi$-SPACE & A component and connector based language based on the $\pi$-calculus that has a focus on architectural evolution. As well as allowing the definition of components and connectors, it allows the architectural description to describe how they can be added, removed or changed during operation. & University of Savoie at Annecy, France & 2000 & General & Case Studies & \cite{chaudet2000-pispace} \\
\hline

\end{longtable}
\end{landscape}

\begin{landscape}
\setstretch{1.0}
\footnotesize
\begin{longtable}{|P{3cm}|P{2cm}|p{8cm}|p{4cm}|P{2cm}|P{2cm}|} 
\caption{ADL Support for Architectural Concepts} \label{table:adl-concepts} \\
\hline
\textbf{ADL} & \textbf{Viewpoints} & \textbf{Architectural Concepts} & \textbf{ Behavioural Semantics} & \textbf{1st Class Conns} & \textbf{1st Class Config} \endfirsthead
\caption[]{ADL Support for Architectural Concepts} \\
\hline
\textbf{ADL} & \textbf{Viewpoints} & \textbf{Architectural Concepts} & \textbf{ Behavioural Semantics} & \textbf{1st Class Conns} & \textbf{1st Class Config} \endhead
\hline

AADL & Functional, Deployment & Component (process, thread, thread group, data, subprogram, processor, memory, device, bus); Port (data port, event data port, synchronous call, direct data access); Interconnection type (message passing, event passing, synchronised data access, RPC); Property System; Information Flow; Operational Mode; Package Annex; Library (extension or specialisation) & Yes (but user defined form) & No & Yes \\
\hline

ABC/ADL & Functional, Deployment & Component; Connector; Aspect; Architecture; Configuration & Yes (but user defined form) & Yes & Yes \\
\hline

AC2-ADL & Functional & Component; Aspect Component; Connector; Aspect Connector; Architecture; Configuration & No & Yes & Yes \\

\hline

ACDL & Functional & Architecture type; Component type; Connector; Architecture (configuration) & Yes ($\pi$-Calculus) & Yes & Yes \\ 
\hline

ACME & Functional & Component; Representation-Map (to capture alternative implementation options for components); Connector; System; Template and Style & No & Yes & Yes \\ 
\hline

ADLARS & Functional, Deployment & Component; Task; System; Interaction Theme; Feature & Yes (but user defined form) & No & Yes \\ 
\hline

ADML & Functional & System; Component; Connector & Yes (built in) & Yes & No \\ 
\hline

Aesop & Functional & Component; Connector; Configuration; Architectural Style  & No & Yes & Yes \\ 
\hline

ALI & Functional & Interface; Component; Connector; System; Pattern Template; Variant Selection (based on feature catalogue)  & No & Yes & Yes \\ 
\hline

AO-ADL & Functional & Component; Connector; Aspect; Architectural Configuration  & No (although other notations can be used) & Yes & Yes \\ 
\hline

Archface & Functional, Development & Component; Connector; Aspect (pointcut \& advice) & Yes (code corresponding to the ADL structure) & Yes & No \\ 
\hline

Aspectual-ACME & Functional & Extends ACME concepts with:; Aspectual Connector & No & Yes & Yes \\ 
\hline

Backbone & Functional & Component (ports, parts [leaf component], connectors) & No & No & No \\ 
\hline

Breeze/ADL & Functional & Component; Connector; Interface; Configuration; Style constraints and state transformation & No & Yes & Yes \\ 
\hline

byADL & Functional & Components and Interfaces; Connectors and Interfaces; Architectural Configurations  & No & Yes & Yes \\ 
\hline

C2SADEL & Functional & Components and Interfaces; Connectors (but only C2 message connections); Architectural configuration; Message Filters (allowing dynamic architectures) & Yes (built in) & Yes & Yes \\ 
\hline

C3 & Functional & Component; Connector (properties, constraints, services, hierarchical level, glue and attachment); Configuration & No & Yes & Yes \\ 
\hline

CBabel & Functional & Components (“modules”); Connectors; Contracts; Architectural Configuration  & Yes (built in) & Yes & Yes \\ 
\hline

CLARA & Functional & Component; Links; Configuration & Yes (Petri-net specifications) & Yes & Yes \\ 
\hline

DAOP-ADL & Functional & Components; Properties; Aspects; Architectural Composition Rules  & No & No (implicit in the platform) & Yes \\ 
\hline

Darwin & Functional & Components; Services; Architectural Configuration (via composite components)  & No & No & Yes \\ 
\hline

DiaSpec-ADL & Functional & Components (Devices, Controllers \& Contexts); Interaction Specifications & No & No & Yes \\ 
\hline

DPD-ADL & Functional & Components; Connectors & No & Yes & Yes \\ 
\hline

DSOPL & Functional & Structural description (services [interfaces]); Variability description (variation point, reference element);; Context description (context type); Configuration description (initialization, dynamic & No & No & Yes \\ 
\hline

EAST-ADL & Functional, Dep, Development & Too many to list, very wide spectrum and detailed language, covering System, Feature, Function, Hardware and Environment modelling. & Yes (via links to descriptions like Simulink or MATLAB) & Yes & Yes \\ 
\hline

FuseJ & Functional, Deployment & Component (regular component or aspect); Interface (and Gate); Connector & No & Yes & No \\ 
\hline

Grasp & Functional & Requirement; Quality Property; Components ; Connectors; Layers; Systems; Architectures; Templates; Rationale & No & Yes & Yes \\ 
\hline

I3 & Functional & Component; Interface Net; Interconnection Net; Interoperation Net & Yes (Petri-Nets) & No & No \\ 
\hline

KADL & Functional & Components and Ports; Architectural Configuration (“glue rules”) & Yes (Korrigan) & No & Yes \\ 
\hline

Koala & Functional, Development & Components and Interfaces; Architectural Configuration & No (link to code) & No & Yes \\ 
\hline

LEDA & Functional & Components; Roles; Composition; Attachment; Adaptors (“glue” allowing construction of composites from components which are not strictly compatible). & Yes ($\pi$-Calculus) & No & No \\ 
\hline

MetaH & Functional, Concurrency, Deployment & Components (“processes”); Hardware Components; Configuration (“connections” and “modes”); Application & No (separate ControlH language) & No & Yes \\ 
\hline

MoDeL & Functional, Concurrency & Module; Subsystem; Process; Mutex & Yes (Z specifications) & No & No \\ 
\hline

MontiArcHV & Functional, Deployment & Component; Port; Connector; Variation Point; Variant; Variant Config & No & No & Yes \\ 
\hline

PrimitiveC-ADL & Functional & Components; Connectors; Design Pattern; Decision Policy; Architecture Configuration & Yes (no details provided) & Yes & Yes \\ 
\hline

PRISMA AOADL & Functional & Component; Connector; Aspect; Architectural Pattern; Architectural Configuration & Yes ($\pi$-Calculus) & Yes & Yes \\ 
\hline

Rapide & Functional, Concurrency & Components (as interfaces) and Functions, Actions and Services; Architectures (with connection rules and constraints); Events and Posets & Yes (built in) & No & Yes \\ 
\hline

SADL & Functional & Components and Interfaces (containing ports); Connectors; (Architectural) configurations; Mappings (between architectures); Architectural Styles; Refinement Patterns & No & Yes & Yes \\ 
\hline

Service-ADL & Functional, Development & Roles (states); Services (interaction); Component (plays, in service); Configuration & Yes & No & Yes \\ 
\hline

SKwyRL-ADL & Functional & Agents (as components); Ports (sensors or effectors) ; Knowledge Base; Goals and Plans; Services; Architectural Configuration; Architecture & No & No & Yes \\ 
\hline

SOADL-EH & Functional, Deployment & Components; Connectors; Service; Policy; Configuration & Yes ($\pi$-calculus, code) & Yes & Yes \\ 
\hline

TADL & Functional, Deployment & Interface Type; Connector Role Type; Connector Type; Architecture Type; Component Type & No & Yes & Yes \\ 
\hline

UniCon & Functional & Components with Interfaces; Connectors with Protocols; Properties; Architectural Configuration (via composite components) & No (link to code) & Yes & Yes \\ 
\hline

vADL & Functional & Component; Connector; Interface; Behaviour; Architecture & Yes ($\pi$-calculus) & No & No \\ 
\hline

Weaves & Functional & Components (“tool fragments”) and Ports; Connectors (message queues); Annotations & No (but link to code) & Yes & No \\ 
\hline

Wright & Functional & Components and Interfaces; Connectors and Roles; Architectural Configurations (“attachments”); Styles & Yes (CSP extension) & Yes & Yes \\ 
\hline

xADL & Functional, Deployment & Components and Interfaces; Connectors and Interfaces; Architectural Configuration (“links”); Runtime Structure (component instances, link instances) & No & Yes & Yes \\ 
\hline

XYZ/ADL & Functional & Components and Interfaces; Connectors and Roles; Architectural configurations (via composite components) & Yes (XYZ/E temporal logic language) & Yes & Yes \\ 
\hline

ZETA & Functional & Interaction Interface (Activity Interface, Artefact Interface, Message Types, Port Interface, Protocol Interface); Interaction; Message; Port; Attachment; Container Types & Yes (built in) & Yes & No \\ 
\hline

$\pi$-ADL & Functional & Components and Ports; Agents (mobile components); Connectors and Ports; Architecture (configuration) & Yes ($\pi$-calculus) & Yes & Yes \\ 
\hline

$\pi$-SPACE & Functional & Port Type; Behaviour Type; Component Type; Connector Type; Composition/ Decomposition/ Recomposition & Yes ($\pi$-calculus) & Yes & Yes \\ 
\hline


\end{longtable}
\end{landscape}


\begin{landscape}
\setstretch{1.0}
\footnotesize
\begin{longtable}{|c|P{2cm}|P{3cm}|P{3cm}|p{2cm}|c|} 
\caption{ADL Language Mechanisms and Support} \label{table:adl-mechanisms} \\
\hline
\textbf{ADL} & \textbf{Structuring} & \textbf{Capturing Qualities} & \textbf{Syntax} & \textbf{Analysis} & \textbf{Tools} \endfirsthead
\caption[]{ADL Language Mechanisms and Support} \\
\hline
\textbf{ADL} & \textbf{Structuring} & \textbf{Capturing Qualities} & \textbf{Syntax} & \textbf{Analysis} & \textbf{Tools} \endhead
\hline

AAADL & Packages; Composition & Properties & Textual; Graphical & Yes (Performance; Real-Time Scheduling; Reliability); others via extensions & Commercial \\
\hline

ABC/ADL & Composition & Properties & Textual; Graphical & Via 3rd party tools & Prototype \\
\hline

AC2-ADL & Composition & None & Textual & None & None \\
\hline
ACDL & Composition & None & Textual & Yes (Consistency) & None \\ 
\hline

ACME & Composition & Properties & Textual; Graphical & Via ADL extensions or 3rd party tools & Research \\ 
\hline

ADLARS & Composition & Specific Attributes (for Timing) & Textual (JavaCC); Graphical & Yes (Concurrency; Consistency) & Prototype \\ 
\hline

ADML & Composition & Properties & Textual & None & None \\ 
\hline

Aesop & Composition & Style Properties & Textual; Graphical & Via 3rd party tools & Research \\ 
\hline

ALI & Composition & None & Textual; Graphical & None & None \\ 
\hline

AO-ADL & Composition & None & Textual; Graphical & None & Research \\ 
\hline

Archface & None & None & Textual & Via 3rd party tools & Prototype \\ 
\hline

Aspectual-ACME & Composition & Properties & Textual; Graphical & Via 3rd party tools & None \\ 
\hline

Backbone & Composition & None & Textual; Graphical & None & Prototype \\ 
\hline

Breeze/ADL & Packages; Composition & None & Textual (XML-based) & None & None \\ 
\hline

byADL & Packages; Composition & Properties & Various & Via ADL extensions & Research \\ 
\hline

C2SADEL & None & None & Textual; Graphical & Via ADL extensions & Research \\ 
\hline

C3 & Composition & Properties & Textual; Graphical (UML) & None & None \\ 
\hline

CBabel & None & Language QoS Contracts; Properties & Textual & Yes (Concurrency; Consistency) & None \\ 
\hline

CLARA & Composition & Specific Attributes (for Timing) & Textual; Graphical & Yes (Concurrency; Consistency) & None \\ 
\hline

DAOP-ADL & None & None & Textual (XML); Graphical & None & Research \\ 
\hline

Darwin & Composition & Properties & Textual; Graphical & Via 3rd party tools & Research \\ 
\hline

DiaSpec-ADL & None & None & Textual & Yes (Consistency) & Research \\ 
\hline

DPD-ADL & Composition & None & Textual; Graphical (UML) & None & None \\ 
\hline

DSOPL & Packages; Composition & None & Textual (XML); Graphical (UML) & None & None \\ 
\hline

EAST-ADL & Packages; Composition (inherited from UML) & Language Elements (for Timing, Safety, Dependability); Properties & Graphical notation, based on and extending UML (i.e. graphical notation and semi-structured natural language) & Via 3rd party tools & Commercial \\ 
\hline

FuseJ & Packages; Composition & None & Textual & None & Prototype \\ 
\hline

Grasp & Composition & Specific Attributes; Properties & Textual & None & Prototype \\ 
\hline

I3 & None & None & Loosely defined graphical notation; Petri net representations & Yes (Concurrency) & None \\ 
\hline

KADL & Composition & None & Textual; Graphical & Yes (Consistency) & Prototype \\ 
\hline

Koala & Packages; Composition & None & Textual; Graphical & None & Commercial \\ 
\hline

LEDA & Composition & None & Textual & Yes (Consistency) & None \\ 
\hline

MetaH & Composition & Specific Attributes (for Timing, Safety, Dependability) & Textual; Graphical & Yes (Real-Time Scheduling) & Commercial \\ 
\hline

MoDeL & Subsystems; Composition & None & Textual; Graphical & None & Research \\ 
\hline

MontiArcHV & Packages; Composition & None & Textual; Graphical. & None & None \\ 
\hline

PrimitiveC-ADL & Composition & None & Textual (XML based) & None & None \\ 
\hline

PRISMA AOADL & Composition & None & Textual; UML Profile & None & Research \\ 
\hline

Rapide & Composition & None & Textual; Simplified graphical & Yes (Correctness; Timing) & Research \\ 
\hline

SADL & Composition & None & Textual & None & Research \\ 
\hline

Service-ADL & Packages; Composition & None & Textual; Graphical & None & None \\ 
\hline

SKwyRL-ADL & Composition & Language Elements (for Security - security constraints, mechanisms, etc.) & Textual & None & None \\ 
\hline

SOADL-EH & Packages; Composition & Properties & Textual (XML), Graphical & Yes (Concurrency; Consistency) & None \\ 
\hline

TADL & Packages; Composition & Language Elements (for Safety and Security, safety constraints, security structures); Properties & Textual & Yes (Safety; Timing; Security) & None \\ 
\hline

UniCon & Composition & Properties & Textual; Graphical & Via 3rd party tools & Prototype \\ 
\hline

vADL & Composition & Properties & Textual & None & None \\ 
\hline

Weaves & Composition & Properties & Graphical & Yes (Performance) & Research \\ 
\hline

Wright & Composition & None & Textual & Yes (Correctness) & Prototype \\ 
\hline

xADL & Composition & Properties (via extension) & XML (and tool specific graphical) & None & Research \\ 
\hline

XYZ/ADL & Composition & None & Textual (with mathematics) & Yes (Consistency) & None \\ 
\hline

ZETA & Composition & None & Textual; Graphical & None & None \\ 
\hline

$\pi$-ADL & Composition & Properties & Textual; Graphical (UML extension) & Yes (Correctness; Consistency) & Research \\ 
\hline

$\pi$-SPACE & Composition & None & Textual; Graphical (UML) & None & None \\ 
\hline

\end{longtable}
\end{landscape}



%\input{Appendices/AppendixB} % Appendix Title

%\input{Appendices/AppendixC} % Appendix Title

%\addtocontents{toc}{\vspace{2em}}  % Add a gap in the Contents, for aesthetics
\backmatter

%% ----------------------------------------------------------------
\label{Bibliography}
\lhead{\emph{Bibliography}}  % Change the left side page header to "Bibliography"
\bibliographystyle{unsrtnat}  % Use the "unsrtnat" BibTeX style for formatting the Bibliography
\bibliography{Bibliography}  % The references (bibliography) information are stored in the file named "Bibliography.bib"

\end{document}  % The End
%% ----------------------------------------------------------------