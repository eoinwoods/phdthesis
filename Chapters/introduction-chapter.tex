\chapter{Introduction}
\label{chapter:introduction}

\section{Motivation}

IT activity has entered a new era. In 2008, the number of things connected to the internet exceeded the number of people \cite{evans2008-iotinfo}, while Cisco predicted that in 2011, 20 typical households would generate more traffic than the entire internet in 2008 \cite{evans2008-iotinfo}. We are witnessing an explosive growth in data driven by more affordable storage systems and the proliferation of mobile, IoT, social media, and smart cities to name a few. Add to this the software applications being created around this data, from business analytics to apps that can monitor your cow's pregnancy through a sensor connected to its tail and send you an SMS when she is most likely to calf. The vast majority of the applications being developed are cloud-based, meaning constantly increasing demand on data centres, where many of the large Tech companies are continuously expanding their capacity. 

Currently, data centres consume a substantial amount of energy and are thought to produce more Greenhouse Gas emissions than the entire aviation sector. In 2013, data centres in the U.S. alone consumed an estimated 91 billion KWh of electricity, and this is expected to rise to 140 billion KWh by 2020 \cite{delforge2014-datacentreenergy}. It is not surprising therefore that a recent survey of data centre managers showed that power density and energy efficiency are among their top current and future concerns (Intel Corporation, 2012) \textit{DO WE HAVE THIS?}. This is particularly important for Colocation and Managed Service data centre providers who are based within or close to large cities where grid power availability is limited. 

So, can data centres cope with this demand and continue to support the exponential growth of the big data and cloud revolution? We cannot be sure, but we can see that the challenges are widely recognised: a number of related areas are the subjects of active research, ranging from new cooling technologies to more energy efficient servers and building designs, to runtime workload consolidation and management techniques. However, the software architecture community has been slower to recognise their potential contribution and to mobilise to meet this challenge. Addressing energy efficiency at the architecture level is still far from being mainstream. Can we continue designing systems without any consideration of their energy and power efficiency and let others worry about running them in an energy efficient way? Should energy efficiency be a bolt-on system property, or a quality attribute that is addressed at the design stage?

Software architects may not be prioritising energy efficiency for a number of reasons. Firstly, we currently have very little understanding of the impact of design decisions on energy efficiency or an understanding of how it affects other system qualities such as user experience, reliability and performance.  Without this knowledge, it is difficult to perform trade-off analysis to understand the benefit or cost of improving energy efficiency. Minor changes to the system design could yield substantial benefits, such as avoiding unnecessary polling or eliminating redundant housekeeping tasks that prevent equipment from entering lower-power states. However, a lack of relevant design tools and frameworks mean that it is still difficult to achieve more sophisticated optimisations that include consideration of contextual information about the runtime environment. 

Second, in order to achieve the next order of magnitude in energy efficiency, we need to think outside traditional design boundaries. This will require people from different specialisations and departments to work together. This can often be difficult to achieve, given current organisational software governance structures, with different teams sometimes having competing objectives, not to mention human dynamics and political barriers. Current technologies also provide few mechanisms to allow communication across different technology layers (that is, the application software, middleware, hardware, network, cooling, power infrastructure, etc.) which would enable cross-layer optimization.

Finally, so far, energy efficiency rarely features as an end-user requirement or concern. On one hand, there is the problem of split incentives where operators of systems (e.g. administrators or data centre managers) do not pick up the energy bill (which tend to come out from the facilities budget). Accordingly, they would see very little return from any savings made from energy efficiency. On the other hand, ICT energy costs, given current energy prices, do not constitute more than 1\% to 3\% of a typical organisation’s budget. So, when efficiency is pursued, it is easier to achieve it by addressing areas with larger budget share (e.g. payroll!). This problem is exacerbated by the lack of benchmarks, metrics and reliable data to allow realistic comparisons of different energy efficiency opportunities and their realistic returns.

\section{Research Questions}

This work aims to answer four specific research questions, namely:

\begin{description}
\item [RQ1] What ADLs exist and can they be used to reason about the energy properties of a system?
\item [RQ2] How can architects prioritise their attention on energy efficiency?
\item [RQ3] What design guidelines can we provide to guide architects to improve energy efficiency of their systems?
\item [RQ4] How can we raise awareness of application energy consumption during operation?
\end{description}

In order to answer these questions, the work has involved the investigation of the use of ADLs for large scale architectural description, the identification of practical advice for how architects can focus attention on critical topics (such as energy efficiency) and the creation of a practical approach for estimating the energy usage of an application processing inbound requests.

\section{Research Methodology}

\section{Contribution}

\section{Structure of Thesis}

This thesis is structured into 9 chapters, each presenting a specific aspect of the research work.

\emph{Chapter 1} is this introductory chapter, setting the motivation and context for the work, defining the research questions, explaining the research approach and explaining the structure of the thesis.

\emph{Chapter 2} contains a literature review, structured into four parts, exploring the research literature in the areas of architectural description languages (ADLs), how architects should prioritise their focus for maximum effectiveness, design guidelines for energy efficiency and operational energy consumption for IT systems.

\emph{Chapter 3} discusses how ADLs can be used to describe large scale software systems and presents a significant industrial case study that explored how effective this was in practice.

\emph{Chapter 4} explores the area of prioritisation of work, specficially from an architect's perspective, asking the question how architects prioritise their time for maximum effectiveness and presenting the results of an industrial survey into the approaches used by experienced practitioners and a validated model to guide architects, based on the insights from the survey.

\emph{Chapter 5} presents heuristic design principles for designing energy efficient software applications.

\emph{Chapter 6} explores the topic of how application energy usage can be monitored and estimated during the operation of a system and presents a theoretical model for solving this problem.

\emph {Chapter 7} presents a practical implementation of the energy estimation model, specialised to estimating energy usage of a group of microservices processing incoming requests.

\emph{Chapter 8} presents the work performed to validate the model implementation technology, which involved running a number of non-trival scenarios and using the tool to estimate their energy usage, while also deriving the same estimation through a separate, independent technique, using these secondary estimations to validate the outputs of the model and the tool.

\emph {Chapter 9} summarises the research work, draws conclusions from it to answer the research questions and discusses possible future research work in each area.



